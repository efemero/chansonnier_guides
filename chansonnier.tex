% Auth: Nicklas Vraa
% Docs: https://github.com/NicklasVraa/LiX
% Everything you need to know about this template is found in on the github repository above. Stars are very appreciated.

\documentclass{novel}

\usepackage{tocloft}

\lang      {french}
\title     {Chansonnier}
\subtitle  {Guides 31\textsuperscript{ème} unité Saint-François de Gembloux\\ Compagnie des Quatre Vents}
\cover     {resources/trefles_front.jpg}{resources/trefles_back.jpg}
\author    {}
\license   {CC}{by-sa}{4.0}{La licence ne se rapporte qu’à la structure du chansonnier, les différentes chansons gardent leur copyright original.}
%\edition   {1}{2024}
\size{a5}
\margins{15mm}

%\note{Lorem ipsum dolor sit amet, consectetur adipiscing elit. Praesent porttitor est arcu, sed euismod metus imperdiet iaculis. Quisque vestibulum molestie nulla, non consectetur tellus mollis a. Nunc commodo magna a elit commodo dignissim.}


\begin{document}
\pagenumbering{gobble}

\renewcommand{\cftchapleader}{\cftdotfill{\cftdotsep}}
\toc


\h*{Chant de rassemblement}
\setcounter{page}{1}
\pagenumbering{arabic}


Nous sommes les guides de Gembloux, guides de Gembloux\\
Qui sommes toujours prêtes à tout, oui prêtes à tout\\
On est ici pour s’amuser, pour s’amuser,\\
Et passer une super année, super année\\
Année, année, année, année, année.\\

Se découvrir et s’enrichir, et s’enrichir,\\
Se retrouver avec plaisir, avec plaisir\\
Penser déjà au prochain camp, au prochain camp,\\
Qui sera bien c’est évident, c’est évident,\\
Au camp, au camp, au camp, au camp, au camp.\\


Animées par not’ super staff, not’ super staff,\\
On n’a pas d’autre rime en “aff”, d’autre rime en “aff”,\\
On ne se doit plus qu’d’apporter, plus qu’d’apporter,\\
Notre bonne humeur et notre gaité et notre gaité,\\
Gaité, gaité, gaité, gaité, gaité.\\


La chanson va se terminer, se terminer,\\
Mais il faut pas vous inquiéter, vous inquiéter,\\
On va vous la chanter encore, chanter encore,\\
Et la recommencer plus fort, encore plus FORT,\\
PLUS FORT, PLUS FORT, PLUS FORT, PLUS FORT, PLUS FORT !

\newpage
\small {
\vbox{
    \begin{minipage}[t][0.4\textheight][t]{\textwidth}
    \vspace{0.05\textheight}
        \h*{Cantique des patrouilles}

\begin{multicols}{2}
Seigneur, rassemblées près des tentes\\
Pour saluer la fin du jour\\
Tes guides laissent leur voix chantantes\\
Monter vers Toi, pleines d'amour\\
Tu dois aimer l'humble prière\\
Qui de ce camp s'en va monter\\
Ô Toi qui n'avais sur la terre\\
Pas de maison pour t'abriter\\

\begin{bfseries}
[Refrain:]\\
Nous venons toutes les patrouilles\\
Te prier pour Te servir mieux\\
Vois au bois silencieux\\
Tes guides qui s'agenouillent\\
Bénis-les, ô Jésus dans les cieux\\
\end{bfseries}
\columnbreak

Merci de ce jour d'existence\\
Où ta bonté nous conserva\\
Merci de ta sainte présence\\
Qui de tout mal nous préserva\\
Merci du bien fait par les guides\\
Merci des conseils reçus\\
Merci de l'amour qui nous groupe\\
Comme des sœurs, ô Jésus.\\

\textbf{[Refrain]}\\

\end{multicols}
    \end{minipage}

    \nointerlineskip
    \begin{minipage}[b][0.55\textheight][t]{\textwidth}
    \vspace{0.2\textheight}

\h*{Chant de la promesse}
\begin{multicols}{2}
\begin{enumerate}
\item Devant tous je m'engage\\
Sur mon honneur\\
Et je te fais hommage\\
De moi, Seigneur.\\

\begin{bfseries}
[Refrain]:\\
Je veux t'aimer sans cesse\\
De plus en plus\\
Protège ma Promesse\\
Seigneur Jésus.\\
\end{bfseries}

\item Je jure de te suivre\\
En fier chrétien\\
Et tout entier je livre\\
Mon cœur au tien\\

\item Fidèle à ma Patrie\\
Je le serai\\
Tous les jours de ma vie\\
Je servirai.\\


\item Je suis de tes apôtres\\
Et chaque jour\\
Je veux aider les autres\\
Pour ton amour\\

\item Ta Règle a sur nous-mêmes\\
Un droit sacré.\\
Je suis faible tu m'aimes\\
Je maintiendrai.\\
\end{enumerate}
\end{multicols}
    \end{minipage}
}
}
\newpage

\h*{Il est libre Max}
Il met de la magie, mine de rien, dans tout ce qu'il fait\\
Il a le sourire facile, même pour les imbéciles\\
Il s'amuse bien, il n'tombe jamais dans les pièges\\
Il n'se laisse pas étourdir par les néons des manèges\\
Il vit sa vie sans s'occuper des grimaces\\
Que font autour de lui les poissons dans la nasse\\

Il est libre Max ! Il est libre Max !\\
Y'en a même qui disent qu'ils l'ont vu voler\\

Il travaille un p'tit peu quand son corps est d'accord\\
Pour lui faut pas s'en faire, il sait doser son effort\\
Dans l'panier de crabes, il joue pas les homards\\
Il n'cherche pas à tout prix à faire des bulles dans la mare\\

Il r'garde autour de lui avec les yeux de l'amour\\
Avant qu't'aies rien pu dire, il t'aime déjà au départ\\
Il n'fait pas de bruit, il n'joue pas du tambour\\
Mais la statue de marbre lui sourit dans la cour\\

Et bien sûr toutes les filles lui font les yeux de velours\\
Lui, pour leur faire plaisir, il leur raconte des histoires\\
Il les emmène par delà les labours\\
Chevaucher des licornes à la tombée du soir\\

Comme il n'a pas d'argent pour faire le grand voyageur\\
Il va parler souvent aux habitants de son cœur\\
Qu'est ce qu'ils s'racontent, c'est ça qu'il faudrait savoir\\
Pour avoir comme lui autant d'amour dans le regard\\

Il est libre Max ! Il est libre Max !\\
Y'en a même qui disent qu'ils l'ont vu voler

\newpage
\normalsize
\h*{The lion sleeps tonight}

Ah-wee-ooh-wee-ooo, wee--ee-ee-ee-oo, we-um-um-a-way\\
Wee-ooh-wee-ooo, wee--ee-ee-ee-oo, we-um-um-a-way\\

Awimobawe, awimbawe, awimbawe, awimbawe, awimobawe, awimbawe, awimbawe

In the jungle the mighty jungle the lion sleeps tonight\\
In the jungle the quiet jungle the lion sleeps tonight

Woo-oo-oo-ooo, we-um-um-a-way\\
Awimobawe, awimbawe, awimbawe, awimbawe, awimobawe, awimbawe, awimbawe

Near the village the peaceful village the lion sleeps tonight\\
Near the village the quiet village the lion sleeps tonight

Woo-oo-oo-ooo, we-um-um-a-way\\
Awimobawe, awimbawe, awimbawe, awimbawe, awimobawe, awimbawe, awimbawe

Hush my darling, don't fear my darling. The lion sleeps tonight.\\
Hush my darling, don't fear my darling. The lion sleeps tonight

Awimobawe, awimbawe, awimbawe, awimbawe, awimobawe, awimbawe, awimbawe

\newpage
\scriptsize{

\vbox{
    \begin{minipage}[t][0.4\textheight][t]{\textwidth}
    \vspace{0.05\textheight}
\h*{Tous les cris, les S.O.S.}

\begin{multicols}{2}

Comme un fou va jeter à la mer\\
Des bouteilles vides et puis espère\\
Qu'on pourra lire à travers, S.O.S. écrit avec de l'air\\
Pour te dire que je me sens seul\\
Je dessine à l'encre vide, un désert\\

Et je cours, je me raccroche à la vie\\
Je me saoule avec le bruit des corps qui m'entourent\\
Comme des lianes nouées de tresses\\
Sans comprendre la détresse, des mots que j'envoie\\

Difficile d'appeler au secours\\
Quand tant de drames nous oppressent\\
Et les larmes nouées de stress\\
Étouffent un peu plus les cris d'amour\\
De ceux qui sont dans la faiblesse\\
Et dans un dernier espoir, disparaissent\\

Et je cours, je me raccroche à la vie\\
Je me saoule avec le bruit des corps qui m'entourent\\
Comme des lianes nouées de tresses\\
Sans comprendre la détresse, des mots que j'envoie\\

\begin{bfseries}
[Refrain]:\\
Tous les cris les S.O.S., partent dans les airs\\
Dans l'eau laissent une trace, dont les écumes font la beauté\\
Pris dans leur vaisseau de verre, les messages luttent\\
Mais les vagues les ramènent\\
En pierres d'étoiles sur les rochers\\
J'ai ramassé les bouts de verre, j'ai recollé tous les morceaux\\
Tout était clair comme de l'eau\\
Contre le passé y a rien à faire, il faudrait changer les héros\\
Dans un monde où le plus beau, reste à faire\\
\end{bfseries}

Et je cours, je me raccroche à la vie\\
Je me saoule avec le bruit des corps qui m'entourent\\
Comme des lianes nouées de tresses\\
Sans comprendre la détresse, des mots que j'envoie\\

\textbf{[Refrain]}

\end{multicols}
\end{minipage}

\nointerlineskip
\begin{minipage}[b][0.55\textheight][t]{\textwidth}
\vspace{0.15\textheight}
\h*{Personne}

\begin{multicols}{2}
J'avais perdu l'habitude, des clés de la solitude\\
J'avais perdu l'amer et les déserts arides\\
Même la chaleur des pull-overs\\
J'avais perdu l'enfer, au paradis...\\

J'avais oublié les refrains, qui nous rappellent à l'ordre\\
Et ton foutu désordre, ce désordre essentiel\\
Mais si confidentiel, l'existence et les roses se fanent\\
Même un lundi, au paradis...\\

\begin{bfseries}
[Refrain]: (2x)\\
Persooonne, ne te remplace\\
Non personne, ne te remplace\\
\end{bfseries}
\columnbreak

C'est un enfer à vivre, mais comment vivre avec\\
Mes envies insensées\\
Car ton armoire est vide, mes rêves me dévorent\\
Et mes draps sont glacés, toutes les nuits...\\

On a plus goût à rien, mais tant besoin de tout\\
C'qui pourrait remplacer un être indélébile\\
On cherche en vain le double, on serait prêt à tout\\
Pour revoir le jour, toutes les nuits...\\

\textbf{[Refrain] (3x)}

\end{multicols}
\end{minipage}
}
}

\newpage
\scriptsize{

\vbox{
    \begin{minipage}[t][0.4\textheight][t]{\textwidth}
\h*{I will survive}

\begin{multicols}{2}

At first I was afraid, I was petrified\\
Kept thinkin' I could never live without you by my side\\
But then I spent so many nights thinkin' how you did me wrong\\
And I grew strong and I learned how to get along\\

And now you're back from outer space\\
And I find you here with that sad look upon your face\\
I should have changed that stupid lock\\
I should have made you leave your key\\
If I'd have known for just one second you'd be back to bother me\\

\begin{bfseries}
[Refrain]:
Go on now, go, walk out the door, just turn around now\\
Cause you're not welcome anymore\\
You’re the one who tried to hurt me with goodbye\\
Did you think I'd crumble, did you think I'd lay down and die?\\
Oh no I will survive\\
Oh, as long as I know how to love I know I'll stay alive\\
I've got all my life to live, I've got all my love to give\\
And I'll survive, I will survive\\
Hey hey\\
\end{bfseries}

It took all the strength I had not to fall apart\\
Kept trying' hard to mend the pieces of my broken heart\\
And I spent oh so many nights, just feeling sorry for myself\\
I used to cry\\
But now I hold my head up high and you see in me somebody new\\
Not that chained up little person still in love with you\\
And so you feel like droppin' in, and just expect me to be free\\
But now I'm savin' all my lovin' for someone who's lovin' me\\

\textbf{[Refrain] (2x)}

\end{multicols}
\end{minipage}

\nointerlineskip
\begin{minipage}[b][0.55\textheight][t]{\textwidth}
\vspace{0.12\textheight}
\h*{Hotel California}

\begin{multicols}{2}
On a dark desert highway, cool wind in my hair\\
Warm smell of colitas rising up through the air\\
Up ahead in the distance, I saw a shimmering light\\
My head grew heavy, and my sight grew dimmer\\
I had to stop for the night\\
There she stood in the doorway;\\
I heard the mission bell\\
And I was thinking to myself,\\
'This could be Heaven or this could be Hell'\\
Then she lit up a candle and she showed me the way\\
There were voices down the corridor, I thought I heard them say...\\

Welcome to the Hotel California\\
Such a lovely place (such a lovely flace)\\
Plenty of room at the Hotel California\\
Any time of year, you can find it here\\

Her mind is Tiffany-twisted, She got the Mercedes Benz\\
She's got a lot of pretty, pretty boys, that she calls friends\\
How they dance in the courtyard, sweet summer sweat.\\
Some dance to remember, some dance to forget\\
So I called up the Captain, 'Please bring me my wine'\\
He said, 'We haven't had that spirit here since 1969'\\
And still those voices are calling from far away\\
Wake you up in the middle of the night\\
Just to hear them say...\\

Welcome to the Hotel California\\
Such a lovely Place (such a lovely face)\\
They livin' it up at the Hotel California\\
What a nice surprise, bring your alibis\\
Mirrors on the ceiling, the pink champagne on ice\\
And she said 'We are all just prisoners here, of our own device'\\
And in the master's chambers, they gathered for the feast\\
They stab it with their steely knives, but they just can't kill the beast\\
Last thing I remember, I was running for the door\\
I had to find the passage back to the place I was before\\
'Relax' said the nightman, We are programed to receive.\\
You can check out any time you like, but you can never leave\\

\end{multicols}
\end{minipage}
}
}

\newpage
\normalsize

\h*{Pour que tu m’aimes encore}
\begin{multicols}{2}

J'ai compris tous les mots, j'ai bien compris, merci\\
Raisonnable et nouveau, c'est ainsi par ici\\
Que les choses ont changé, que les fleurs ont fané\\
Que le temps d'avant, c'était le temps d'avant\\
Que si tout zappe et lasse, les amours aussi passent\\

Il faut que tu saches\\

\begin{bfseries}
[Refrain]:\\
J'irai chercher ton cœur si tu l'emportes ailleurs\\
Même si dans tes danses d'autres dansent tes heures\\
J'irai chercher ton âme dans les froids dans les flammes\\
Je te jetterai des sorts pour que tu m'aimes encore\\
\end{bfseries}

Fallait pas commencer m'attirer, me toucher\\
Fallait pas tant donner moi je sais pas jouer\\
On me dit qu'aujourd'hui, on me dit que les autres font ainsi\\
Je ne suis pas les autres\\
Avant que l'on s'attache, avant que l'on se gâche\\
\columnbreak

Je veux que tu saches\\

\textbf{[Refrain]}\\

Je trouverai des langages pour chanter tes louanges\\
Je ferai nos bagages pour d'infinies vendanges\\
Les formules magiques des marabouts d'Afrique\\
J'les dirais sans remords pour que tu m'aimes encore\\

Je m'inventerai reine pour que tu me retiennes\\
Je me ferai nouvelle pour que le feu reprenne\\
Je deviendrai ces autres qui te donnent du plaisir\\
Vos jeux seront les nôtres, si tel est ton désir\\

Plus brillante plus belle pour une autre étincelle\\
Je me changerai en or pour que tu m'aimes encore.\\
\end{multicols}

\newpage
\normalsize

\h*{Mon frère}
\begin{multicols}{2}

Toi le frère que je n'ai jamais eu\\
Sais-tu si tu avais vécu\\
Ce que nous aurions fait ensemble\\
Un an après moi, tu serais né\\
Alors on n'se s'rait plus quittés\\
Comme deux amis qui se ressemblent\\
On aurait appris l'argot par cœur\\
J'aurais été ton professeur\\
A mon école buissonnière\\
Sur qu'un jour on se serait battu\\
Pour peu qu'alors on ait connu\\
Ensemble la même première\\

\begin{bfseries}
[Refrain]:\\
Mais tu n'es pas la\\
A qui la faute\\
Pas à mon père\\
Pas à ma mère\\
Tu aurais pu chanter cela\\
\end{bfseries}

\columnbreak
Toi le frère que je n'ai jamais eu\\
Si tu savais ce que j'ai bu\\
De mes chagrins en solitaire\\
Si tu m'avais pas fait faux bond\\
Tu aurais fini mes chansons\\
Je t'aurais appris à en faire\\
Si la vie s'était comportée mieux\\
Elle aurait divisé en deux\\
Les paires de gants, les paires de claques\\
Elle aurait sûrement partagé\\
Les mots d'amour et les pavés\\
Les filles et les coups de matraque\\

\textbf{[Refrain]}\\

Toi le frère que je n'aurais jamais\\
Je suis moins seul de t'avoir fait\\
Pour un instant, pour une fille\\
Je t'ai dérangé, tu me pardonnes\\
Ici quand tout vous abandonne\\
On se fabrique une famille\\
\end{multicols}

\newpage
\normalsize

\h*{Mon fils, ma bataille}

\begin{multicols}{2}
Ça fait longtemps que t'es partie, maintenant\\
Je t'écoute démonter ma vie, en pleurant\\
Si j'avais su qu'un matin\\
Je serai là, sali, jugé, sur un banc\\
Par l'ombre d'un corps\\
Que j'ai serré si souvent\\
Pour un enfant\\

Tu leur dis que mon métier, c'est du vent\\
Qu'on ne sait pas ce que je serai\\
Dans un an\\
S'ils savaient que pour toi\\
Avant de tous les chanteurs j'étais le plus grand\\
Et que c'est pour ça\\
Que tu voulais un enfant\\
Devenu grand\\

\columnbreak
\begin{bfseries}
[Refrain]:\\
Les juges et les lois\\
Ça m'fait pas peur\\
C'est mon fils ma bataille\\
Fallait pas qu'elle s'en aille\\
Je vais tout casser\\
Si vous touchez\\
Au fruit de mes entrailles\\
Fallait pas qu'elle s'en aille\\
\end{bfseries}

Bien sûr c'est elle qui l'a porté\\
Et pourtant\\
C'est moi qui lui construis sa vie lentement\\
Tout ce qu'elle peut dire sur moi\\
N'est rien à côté du sourire qu'il me tend\\
L'absence a ses torts\\
Que rien ne défend\\
C'est mon enfant\\

\textbf{[Refrain] (2x)}
\end{multicols}

\newpage
\h*{Le lundi au soleil}

\begin{multicols}{2}
Regarde ta montre\\
Il est déjà huit heures\\
Embrassons-nous tendrement\\
Un taxi t'emporte\\
Tu t'en vas, mon cœur\\
Parmi ces milliers de gens\\
C'est une journée idéale\\
Pour marcher dans la forêt\\
On trouverait plus normal\\
D'aller se coucher\\
Seuls dans les genêts\\

\begin{bfseries}
[Refrain:]\\
Le lundi au soleil\\
C'est une chose qu'on n'aura jamais\\
Chaque fois c'est pareil\\
C'est quand on est derrière les carreaux\\
Quand on travaille que le ciel est beau\\
Qu'il doit faire beau sur les routes\\
Le lundi au soleil\\

Le lundi au soleil\\
On pourrait le passer à s'aimer\\
Le lundi au soleil\\
On serait mieux dans l'odeur des foins\\
On aimerait mieux cueillir le raisin\\
Ou simplement ne rien faire\\
Le lundi au soleil\\
\end{bfseries}

Toi, tu es à... l'autre bout\\
De cette ville\\
Là-bas, comme chaque jour\\
Les dernières heures\\
Sont les plus difficiles\\
J'ai besoin de ton amour\\
Et puis dans la foule au loin\\
Je te vois, tu me souris\\
Les néons des magasins\\
Sont tous allumés\\
C'est déjà la nuit\\

\textbf{[Refrain]}\\
\end{multicols}

\newpage
\normalsize

\h*{It’s not because you are}

\begin{multicols}{2}
When I have rencontred you,\\
You was a jeune fille au pair,\\
And I put a spell on you,\\
And you roule a pelle to me.\\

Together we go partout\\
On my mob il was super\\
It was friday on my mind,\\
It was story d'amour.\\

\begin{bfseries}
[Refrain:]\\
It is not because you are,\\
I love you because I do\\
C'est pas parc' que you are me qu'I am you.\\
\end{bfseries}

You was really beautiful\\
In the middle of the foule.\\
Don't let me misunderstood,\\
Don't let me sinon I boude.\\

My loving, my marshmallow,\\
You are belle and I are beau\\
You give me all what You have\\
I say thank you, you are bien brave.\\

\textbf{[Refrain]}\\

I wanted marry with you,\\
And make love very beaucoup,\\
To have a max of children,\\
Just like Stone and Charden.\\

But one day that must arrive,\\
Together we disputed.\\
For a stupid story of fric,\\
We decide to divorced.\\

\textbf{[Refrain]}\\

You chialed comme une madeleine,\\
Not me, I have my dignité.\\
You tell me : you are a sale mec !\\
I tell you : poil to the bec !\\

That's comme ça that you thank me\\
To have learning you english ?\\
Eh ! That's not you qui m'a appris,\\
My grand father was rosbeef !\\

\textbf{[Refrain]}\\
\end{multicols}

\newpage
\normalsize

\h*{Hier encore}

\begin{multicols}{2}
Hier encore \\
J'avais vingt ans \\
Je caressais le temps \\
Et jouais de la vie \\
Comme on joue de l'amour \\
Et je vivais la nuit \\
Sans compter sur mes jours \\
Qui fuyaient dans le temps \\

J'ai fait tant de projets \\
Qui sont restés en l'air \\
J'ai fondé tant d'espoirs \\
Qui se sont envolés \\
Que je reste perdu \\
Ne sachant où aller \\
Les yeux cherchant le ciel \\
Mais le cœur mis en terre \\

Hier encore \\
J'avais vingt ans \\
Je gaspillais le temps \\
En croyant l'arrêter \\
Et pour le retenir \\
Même le devancer \\
Je n'ai fait que courir \\
Et me suis essoufflé \\

Ignorant le passé \\
Conjuguant au futur \\
Je précédais de moi \\
Toute conversation \\
Et donnais mon avis \\
Que je voulais le bon \\
Pour critiquer le monde \\
Avec désinvolture \\

Hier encore \\
J'avais vingt ans \\
Mais j'ai perdu mon temps \\
A faire des folies \\
Qui ne me laissent au fond \\
Rien de vraiment précis \\
Que quelques rides au front \\
Et la peur de l'ennui \\

Car mes amours sont mortes \\
Avant que d'exister \\
Mes amis sont partis \\
Et ne reviendront pas \\
Par ma faute j'ai fait \\
Le vide autour de moi \\
Et j'ai gâché ma vie \\
Et mes jeunes années \\

Du meilleur et du pire \\
En jetant le meilleur \\
J'ai figé mes sourires \\
Et j'ai glacé mes pleurs \\
Où sont-ils à présent \\
A présent mes vingt ans? \\
\end{multicols}

\newpage
\normalsize

\h*{Foule sentimentale}

\begin{multicols}{2}
Oh la la la vie en rose \\
Le rose qu'on nous propose \\
D'avoir les quantités d'choses \\
Qui donnent envie d'autre chose \\
Aïe, on nous fait croire \\
Que le bonheur c'est d'avoir \\
De l'avoir plein nos armoires \\
Dérisions de nous dérisoires car \\

\begin{bfseries}
[Refrain:]\\
Foule sentimentale \\
On a soif d'idéal \\
Attirée par les étoiles, les voiles \\
Que des choses pas commerciales \\
Foule sentimentale \\
Il faut voir comme on nous parle \\
Comme on nous parle \\
\end{bfseries}

\columnbreak
Il se dégage \\
De ces cartons d'emballage \\
Des gens lavés, hors d'usage \\
Et tristes et sans aucun avantage \\
On nous inflige \\
Des désirs qui nous affligent \\
On nous prend faut pas déconner dès qu'on est né \\
Pour des cons alors qu'on est \\
Des \\

\textbf{[Refrain]} \\

On nous Claudia Schieffer \\
On nous Paul-Loup Sulitzer \\
Oh le mal qu'on peut nous faire \\
Et qui ravagea la moukère \\
Du ciel dévale \\
Un désir qui nous emballe \\
Pour demain nos enfants pâles \\
Un mieux, un rêve, un cheval \\

\textbf{[Refrain]} \\
\end{multicols}

\newpage
\normalsize

\h*{L’été indien}

Tu sais, je n'ai jamais été aussi heureux que ce matin-là \\
Nous marchions sur une plage un peu comme celle-ci \\
C'était l'automne, un automne où il faisait beau \\
Une saison qui n'existe que dans le Nord de l'Amérique \\
Là-bas on l'appelle l'été indien \\
Mais c'était tout simplement le nôtre \\
Avec ta robe longue tu ressemblais \\
A une aquarelle de Marie Laurencin \\
Et je me souviens, je me souviens très bien \\
De ce que je t'ai dit ce matin-là \\
Il y a un an, y a un siècle, y a une éternité \\

\begin{bfseries}
[Refrain:]\\
On ira où tu voudras, quand tu voudras \\
Et on s'aimera encore, lorsque l'amour sera mort \\
Toute la vie sera pareille à ce matin \\
Aux couleurs de l'été indien \\
\end{bfseries}

Aujourd'hui je suis très loin de ce matin d'automne \\
Mais c'est comme si j'y étais. Je pense à toi. \\
Où es-tu? Que fais-tu? Est-ce que j'existe encore pour toi? \\
Je regarde cette vague qui n'atteindra jamais la dune \\
Tu vois, comme elle je reviens en arrière \\
Comme elle je me couche sur le sable \\
Et je me souviens, je me souviens des marées hautes \\
Du soleil et du bonheur qui passaient sur la mer \\
Il y a une éternité, un siècle, il y a un an \\

\textbf{[Refrain]} \\


\newpage
\scriptsize
\h*{Dès que le vent soufflera}

\begin{multicols}{3}

C'est pas l'homme qui prend la mer \\
C'est la mer qui prend l'homme, Tatatin \\
Moi la mer elle m'a pris \\
Je m' souviens un Mardi \\
J'ai troqué mes santiags \\
Et mon cuir un peu zone \\
Contre une paire de docksides \\
Et un vieux ciré jaune \\
J'ai déserté les crasses \\
Qui m' disaient «~Sois prudent~» \\
La mer c'est dégueulasse \\
Les poissons baisent dedans \\

\begin{bfseries}
[Refrain:]\\
Dès que le vent soufflera \\
Je repartira \\
Dès que les vents \\
tourneront \\
Nous nous en allerons \\
\end{bfseries}

C'est pas l'homme qui prend la mer \\
C'est la mer qui prend l'homme \\
Moi la mer elle m'a pris \\
Au dépourvu tans pis \\
J'ai eu si mal au cœur \\
Sur la mer en furie \\
Qu' j'ai vomi mon quatre heures \\
Et mon minuit aussi \\
J' me suis cogné partout \\
J'ai dormi dans des draps mouillés \\
Ça m'a coûté ses sous \\
C'est d' la plaisance, c'est le pied \\

\textbf{[Refrain]} \\

Ho ho ho ho ho hissez haut \\
ho ho ho \\

C'est pas l'homme quiprend la mer \\
C'est la mer qui prend l'homme \\
Mais elle prend pas la femme \\
Qui préfère la campagne \\
La mienne m'attend au port \\
Au bout de la jetée \\
L'horizon est bien mort \\
Dans ses yeux délavés \\
Assise sur une bitte \\
D'amarrage, elle pleure \\
Son homme qui la quitte \\
La mer c'est son malheur \\

\textbf{[Refrain]} \\

C'est pas l'homme qui prend la mer \\
C'est la mer qui prends l'homme \\
Moi la mer elle m'a pris \\
Comme on prend un taxi \\
Je ferai le tour du monde \\
Pour voir à chaque étape \\
Si tous les gars du monde \\
Veulent bien m' lâcher la grappe \\
J'irais aux quatre vents \\
Foutre un peu le boxon \\
Jamais les océans \\
N'oublieront mon prénom \\

\textbf{[Refrain]} \\

Ho ho ho ho ho hissez haut \\
ho ho ho \\

C'est pas l'homme qui prend la mer \\
C'est la mer qui prends l'homme \\
Moi la mer elle m'a pris \\
Et mon bateau aussi \\
Il est fier mon navire \\
Il est est beau mon bateau \\
C'est un fameux trois mats \\
Fin comme un oiseau \emph{[Hissez haut]} \\
Tabarly, Pageot \\
Kersauson ou Riguidel \\
Naviguent pas sur des cageots \\
Ni sur des poubelles \\

\textbf{[Refrain]} \\

C'est pas l'homme qui prend la mer \\
C'est la mer qui prends l'homme \\
Moi la mer elle m'a pris \\
Je m' souviens un Vendredi \\
Ne pleure plus ma mère \\
Ton fils est matelot \\
Ne pleure plus mon père \\
Je vis au fil de l'eau \\
Regardez votre enfant \\
Il est parti marin \\
Je sais c'est pas marrant \\
Mais c'était mon destin \\

\textbf{[Refrain 3x]} \\

Dès que le vent soufflera \\
Nous repartira \\
Dès que les vents \\
tourneront \\
Je me n'en allerons \\
\end{multicols}


\newpage
\small

\h*{Chanson pour Pierrot}

\begin{multicols}{2}
T'es pas né dans la rue \\
T'es pas né dans l' ruisseau \\
T'es pas un enfant perdu \\
Pas un enfant d' salaud, \\
Vu qu' t'es né dans ma tête \\
Et qu' tu vis dans ma peau \\
J'ai construit ta planète \\
Au fond de mon cerveau. \\

\begin{bfseries}
[Refrain:]\\
Pierrot, mon gosse, mon frangin, mon poteau, \\
Mon copain tu m' tiens chaud. \\
Pierrot. \\
\end{bfseries}

Depuis l' temps que j' te rêve, \\
Depuis l' temps que j' t'invente, \\
De pas te voir j'en crève \\
Et j' te sens dans mon ventre. \\
Le jour où tu ramène, \\
J'arrête de boire : promis, \\
Au moins toute une semaine, \\
Ce s'ra dur, mais tant pis. \\

\textbf{[Refrain]} \\

Qu' tu sois fils de princesse, \\
Ou qu' tu sois fils de rien, \\
Tu s'ras fils de tendresse, \\
Tu s'ras pas pas orphelin. \\
Mais j' connais pas ta mère : \\
Je la cherche en vain. \\
Je connais qu' la misère \\
D'être tout seul sur le ch'min. \\

\textbf{[Refrain]} \\

Dans un coin de ma tête \\
Y a déjà ton trousseau : \\
Un jean, une mobylette \\
Une paire de Santiago. \\
T'iras pas à l'école, \\
J' t'apprendrai les gros mots. \\
On jouera au football, \\
On ira au bistrot. \\

\textbf{[Refrain]} \\

Tu t' lav'ras pas les pognes \\
Avant d' venir à table. \\
Et tu m' trait'ras d'ivrogne \\
Quand j' piquerai ton cartable. \\
J' t'apprendrai des chansons \\
Tu les trouveras débiles. \\
T'auras p't' être bien raison \\
Mais j' s'rai vexé quand même. \\

\textbf{[Refrain]} \\

Allez viens mon Pierrot, \\
Tu s'ras l' chef de ma bande. \\
J' te r'filerai mon couteau, \\
J' t'apprendrai la truande. \\
Allez viens mon copain, \\
J' t'ai trouvé une maman : \\
Tous les trois ça s'ra bien \\
Allez viens, je t'attends. \\

\textbf{[Refrain]} \\

\end{multicols}

\newpage
\footnotesize

\h*{La bohème}

\begin{multicols}{2}
Je vous parle d'un temps \\
Que les moins de vingt ans \\
Ne peuvent pas connaître \\
Montmartre en ce temps-là \\
Accrochait ses lilas \\
Jusque sous nos fenêtres \\
Et si l'humble garni \\
Qui nous servait de nid \\
Ne payait pas de mine \\
C'est là qu'on s'est connu \\
Moi qui criait famine \\
Et toi qui posais nue \\

La bohème, la bohème \\
Ça voulait dire on est heureux \\
La bohème, la bohème \\
Nous ne mangions qu'un jour sur \\
deux \\

Dans les cafés voisins \\
Nous étions quelques-uns \\
Qui attendions la gloire \\
Et bien que miséreux \\
Avec le ventre creux \\
Nous ne cessions d'y croire \\
Et quand quelque bistro \\
Contre un bon repas chaud \\
Nous prenait une toile \\
Nous récitions des vers \\
Groupés autour du poêle \\
En oubliant l'hiver \\

La bohème, la bohème \\
Ça voulait dire tu es jolie \\
La bohème, la bohème \\
Et nous avions tous du génie \\

Souvent il m'arrivait \\
Devant mon chevalet \\
De passer des nuits blanches \\
Retouchant le dessin \\
De la ligne d'un sein \\
Du galbe d'une hanche \\
Et ce n'est qu'au matin \\
Qu'on s'asseyait enfin \\
Devant un café-crème \\
Epuisés mais ravis \\
Fallait-il que l'on s'aime \\
Et qu'on aime la vie \\

La bohème, la bohème \\
Ça voulait dire on a vingt ans \\
La bohème, la bohème \\
Et nous vivions de l'air du temps \\

Quand au hasard des jours \\
Je m'en vais faire un tour \\
A mon ancienne adresse \\
Je ne reconnais plus \\
Ni les murs, ni les rues \\
Qui ont vu ma jeunesse \\
En haut d'un escalier \\
Je cherche l'atelier \\
Dont plus rien ne subsiste \\
Dans son nouveau décor \\
Montmartre semble triste \\
Et les lilas sont morts \\

La bohème, la bohème \\
On était jeunes, on était fous \\
La bohème, la bohème \\
Ça ne veut plus rien dire du tout \\
\end{multicols}

\newpage
\normalsize

\h*{L’Aziza}
Petite rue de Casbah au milieu de Casa \\
Petite brune enroulée d'un drap court autour de moi \\
Ses yeux remplis de pourquoi cherchent une réponse en moi \\
Elle veut vraiment que rien ne soit sûr dans tout ce qu'elle croit \\

\begin{bfseries}
[Refrain:]\\
Ta couleur et tes mots tout me va \\
Que tu vives ici ou là-bas \\
Danse avec moi \\
Si tu crois que ta vie est là \\
Ce n'est pas un problème pour moi \\
L'Aziza, \\
Je te veux si tu veux de moi \\
\end{bfseries}

Et quand tu marches le soir ne trembles pas a- a- \\
Laisse glisser les mauvais regards qui pèsent sur toi, \\
L'Aziza \\
Ton étoile jaune c'est ta peau tu n'as pas le choix \\
Ne la porte pas comme on porte un fardeau, ta force c'est ton droit \\

\textbf{[Refrain]} \\

L’Aziza \\
Ta couleur et tes mots tout me va \\
Danse avec moi \\
Que tu vives ici ou là-bas \\
Ce n'est pas un problème pour moi \\
L'Aziza, \\
Si tu crois que ta vie est la \\
Il n'y a pas de loi contre ça \\
L'Aziza, fille enfant de prophète roi \\

\textbf{[Refrain]}

\newpage
\normalsize

\h*{Alexandrie, Alexandra}

\begin{multicols}{2}
Voiles sur les filles \\
Barques sur le Nil \\
Je suis dans ta vie \\
Je suis dans tes bras \\
Alexandra Alexandrie \\

Alexandrie où l'amour danse avec la nuit \\
J'ai plus d'appétit \\
Qu'un Barracuda \\
Je boirai tout le Nil si tu n'me regarde pas \\
Je boirai tout le Nil si tu n’me retiens pas \\
Alexandrie \\
Alexandra \\
Alexandrie où l'amour danse au fond des draps \\
Ce soir j'ai de la fièvre et toi tu meurs de froid \\

\begin{bfseries}
[Refrain:]\\
Les sirènes du port d'Alexandrie \\
Chantent encore la même mélodie wowo \\
La lumière du phare d'Alexandrie \\
Fait naufrager les papillons de ma jeunesse. \\
\end{bfseries}

Voiles sur les filles \\
Barques sur le Nil \\
Je suis dans ta vie \\
Je suis dans tes bras \\
Alexandra Alexandrie \\
Alexandrie où tout commence et tout finit \\
J'ai plus d'appétit \\
Qu'un Barracuda \\
Je te mangerai crue si tu n'me \\
reviens pas \\
Je te mangerai crue si tu n’me \\
retiens pas \\
Alexandrie \\
Alexandra \\

Alexandrie ce soir je danse dans tesdraps \\
Je te mangerai crue si tu n'me retiens pas \\

\textbf{[Refrain]} \\

Ah Aaah \\
Ah Aaah \\

Voiles sur les filles \\
Et barques sur le Nil \\
Alexandrie Alexandra \\
Ce soir j'ai la fièvre et tu meurs de froid \\
Ce soir je dans', je dans', je danse dans tes draps. \\
\end{multicols}

\newpage
\footnotesize

\h*{Le pouvoir des fleurs}

\begin{multicols}{2}
Je m'souviens on avait des projets \\
pour la terre \\
pour les hommes comme la nature \\
faire tomber les barrières, les murs, \\
les vieux parapets d'Arthur \\
fallait voir \\
imagine notre espoir \\
on laissait nos cœurs \\
au pouvoir des fleurs \\
jasmin, lilas, \\
c'étaient nos divisions nos soldats \\
pour changer tout ça \\

\begin{bfseries}
[Refrain:]\\
changer le monde \\
changer les choses avec des \\
bouquets de roses \\
changer les femmes \\
changer les hommes \\
avec des géraniums \\
\end{bfseries}

je m'souviens, on avait des chansons, \\
des paroles \\
comme des pétales et des corolles \\
qu'écoutait en rêvant \\
la petite fille au tourne-disque folle \\
le parfum \\
imagine le parfum \\
l'Eden, le jardin, \\
c'était pour demain, \\
mais demain c'est pareil, \\
le même désir veille \\
là tout au fond des cœurs \\
tout changer en douceur \\

changer les âmes \\
changer les cœurs avec des bouquets \\
de fleurs \\
la guerre au vent \\
l'amour devant \\
grâce à des fleurs des champs \\

ah! sur la terre \\
il y a des choses à faire \\
pour les enfants, les gens, les \\
éléphants \\
ah! tant de choses à faire \\
moi pour \\
te donner du cœur \\
je t'envoie des fleurs \\

tu verras qu'on aura des foulards, \\
des chemises \\
et que voici les couleurs vives \\
et que même si l'amour est parti \\
ce n'est que partie remise \\
pour les couleurs, les accords, les \\
parfums \\
changer le vieux monde \\
pour faire un jardin \\
tu verras \\
tu verras \\
le pouvoir des fleurs \\
y a une idée pop dans mon air \\

\textbf{[Refrain x2]} \\

changer les... \\
Changer les cœurs... \\
\end{multicols}

\newpage
\normalsize

\h*{Petit clown}

Tu fais rire les gens \\
Tu fais rire les enfants \\
Mais le soir en pleurant \\
Tu dois revenir sur terre \\
Tu dois redevenir petit clown de misère \\
Et tu devras reporter ton masque demain \\

Petit clown de misère \\
Ces yeux que je connais \\
Ont perdu de la lumière \\
D’avoir tant et tant pleuré \\
Et ce bleu pris à la mer \\
Peu à peu s’est fané \\
Et ton cœur est fatigué \\
De faire rire et d’amuser \\

Aujourd’hui les enfants, \\
Ne prennent plus la peine \\
Même de faire semblant \\
De rire de tes scènes \\
Et ton cœur bien souvent \\
Aussi froid que la foule \\
Ne peut plus supporter \\
Les larmes sur ta joue \\

Mais un jour en riant \\
Tu quitteras cette terre \\
Mais un jour en riant \\
Tu quitteras nos misères \\
Et je sais que là-haut \\
Tu feras sourire les anges \\
Car tu auras trouvé \\
La paix que tu cherchais (bis)


\newpage
\normalsize

\h*{L’oiseau et l’enfant}

Comme un enfant aux yeux de lumière \\
Qui voit passer au loin les oiseaux \\
Comme l'oiseau bleu survolant la Terre \\
Vois comme le monde, le monde est beau \\

Beau le bateau, dansant sur les vagues \\
Ivre de vie, d'amour et de vent \\
Belle la chanson naissante des vagues \\
Abandonnée au sable blanc \\

Blanc l'innocent, le sang du poète \\
Qui en chantant, invente l'amour \\
Pour que la vie s'habille de fête \\
Et que la nuit se change en jour \\

Jour d'une vie où l'aube se lève \\
Pour réveiller la ville aux yeux lourds \\
Où les matins effeuillent les rêves \\
Pour nous donner un monde d'amour \\

L'amour c'est toi, l'amour c'est moi \\
L'oiseau c'est toi, l'enfant c'est moi. \\

Comme un enfant aux yeux de lumière \\
Qui voit passer au loin les oiseaux \\
Comme l'oiseau bleu survolant la terre \\
Nous trouverons ce monde d'amour \\

L'amour c'est toi, l'amour c'est moi \\
L'oiseau c'est toi, l'enfant c'est moi \\

L'amour c'est toi, l'amour c'est moi \\
L'oiseau c'est toi, l'enfant c'est moi.


\newpage
\footnotesize

\h*{Non, non, rien n’a changé}

\begin{multicols}{2}
C'est l'histoire d'une trêve \\
Que j'avais demandée \\
C'est l'histoire d'un soleil \\
Que j'avais espéré \\
C'est l'histoire d'un amour \\
Que je croyais vivant \\
C'est l'histoire d'un beau jour \\
Que moi, petit enfant \\
Je voulais très heureux \\
Pour toute la planète \\
Je voulais, j'espérais \\
Que la paix règne en maître \\
En ce soir de Noël \\
Mais tout a continué \\
Mais tout a continué \\
Mais tout a continué \\

\begin{bfseries}
[Refrain:]\\
Non, non, rien n'a changé \\
Tout, tout a continué \\
Non, non, rien n'a changé \\
Tout, tout a continué \\
Héhé! Héhé! \\
\end{bfseries}

Et pourtant bien des gens \\
Ont chanté avec nous \\
Et pourtant bien des gens \\
Se sont mis à genoux \\
Pour prier, oui pour prier \\
Pour prier, oui pour prier \\
Mais j'ai vu tous les jours \\
A la télévision \\
Même le soir de Noël \\
Des fusils, des canons \\
J'ai pleuré, oui j'ai pleuré \\
J'ai pleuré \\
Qui pourra m'expliquer que... \\

\textbf{[Refrain]} \\

Moi je pense à l'enfant \\
Entouré des soldats \\
Moi je pense à l'enfant \\
Qui demande pourquoi \\
Tout le temps, oui tout le temps \\
Tout le temps, oui tout le temps \\
Moi je pense à tout ça \\
Mais je ne devrais pas \\
Toutes ces choses-là \\
Ne me regardent pas \\
Et pourtant, oui et pourtant \\
Et pourtant, je chante, je chante... \\

\textbf{[Refrain]} \\

C'est l'histoire d'une trêve \\
Que j'avais demandée \\
C'est l'histoire d'un soleil \\
Que j'avais espéré \\
C'est l'histoire d'un amour \\
Que je croyais vivant \\
C'est l'histoire d'un beau jour \\
Que moi, petit enfant \\
Je voulais très heureux \\
Pour toute la planète \\
Je voulais, j'espérais \\
Que la paix règne en maître \\
En ce soir de Noël \\
Mais tout a continué \\
Mais tout a continué \\
Mais tout a continué \\

\textbf{[Refrain]} \\

Héhé! Héhé!
\end{multicols}

\newpage
\normalsize

\h*{Liberté}

Ne nous parlez plus de héros \\
Ne nous parlez plus de révolutions \\
Dites-nous combien il reste encore \\
Vous laissez derrière vous des rêves pillés \\
Des mondes gaspillés, des soleils brûlés, \\
Laissez-nous créer une arme d’amour \\
Une bombe à lumière, un fusil à fleurs \\
Une vie sans barrière. \\
Laissez-nous rêver d’un enfant président \\
D’un roi sans couronne, d’un Jésus indien \\
D’un Dieu qui pardonne même ceux qui l’oublient. \\
Ne nous parlez plus de héros \\
Ne nous parlez plus de révolutions \\
Dites-nous combien il reste encore \\
Vous laissez derrière vous des mères matraquées, \\
Des lunes piétinées, des hommes qui mourraient \\
Pour la Liberté.

\newpage
\normalsize

\h*{Rêve ta vie}

\begin{bfseries}
[Refrain:]\\
Rêve, rêve, rêve ta vie (bis) \\
Dans ton sommeil \\
Et au réveil \\
Vis ton rêve ! \\
\end{bfseries}

Rêve un matin d'été, un grand champ de blé \\
Un ami trouvé en chemin \\
Rêve un ruisseau d'eau claire, près d'une clairière \\
Un boulanger qui donne son pain \\
Ferme les yeux et dis-toi que c'est arrivé \\
Il faut y croire très fort et ne pas en douter \\

\textbf{[Refrain]} \\

Rêve un grand feu de bois, quand dehors il fait froid \\
Un ennemi qui tend la main \\
Rêve que tu cours dans les champs, c'est déjà le printemps \\
La récolte est bonne au Pakistan \\
Ferme les yeux et dis-toi que c'est arrivé \\
Il faut y croire très fort et ne pas en douter

\newpage
\normalsize

\h*{A nos actes manqués}

A tous mes loupés, mes ratés, mes vrais soleils \\
Tous les chemins qui me sont passés à côté \\
A tous mes bateaux manqués, mes mauvais sommeils \\
A tous ceux que je n'ai pas été \\

Aux malentendus, aux mensonges, à nos silences \\
A tous ces moments que j'avais cru partager \\
Aux phrases qu'on dit trop vite et sans qu'on les pense \\
A celles que je n'ai pas osées \\
A nos actes manqués \\

Aux années perdues à tenter de ressembler \\
A tous les murs que je n'aurais pas su briser \\
A tout c'que j'ai pas vu tout près, juste à côté \\
Tout c'que j'aurais mieux fait d'ignorer \\

Au monde, à ses douleurs qui ne me touchent plus \\
Aux notes, aux solos que je n'ai pas inventés \\
Tous ces mots que d'autres ont fait rimer et qui me tuent \\
Comme autant d'enfants jamais portés \\
A nos actes manqués \\

Aux amours échouées de s'être trop aimé \\
Visages et dentelles croisés justes frôlés \\
Aux trahisons que j'ai pas vraiment regrettées \\
Aux vivants qu'il aurait fallu tuer \\

A tout ce qui nous arrive enfin, mais trop tard \\
A tous les masques qu'il aura fallu porter \\
A nos faiblesses, à nos oublis, nos désespoirs \\
Aux peurs impossibles à échanger \\

A nos actes manqués


\newpage
\footnotesize

\h*{Je te donne}

I can give you a voice, bred with rythms and soul \\
the heart of a Welsh boy who's lost his home \\
put it in harmony , let the words ring \\
carry your thoughts in the song we sing \\
Je te donne mes notes , je te donne mes mots \\
quand ta voix les emporte a ton propre tempo \\
une épaule fragile et solide a la fois \\
ce que j'imagine et ce que je crois . \\

Je te donne toutes mes differences, \\
tous ces défauts qui sont autant de chances \\
on sera jamais des standards, des gens bien comme il faut \\
je te donne ce que j'ai ce que je vaux \\

I can give you the force of my ancestral pride \\
the will to go on when i'm hurt deep inside \\
whatever the feeling, whatever the way \\
it helps me to go on from day to day \\
je te donne nos doutes et notre indicible espoir \\
les questions que les routes ont laissées dans l'histoire \\
nos filles sont brunes et l'on parle un peu fort \\
et l'humour et l'amour sont nos trésors \\

Je te donne , donne , donne ce que je suis \\

I can give you my voice, bred with rythm and soul, \\
je te donne mes notes , je te donne ma voix \\
the songs that i love, and the stories i've told \\
ce que j'imagine et ce que je crois \\
i can make you feel good even when i'm down \\
les raisons qui me portent et ce stupide espoir \\
my force is a platform that you can climb on \\
une épaule fragile et forte a la fois \\

\begin{bfseries}
[5x:] \\
je te donne, je te donne tout ce que je vaux , ce que je suis, mes dons, \\
mes défauts, mes plus belles chances, mes différences
\end{bfseries}

\newpage
\normalsize

\h*{C’est écrit}

\begin{multicols}{2}
Elle te fera changer la course des nuages, \\
Balayer tes projets, vieillir bien avant l'âge, \\
Tu la perdras cent fois dans les vapeurs des ports, \\
C'est écrit... \\
Elle rentrera blessée dans les parfums d'un autre, \\
Tu t'entendras hurler «~que les diables l'emportent~» \\
Elle voudra que tu pardonnes, et tu pardonneras, \\
C'est écrit... \\
Elle n'en sort plus de ta mémoire \\
Ni la nuit, ni le jour, \\
Elle danse derrière les brouillards \\
Et toi, tu cherches et tu cours. \\
Tu prieras jusqu'aux heures ou personne n'écoute, \\
Tu videras tous les bars qu'elle mettra sur ta route, \\
T'en passeras des nuits à regarderdehors. \\
C'est écrit... \\
Elle n'en sort plus de ta mémoire \\
Ni la nuit, ni le jour, \\
Elle danse derrière les brouillards \\
Et toi, tu cherches et tu cours, \\
Mais y a pas d'amours sans histoires. \\
Et tu rêves, tu rêves... \\
Qu'est-ce qu'elle aime, qu'est-ce qu'elle veut ? \\
Et ses ombres qu'elle te dessine autour des yeux ? \\
Qu'est-ce qu'elle aime ? \\
Qu'est-ce qu'elle rêve, qui elle voit ? \\
Et ces cordes qu'elle t'enroule autour des bras ? \\
Qu'est-ce qu'elle aime ? \\
Je t'écouterai me dire ses soupirs, ses dentelles, \\
Qu'à bien y réfléchir, elle n'est plus vraiment belle, \\
Que t'es déjà passé par des moments plus forts, \\
Depuis... \\
Elle n'en sort plus de ta mémoire \\
Ni la nuit, ni le jour, \\
Elle danse derrière les brouillards \\
Et toi, tu cherches et tu cours, \\
Mais y a pas d'amours sans histoires. \\
Oh tu rêves, tu rêves... \\
Elle n'en sort plus de ta mémoire \\
Elle danse derrière les brouillards \\
Et moi j'ai vécu la même histoire \\
Depuis je compte les jours \\
Depuis je compte les nuits
\end{multicols}

\newpage
\footnotesize

\h*{Place des grands hommes}

\begin{multicols}{2}
\begin{bfseries}
[Refrain:] \\
On s'était dit rendez-vous dans 10 ans \\
Même jour, même heure, mêmes port \\
On verra quand on aura 30 ans \\
Sur les marches de la place des grands hommes \\
\end{bfseries}

Le jour est venu et moi aussi \\
Mais j' veux pas être le premier. \\
Si on avait plus rien à se dire et si et si... \\

Je fais des détours dans le quartier. \\
C'est fou c'qu'un crépuscule de printemps. \\
Rappelle le même crépuscule qu'il y a 10 ans, \\
Trottoirs usés par les regards baissés. \\
Qu'est-ce que j'ai fait de ces années ? \\

J'ai pas flotté tranquille sur l'eau, \\
Je n'ai pas nagé le vent dans le dos. \\
Dernière ligne droite, la rue Soufflot, \\
Combien seront là 4, 3, 2, 1... 0 ? \\

\textbf{[Refrain]} \\

J'avais eu si souvent envie d'elle. \\
La belle Séverine me regardera-t-elle ? \\
Eric voulait explorer le subconscient. \\
Remonte-t-il à la surface de temps en temps ? \\
J'ai un peu peur de traverser l' miroir. \\
Si j'y allais pas... J' me serais trompé d'un soir. \\
Devant une vitrine d'antiquités, \\
J'imagine les retrouvailles de l'amitié. \\
«~T'as pas changé, qu'est-ce que tudeviens ? \\
Tu t'es mariée, t'as trois gamins. \\
T'as réussi, tu fais médecin ? \\
Et toi Pascale, tu t' marres toujourspour rien ?~» \\

\textbf{[Refrain]} \\

J'ai connu des marées hautes et des marées basses, \\
Comme vous, comme vous, comme vous. \\
J'ai rencontré des tempêtes et des bourrasques, \\
Comme vous, comme vous, comme vous. \\
Chaque amour morte à une nouvelle a fait place, \\
Et vous, et vous...et vous ? \\
Et toi Marco qui ambitionnait simplement d'être heureux dans la vie, \\
As-tu réussi ton pari ? \\
Et toi François, et toi Laurence, et toi Marion, \\
Et toi Gégé...et toi Bruno, et toi Evelyne ? \\

\textbf{[Refrain]} \\

Et bien c'est formidable les copains! \\
On s'est tout dit, on s' serre la main ! \\
On ne peut pas mettre 10 ans sur table \\
Comme on étale ses lettres au Scrabble. \\
Dans la vitrine je vois le reflet \\

D'une lycéenne derrière moi. \\
Si elle part à gauche, je la suivrai. \\
Si c'est à droite... Attendez-moi ! \\
Attendez-moi ! Attendez-moi ! \\
Attendez-moi ! \\

On s'était dit rendez-vous dans 10 ans, \\
Même jour, même heure, mêmes pommes. \\
On verra quand on aura 30 ans \\
Si on est d'venus des grands hommes... \\
Des grands hommes... des grands hommes... \\
Tiens si on s' donnait rendez-vous dans 10 ans... \\

\end{multicols}

\newpage
\footnotesize

\h*{Mistral gagnant}
A m'asseoir sur un banc cinq minutes avec toi \\
Et regarder les gens tant qu'y en a \\
Te parler du bon temps qui est mort ou qui r'viendra \\
En serrant dans ma main tes p'tits doigts \\
Pis donner à bouffer à des pigeons idiots \\
Leur filer des coups d' pieds pour de faux \\
Et entendre ton rire qui lézarde les murs \\
Qui sait surtout guérir mes blessures \\

Te raconter un peu comment j'étais mino \\
Les bonbecs fabuleux, qu'on piquait chez l' marchand \\
Car-en-sac et Minto, caramel à un franc \\
Et les mistrals gagnants \\

Par marcher sous la pluie cinq minutes avec toi \\
Et regarder la vie tant qu'y en a \\
Te raconter la Terre en te bouffant des yeux \\
Te parler de ta mère un p'tit peu \\
Et sauter dans les flaques pour la faire râler \\
Bousiller nos godasses et s' marrer \\
Et entendre ton rire comme on entend la mer \\
S'arrêter, r'partir en arrière \\
Te raconter surtout les carambars d'antan et les cocos bohères \\
Et les vrais roudoudous qui nous coupaient les lèvres \\
Et nous niquaient les dents \\
Et les mistrals gagnants \\

A m'asseoir sur un banc cinq minutes avec toi \\
Et r'garder le soleil qui s'en va \\
Te parler du bon temps qui est mort et je m'en fous \\
Te dire que les méchants c'est pas nous \\
Que si moi je suis barge, ce n'est que de tes yeux \\
Car ils ont l'avantage d'être deux \\
Et entendre ton rire s'envoler aussi haut \\
Que s'envolent les cris des oiseaux \\

Te raconter enfin qu'il faut aimer la vie \\
Et l'aimer même si \\
le temps est assassin \\
Et emporte avec lui les rires des enfants \\
Et les mistrals gagnants \\
Et les mistrals gagnants

\newpage
\normalsize

\vbox{
\begin{minipage}[t][0.4\textheight][t]{\textwidth}
\vspace{0.02\textheight}
\h*{La bille de verre}
\scriptsize
\begin{multicols}{2}
Un bateau de bois \\
Emporte papa \\
Tout au bout d'la terre. \\
Il verra la Chine \\
Et les îles opalines \\
Où les gens vivent nus. \\
Moi, j'deviendrai un homme, \\
Mes notes seront bonnes, \\
Il sera fier de moi. \\
Il me rapportera une bille de verre \\
Et un ver à soie (bis). \\

Si la nuit m'fait peur \\
J'lui dirai que mon cœur \\
Est au bout d'la terre \\
Où les enfants des rois \\
Ont des sabres qui coupent \\
Et des chevaux vivants. \\
Moi, je ferai l'grand, \\
Je défendrai maman \\
Contre les voleurs. \\
Il me rapportera une bille de verre \\
Et un ver à soie (bis). \\

\columnbreak

Plus tard il y aura les caresses des femmes, \\
Les secrets qui planent \\
Aux oreilles des grands, \\
Les départs à minuit, les tempêtes, les \\
drames, \\
L'océan... \\
Et quelqu'un qui attend une bille de verre \\
Et un ver à soie (bis). \\

Un bateau de bois \\
Emporte papa \\
Tout au bout d'la terre. \\
Il chass'ra le fauve \\
Au fond des jungles mauves \\
Où le jour n'entre pas. \\
Je cach'rai ma peine. \\
J'attendrai qu'il revienne. \\
Il sera fier de moi. \\
Il me rapportera une bille de verre \\
Et un ver à soie (bis). \\

Il me rapportera une bille de verre... (x2) \\
Il me rapportera une bille de verre \\
Et un ver à soie. \\

\end{multicols}
\end{minipage}
\vspace{0.02\textheight}

\begin{minipage}[b][0.55\textheight][t]{\textwidth}
\h*{Il changeait la vie}
\begin{multicols}{2}
\scriptsize
C'était un cordonnier, sans rien d'particulier \\
Dans un village dont le nom m'a échappé \\
Il faisait des souliers si jolis, si légers \\
Que nos vies semblaient un peu moins lourdes à porter \\

Il y mettait du temps, du talent et du cœur \\
Ainsi passait sa vie au milieu de nos heures \\
Et loin des beaux discours, des grandes théories \\
A sa tâche chaque jour, on pouvait dire de lui \\
Il changeait la vie \\

C'était un professeur, un simple professeur \\
Qui pensait que savoir était un grand trésor \\
Que tous les moins que rien n'avaient pour s'en sortir \\
Que l'école et le droit qu'a chacun de s'instruire \\

Il y mettait du temps, du talent et du cœur \\
Ainsi passait sa vie au milieu de nos heures \\
Et loin des beaux discours, des grandes théories \\
A sa tâche chaque jour, on pouvait dire de lui \\
Il changeait la vie \\

C'était un p'tit bonhomme, rien qu'un tout p'tit bonhomme \\
Malhabile et rêveur, un peu loupé en somme \\
Se croyait inutile, banni des autres hommes \\

Il pleurait sur son saxophone \\

Il y mit tant de temps, de larmes et de douleur \\
Les rêves de sa vie, les prisons de son cœur \\
Et loin des beaux discours, des grandes théories \\
Inspiré jour après jour de son souffle et de ses cris \\
Il changeait la vie (x5) \\
\end{multicols}
\end{minipage}
}


\newpage
\normalsize

\vbox{
\begin{minipage}[t][0.4\textheight][t]{\textwidth}
\vspace{0.02\textheight}
\h*{Cœur de loup}
\scriptsize
\begin{multicols}{2}
Cœur de Loup
Pas le temps de tout lui dire \\
Pas le temps de tout lui taire \\
Juste assez pour tenter la satyre \\
Qu'elle sente que j'veux lui plaire \\
Sous le pli de l'emballage \\
La lubie de faufiler \\
La folie de rester sage si elle veut \\
De n'pas l'embrasser \\
Quand d'un coup d'aile se déplume \\
Mon œillet luit fait de l'œil \\
Même hululer sous la lune ne m'fait pas peur \\
Pourvu qu'elle veuille \\

Je n'ai qu'une seule envie \\
Me laisser tenter \\
La victime est si belle \\
Et le crime est si gai \\
Cœur de loup, peur du lit, séduis-là, sans délais \\
Suis le swing, c’est le coup de gong du kingbong \\

Pas besoin de beaucoup \\
Mais pas de peu non plus \\
Par le biais d'un billet fou \\
Lui faire savoir que j'n'en peux plus \\
C'est le cas du kamikaze \\
C'est l'ABC du condamné \\
Le légionnaire qui veut l'avantage des voyages \\
Sans s'engager \\
Elle est si frêle esquive \\
Sous mes bordées d'amour \\
Je suppose qu'elle suppose \\
Que je l'aimerai toujours \\
Le doigts sur l'aventure \\
Le pied dans l'inventaire \\
Même si l'affaire n'est pas sûre \\
Ne pas s'enfuir \\
Ne pas s'en faire \\

Je n'ai qu'une seule envie \\
Me laisser tenter \\
La victime est si belle \\
Et le crime est si gai \\

Pas le temps de mentir \\
Ni de quitter la scène \\
YEP ! Elle aura beau rougir \\
De toute façon il faut qu'elle m'aime \\
Je n'ai qu'un seule envie \\
Me laisser tenter… 
\end{multicols}
\end{minipage}
\vspace{0.02\textheight}

\begin{minipage}[b][0.55\textheight][t]{\textwidth}
\h*{L’encre de tes yeux}
\begin{multicols}{2}
\scriptsize
Puisqu'on ne vivra jamais tous les deux \\
Puisqu'on est fou, puisqu'on est seuls \\
Puisqu'ils sont si nombreux \\
Même la morale parle pour eux \\

J'aimerais quand même te dire \\
Tout ce que j'ai pu écrire \\
Je l'ai puisé à l'encre de tes yeux. \\

Je n'avais pas vu que tu portais des chaînes \\
À trop vouloir te regarder, \\
J'en oubliais les miennes \\
On rêvait de Venise et de liberté \\
J'aimerais quand même te dire \\
Tout ce que j'ai pu écrire \\
C'est ton sourire qui me l'a dicté. \\

Tu viendras longtemps marcher dans mes rêves \\
Tu viendras toujours du côté \\
Où le soleil se lève \\
Et si malgré ça j'arrive à t'oublier \\
J'aimerais quand même te dire \\
Tout ce que j'ai pu écrire \\
Aura longtemps le parfum des regrets. \\
\end{multicols}
\end{minipage}
}

\newpage
\normalsize

\h*{Étoile des neiges}
\begin{multicols}{2}
Dans un coin perdu de montagne \\
Un tout petit savoyard \\
Chantait son amour \\
Dans le calme du soir \\
Près de sa bergère \\
Au doux regard. \\

\begin{bfseries}
[Refrain 1:] \\
Étoile des neiges \\
Mon coeur amoureux \\
S'est pris au piège \\
De tes grands yeux \\
Je te donne en gage \\
Cette croix d'argent \\
Et de t'aimer toute ma vie \\
Je fais serment. \\
\end{bfseries}

Hélas soupirait la bergère \\
Que répondront nos parents \\
Comment ferons-nous \\
Nous n'avons pas d'argent \\
Pour nous marier \\
Dès le printemps ? \\

\begin{bfseries}
[Refrain 2:] \\
Étoile des neiges \\
Sèche tes beaux yeux \\
Le ciel protège \\
Les amoureux \\
Je pars en voyage \\
Pour qu'à mon retour \\
À tout jamais plus rien \\
N'empêche notre amour. \\
\end{bfseries}

Alors il partit vers la ville \\
Et ramoneur il se fit \\
Sur tous les chemins \\
Dans le vent et la pluie \\
Comme un petit diable \\
Noir de suie. \\


\begin{bfseries}
[Refrain 3:] \\
Étoile des neiges \\
Sèche tes beaux yeux \\
Le ciel protège \\
Ton amoureux \\
Ne perds pas courage \\
Il te reviendra \\
Et tu seras bientôt \\
Encore entre ses bras. \\
\end{bfseries}

Et quand les beaux jours refleurirent \\
Il s'en revint au hameau \\
Et sa fiancée \\
L'attendait tout là-haut \\
Parmi les clochettes \\
Des troupeaux. \\

\begin{bfseries}
[Refrain 4:] \\
Étoile des neiges \\
Tes garçons d'honneur \\
Vont en cortège \\
Portant des fleurs \\
Par un mariage \\
Finit mon histoire \\
De la bergère et de son petit savoyard \\
\end{bfseries}
\end{multicols}

\newpage
\normalsize

\vbox{
\begin{minipage}[t][0.4\textheight][t]{\textwidth}
\vspace{0.02\textheight}
\h*{Mon mec à moi}
\scriptsize
\begin{multicols}{2}
Il joue avec mon cœur, \\
il triche avec ma vie, \\
il dit des mots menteurs, \\
et moi, je crois tout c' qu'il dit \\
Les chansons qu'il me chante, \\
les rêves qu'il fait pour deux, \\
c'est comme les bonbons menthe, \\
ça fait du bien quand il pleut. \\
Je m' raconte des histoires, \\
en écoutant sa voix, \\
c'est pas vrai ces histoires, \\
mais moi j'y crois. \\

Mon mec à moi \\
il me parle d'aventures, \\
et quand elles brillent dans ses yeux, \\
j' pourrais y passer la nuit \\
Il parle d'amour \\
comme il parle des voitures \\
et moi j'l'suis où il veut, \\
tellement je crois tout c'qu'il m'dit \\
tellement je crois tout c'qu'il m'dit \\
Oh oui \\
Mon mec à moi. \\

Sa façon d'être à moi \\
sans jamais dire “ je t'aime “, \\
c'est rien qu'du cinéma, \\
mais c'est du pareil au même. \\
Ce film en noir et blanc \\
qu'il m'a joué deux cents fois, \\
c'est Gabin et Morgan \\
enfin, ça ressemble à tout ça \\
Je m'raconte des histoires, \\
des scénarios chinois, \\
c'est pas vrai ces histoires, \\
mais moi j'y crois \\

Mon mec à moi \\
il me parle d'aventures \\
et quand elles brillent dans ses yeux, \\
j'pourrais y passer la nuit \\
Il parle d'amour \\
comme il parle des voitures, \\
et moi j'l'suis où il veut, \\
tellement je crois tout c'qu'il m'dit \\
tellement je crois tout c'qu'il m'dit \\
Oh oui \\
\end{multicols}
\end{minipage}
\vspace{0.07\textheight}

\begin{minipage}[b][0.55\textheight][t]{\textwidth}
\h*{Elle est d’ailleurs}
\begin{multicols}{2}
\footnotesize
Elle a de ces lumières au fond des yeux \\
Qui rendent aveugles ou amoureux \\
Elle a des gestes de parfum \\
Qui rendent bête ou rendent chien \\
Et si lointaine dans son cœur \\
Pour moi c'est sûr, elle est d'ailleurs \\

Elle a de ces longues mains de dentellière \\
A damner l'âme d'un Werner \\
Cette silhouette vénitienne \\
Quand elle se penche à ses persiennes \\
Ce geste je le sais par cœur \\
Pour moi c'est sûr, elle est d'ailleurs \\

Et moi je suis tombé en esclavage \\
De ce sourire, de ce visage \\
Et je lui dis emmène moi \\
Et moi je suis prêt à tous les sillages \\
Vers d'autres lieux, d'autres rivages \\
Mais elle passe et ne répond pas \\

Et moi je suis tombé en esclavage \\
De ce sourire, de ce visage \\
Et je lui dis emmène moi \\
Et moi je suis prêt à tous les sillages \\
Vers d'autres lieux, d'autres rivages \\
Mais elle passe et ne répond pas \\
Les mots pour elle sont sans valeur \\
Pour moi c’est sur, elle est d’ailleurs \\

Elle a de ces manières de ne rien dire \\
Qui parlent au bout des souvenirs \\
Cette manière de traverser \\
Quand elle s'en va chez le boucher \\
Quand elle arrive à ma hauteur \\
Pour moi c'est sûr, elle est d'ailleurs \\

\end{multicols}
\end{minipage}
}
\newpage
\normalsize

\vbox{
\begin{minipage}[t][0.4\textheight][t]{\textwidth}
\vspace{0.02\textheight}
\h*{Belle Ile-en-Mer}
\small
\begin{multicols}{2}
\begin{bfseries}
[Refrain:] \\
Belle-Ile-en-Mer, Marie-Galante, \\
Saint-Vincent, Loin Singapour \\
Seymour Ceylan, Vous c'est l'eau, \\
c'est l'eau qui vous sépare \\
Et vous laisse à part \\
\end{bfseries}

Moi des souvenirs d'enfance en France, violence \\
Manque d'indulgence par les différences que j'ai \\
Café Léger au lait mélangé \\
Séparé petit enfant tout comme vous, je connais ce sentiment \\
De solitude et d'isolement \\

\textbf{[Refrain]} \\

Comme laissé tout seul en mer, \\
Corsaire sur terre \\
Un peu solitaire l'amour je l'voyais passer, \\
Ohé Ohé, je l'voyais passer \\
Séparé petit enfant tout comme vous je connais ce sentiment \\
De solitude et d'isolement \\

\textbf{[Refrain]} \\

Karukera, Calédonie, Ouessant,
Vierges des mers, toutes seules tout l'temps
Vous c'est l'eau, c'est l'eau qui vous sépare
Et vous laisse à part
Oh oh... \\

\end{multicols}
\end{minipage}
\vspace{0.09\textheight}

\begin{minipage}[b][0.55\textheight][t]{\textwidth}
\h*{Il y a}
\footnotesize
Il y a du thym, de la bruyère et des bois de pin, rien de bien malin \\
Il y a des ruisseaux, des clairières, pas de quoi en faire un plat de ce coin \\
Il y a des odeurs de menthe et des cheminées et des feux dedans \\
Il y a des jours et des nuits lentes et l'histoire absente banalement \\

Et loin de tout, loin de moi \\
C'est là que tu te sens chez toi \\
De là que tu pars, où tu reviens chaque fois \\
Et où tout finira \\

Il y a des enfants, des grand-mères \\
Une petite église et un grand café \\
Il y a au fond du cimetière des joies, des misères et du temps passé \\
Il y a une petite école et des bancs de bois, tout comme autrefois \\
Il y a des images qui collent au bout de tes doigts \\
Et ton cœur qui bat \\

Et loin de tout, loin de moi \\
C'est là que tu te sens chez toi \\
De là que tu pars, où tu reviens chaque fois \\
Et où tout finira \\
\end{minipage}
}

\newpage
\normalsize

\h*{Marchand de cailloux}
\begin{multicols}{2}
Dis Papa, quand c'est qu'y passe \\
Le marchand d'cailloux \\
J'en voudrais dans mes godasses \\
A la place des joujoux \\

Avec mes copines en classe \\
On comprend pas tout \\
Pourquoi des gros dégueulasses \\
Font du mal partout \\
Pourquoi les enfants de Belfast \\
Et d'tous les ghettos \\
Quand y balancent un caillasse \\
On leur fait la peau \\
J'croyais qu'David et Goliath \\
Ça marchait encore \\
Les plus p'tits pouvaient s'débattrent \\
Sans être les plus morts \\

Dis Papa, quand c'est qu'y passe \\
Le marchand d'liberté \\
Il en a oublié un max \\
En f'sant sa tournée \\
Pourquoi des mômes crèvent de faim \\
Pendant qu'on étouffe \\
D'vant nos télés, comme des crétins \\
Sous des tonnes de bouffe \\

Dis Papa, quand c'est qu'y passe \\
Le marchand d'tendresse \\
S'il est sur l'trottoir d'en face \\
Dis-y qu'y traverse \\
J'peux lui en r'filer un peu \\
Pour ceux qu'en ont b'soin \\
J'en ai r'çu tellement mon vieux \\
Qu'j'peux en donner tout plein \\
J'veux partager mon Mac Do \\
Avec ceux qui ont faim \\
J'veux donner d'amour bien chaud \\
A ceux qu'on plus rien \\
Est-ce que c'est ça être coco \\
Ou être un vrai chrétien \\
Moi j'me fous de tous ces mots \\
J'veux être un vrai humain \\

Dis Papa, tous ces discours \\
Me font mal aux oreilles \\
Même ceux qui sont plein d'amour \\
C'est kif-kif-pareil \\
Ça m'fais comme des trous dans la tête \\
Ça m'pollue la vie et tout \\
Ça fait qu'je vois sur ma planète \\
Des «~Inti Fada~» partout \\

[Répétition] \\

(x2) \\
Et p't'être que sur ta guitare \\
J'en jetterai aussi \\
Si tu t'sers de moi, trouillard \\
Pour chanter tes conneries \\
\end{multicols}

\newpage
\normalsize
\vbox{
\begin{minipage}[t][0.4\textheight][t]{\textwidth}
\vspace{0.02\textheight}
\h*{Morgane de toi}
\scriptsize
\begin{multicols}{2}
Y a un mariolle, il a au moins quatre ans \\
Y veut t' piquer ta pelle et ton seau \\
Ta couche culotte avec tes bonbecs dedans \\
Lolita, défend-toi, fous-y un coup d' râteau dans l' dos \\
Attend un peu avant de t'faire emmerder \\
Par ces p'tits machos qui pensent qu'à une chose \\
Jouer au docteur non conventionné \\
J'y ai joué aussi, je sais de quoi j' cause \\
J' les connais bien les play-boys des bacs à sable \\
J' draguais leurs mères avant d' connaître la tienne \\
Si tu les écoutes y t' feront porter leurs cartables \\
'Reusement qu' j' suis là, que j' te regarde et que j' t'aime \\

\begin{bfseries}
[Refrain:] \\
Lola \\
J' suis qu'un fantôme quand \\
tu vas où j' suis pas \\
Tu sais ma môme \\
Que j' suis morgane de toi \\
\end{bfseries}

Comme j'en ai marre de m' faire tatouer des machins \\
Qui m' font comme une bande dessinée sur la peau \\
J'ai écrit ton nom avec des clous dorés \\
Un par un, plantés dans le cuir de mon blouson dans l' dos \\
T'es la seule gonzesse que j'peux tenir dans mes bras \\
Sans m' démettre une épaule, sans plier sous ton poids \\
Tu pèses moins lourd qu'un moineau qui mange pas \\
Déploie jamais tes ailes, Lolita t'envole pas \\
Avec tes miches de rat qu'on dirait des noisettes \\
Et ta peau plus sucrée qu'un pain au chocolat \\
Tu risques de donner faim a un tas de p'tits mecs \\
Quand t'iras à l'école, si jamais t'y vas \\

\textbf{[Refrain]} \\

Qu'est-ce qu' tu m' racontes tu veux un p'tit frangin \\
Tu veux qu' j' t'achète un ami Pierrot \\
Eh les bébés ça s' trouve pas dans les magasins \\
Puis j' crois pas que ta mère voudra qu' j' lui fasse un p'tit dans l' dos \\
Ben quoi Lola on est pas bien ensemble \\
Tu crois pas qu'on est déjà bien assez nombreux \\
T'entends pas c' bruit, c'est le monde qui tremble \\
Sous les cris des enfants qui sont malheureux \\
Allez viens avec moi, j't'embarque dans ma galère \\
Dans mon arche y a d' la place pour tous les marmots \\
Avant qu' ce monde devienne un grand cimetière \\
Faut profiter un peu du vent qu'on a dans l' dos \\

\textbf{[Refrain 2x]}

\end{multicols}
\end{minipage}

\vspace{0.09\textheight}
\begin{minipage}[t][0.55\textheight][t]{\textwidth}
\h*{Elisa}
\begin{multicols}{2}
\scriptsize
Elisa, Elisa \\
Elisa saute-moi au cou \\
Elisa, Elisa \\
Elisa cherche-moi des poux, \\
Enfonce bien tes ongles, \\
Et tes doigts délicats \\
Dans la jungle \\
De mes cheveux Lisa \\

Elisa, Elisa \\
Elisa saute-moi au cou \\
Elisa, Elisa \\
Elisa cherche-moi des poux, \\
Fais-moi quelques anglaises, \\
Et la raie au milieu \\
On a treize \\
Quatorze ans à nous deux \\

Elisa, Elisa \\
Elisa les autr's on s'en fout, \\
Elisa, Elisa \\
Elisa rien que toi, moi, nous \\
Tes vingt ans, mes quarante \\
Si tu crois que cela \\
Me tourmente \\
Ah non vraiment Lisa \\

Elisa, Elisa \\
Elisa saute-moi au cou \\
Elisa, Elisa \\
Elisa cherche-moi des poux, \\
Enfonce bien tes ongles, \\
Et tes doigts délicats \\
Dans la jungle \\
De mes cheveux Lisa \\
\end{multicols}
\end{minipage}
}

\newpage
\normalsize

\h*{Elle écoute pousser les fleurs}

\begin{multicols}{2}
Elle écoute pousser les fleurs \\
Au milieu du bruit des moteurs \\
Avec de l'eau de pluie \\
Et du parfum d'encens \\
Elle voyage de temps en temps \\
Elle n'a jamais rien entendu \\
Des chiens qui aboient dans la rue \\
Elle fait du pain doré \\
Tous les jours à quatre heures \\
Elle mène sa vie en couleur \\

Elle collectionne \\
Les odeurs de l'automne \\
Et les brindilles de bois mort \\
Quand l'hiver arrive \\
Elle ferme ses livres \\

Et puis doucement \\
Elle s'endort sur des tapis de laine \\
Au milieu des poupées indiennes \\
Sur les ailes en duvet \\
De ses deux pigeons blancs \\
Jusqu'aux premiers jours du printemps \\

Elle dit qu'elle va faire \\
Le tour de la terre \\
Qu'elle sera rentrée pour dîner \\
Les instants fragiles \\
Les mots inutiles \\

Elle sait tout cela \\
Quand elle écoute pousser les fleurs \\
Au milieu du bruit des moteurs \\
Quand les autres s'emportent \\
Quand j'arrive à m'enfuir \\
C'est chez elle que je vais dormir \\

Et c'est vrai que j'ai peur de lui faire un enfant... \\
\end{multicols}

\newpage
\normalsize

\h*{Quelque chose de Tennessee}
On a tous en nous quelque chose de Tennessee \\
Cette volonté de prolonger la nuit \\
Ce désir fou de vivre une autre vie \\
Ce rêve en nous avec ses mots à lui \\
Quelque chose en nous de Tennessee \\

Quelque chose de Tennessee \\
Cette force qui nous pousse vers l'infini \\
Y a peu d'amour avec tell'ment d'envie \\
Si peu d'amour avec tell'ment de bruit \\
Quelque chose de Tennessee \\

Ainsi vivait Tennessee \\
Le cœur en fièvre et le corps démoli \\
Avec cette formidable envie de vie \\
Ce rêve en nous c'était son cri à lui \\
Quelque chose de Tennessee \\

Comme une étoile qui s'éteint dans la nuit \\
A l'heure où d'autres s'aiment à la folie \\
Sans un éclat de voix et sans un bruit \\
Sans un seul amour, sans un seul ami \\
Ainsi disparut Tennessee \\

A certaines heures de la nuit \\
Quand le cœur de la ville s'est endormi \\
Il flotte un sentiment comme une envie \\
Ce rêve en nous, avec ses mots à lui \\
Quelque chose de Tennessee \\
Oh oui Tennessee \\
Y a quelque chose en nous de Tennessee... \\

\newpage
\normalsize

\h*{Une chanson douce}
\begin{multicols}{2}
Une chanson douce \\
Que me chantait ma maman, \\
En suçant mon pouce \\
J'écoutais en m'endormant. \\
Cette chanson douce, \\
Je veux la chanter pour toi \\
Car ta peau est douce \\
Comme la mousse des bois. \\

La petite biche est aux abois. \\
Dans le bois, se cache le loup, \\
Ouh, ouh, ouh ouh ! \\
Mais le brave chevalier passa. \\
Il prit la biche dans ses bras. \\
La, la, la, la. \\

La petite biche, \\
Ce sera toi, si tu veux. \\
Le loup, on s'en fiche. \\
Contre lui, nous serons deux. \\
Une chanson douce \\
Pour tous les petits enfants \\
Une chanson douce \\
Que me chantait ma maman. \\

O le joli conte que voilà, \\
La biche, en femme, se changea, \\
La, la, la, la \\
Et dans les bras du beau chevalier, \\
Belle princesse elle est restée, \\
A tout jamais \\

La belle princesse \\
Avait tes jolis cheveux, \\
La même caresse \\
Se lit au fond de tes yeux. \\
Cette chanson douce \\
Que me chantait ma maman \\
En suçant mon pouce \\
Je l’écoutais en m’endormant. \\
\end{multicols}

\newpage
\normalsize

\h*{Les murs de poussière}
\begin{multicols}{2}
Il rêvait d'une ville étrangère \\
Une ville de filles et de jeux \\
Il voulait vivre d'autres manières \\
Dans un autre milieu \\
Il rêvait sur son chemin de pierres \\
"Je partirai demain, si je veux \\
J'ai la force qu'il faut pour le faire \\
Et j'irai trouver mieux" \\

Il voulait trouver mieux \\
Que son lopin de terre \\
Que son vieil arbre tordu au milieu \\
Trouver mieux que la douce lumière \\
du soir \\
Près du feu \\
Qui réchauffait son père \\
Et la troupe entière de ses aïeux \\
Le soleil sur les murs de poussière \\
Il voulait trouver mieux... \\

Il a fait tout le tour de la terre \\
Il a même demandé à Dieu \\
Il a fait tout l'amour de la terre \\
Il n'a pas trouvé mieux \\
Il a croisé les rois de naguère \\
Tout drapés de diamants et de feu \\
Mais dans les châteaux des rois de \\
naguère \\
Il n'a pas trouvé mieux... \\

Il n’a pas trouvé mieux \\
Que son lopin de terre \\
Que son vieil arbre tordu au milieu \\
Trouver mieux que la douce lumière \\
du soir \\
Près du feu \\
Qui réchauffait son père \\
Et la troupe entière de ses aïeux \\
Le soleil sur les murs de poussière \\
Il n'a pas trouvé mieux... \\

Il a dit "Je retourne en arrière \\
Je n'ai pas trouvé ce que je veux" \\
Il a dit "Je retourne en arrière" \\
Il s'est brûlé les yeux \\
Il s’est brûlé les yeux \\
Sur son lopin de terre \\
Sur son vieil arbre tordu au milieu \\
Aux reflets de la douce lumière \\
Du soir près du feu \\
Qui réchauffait son père \\
Et la troupe entière de ses aïeux \\
Au soleil sur les murs de poussière \\
Il s’est brûlé les yeux. \\
\end{multicols}

\newpage
\normalsize

\h*{Petite Marie}
Petite Marie, je parle de toi parc'qu'avec, ta petite voix, tes petite manies \\
Tu as versé sur ma vie des milliers de roses \\

Petite furie, je me bats pour toi pour que dans dix mille ans de ça \\
On se retrouve à l'abri sous un ciel aussi joli que de millier de roses \\

\begin{bfseries}
[Refrain:]\\
Je viens du ciel et les étoiles entre elles ne parlent que de toi \\
D'un musicien qui fait jouer ses mains sur un morceau de bois \\
De leur amour plus bleu que le ciel autour \\
\end{bfseries}

Petite Marie, je t'attends transi sous une tuile de ton toit \\
Le vent de la nuit froide me renvoie la ballade que j'avais écrite pour toi \\
Petite furie, tu dis que la vie c'est une bague à chaque doigt \\
Au soleil de Floride, moi mes poches sont vides et mes yeux pleurent de froid \\

\textbf{[Refrain]}\\

Dans la pénombre de ta rue, petite Marie, m'entends-tu ? \\
Je n'attends plus que toi pour partir... \\
Dans la pénombre de ta rue, petite Marie, m'entends-tu ? \\
Je n'attends plus que toi pour partir... \\

\newpage
\normalsize

\h*{Ça fait rire les oiseaux}

\begin{bfseries}
[Refrain:]\\
Ça fait rir' les oiseaux, ça fait chanter les abeilles. \\
Ça chasse les nuages et fait briller le soleil. \\
Ça fait rir' les oiseaux et danser les écureuils. \\
Ça rajoute des couleurs aux couleurs de l'arc-en-ciel. \\
Ça fait rir' les oiseaux, \\
Oh, oh, oh, rir' les oiseaux \\
\end{bfseries}

Une chanson d'amour, c'est comme un looping en avion : \\
Ça fait battre le cœur des filles et des garçons. \\
Une chanson d'amour, c'est l'oxygèn' dans la maison. \\
Tes pieds n'touch'nt plus par terre, t'es en lévitation. \\
Si y a d' la pluie dans ta vie, le soir te fait peur. \\
La musique est là pour ça. \\
Y a toujours une mélodie pour des jours meilleurs. \\
Allez, tape dans tes mains, ça porte bonheur. \\
C'est magique, un refrain qu'on reprend tous en chœur. \\

\textbf{[Refrain]}\\

T'es revenu chez toi la tête pleine de souvenirs : \\
Des soirs au clair de lune, des moments de plaisir. \\
T'es revenu chez toi et tu veux déjà repartir \\
Pour trouver l'aventure qui n'arrête pas de finir. \\
Si y a du gris dans ta nuit, des larmes dans ton cœur. \\
La musique est là pour ça. \\
Y a toujours une mélodie pour des jours meilleurs. \\
Allez, tape dans tes mains ça porte bonheur. \\
C'est magique, un refrain qu'on reprend tous en chœur. \\

\textbf{[Refrain 2x]}

\newpage
\normalsize

\h*{L’Amérique}

Mes amis, je dois m'en aller \\
Je n'ai plus qu'à jeter mes clés \\
Car elle m'attend depuis que je suis né \\
L'Amérique, l’Amérique \\

J'abandonne sur mon chemin \\
Tant de choses que j'aimais bien \\
Cela commence par un peu de chagrin \\
L'Amérique, l’Amérique \\

\begin{bfseries}
[Refrain:]\\
L'Amérique, l'Amérique, je veux l'avoir et je l'aurai \\
L'Amérique, l'Amérique, si c'est un rêve, je le saurai \\
Tous les sifflets des trains, toutes les sirènes des bateaux \\
M'ont chanté cent fois la chanson de l'Eldorado \\
De l'Amérique \\
\end{bfseries}

Mes amis, je vous dis adieu \\
Je devrais vous pleurer un peu \\
Pardonnez-moi si je n'ai dans les yeux \\
Que l'Amérique, l’Amérique \\

Je reviendrai je ne sais pas quand \\
Cousu d'or et brodé d'argent \\
Ou sans un sou, mais plus riche qu'avant \\
De l'Amérique \\

\textbf{[Refrain]}\\


\newpage
\normalsize

\h*{Mon vieux}

\begin{multicols}{2}
Dans son vieux pardessus râpé \\
Il s'en allait l'hiver, l'été \\
Dans le petit matin frileux \\
Mon vieux. \\

Y avait qu'un dimanche par semaine \\
Les autres jours, c'était la graine \\
Qu'il allait gagner comme on peut \\
Mon vieux. \\

L'été, on allait voir la mer \\
Tu vois c'était pas la misère \\
C'était pas non plus l'paradis \\
Hé oui tant pis. \\

Dans son vieux pardessus râpé \\
Il a pris pendant des années \\
L'même autobus de banlieue \\
Mon vieux. \\

L'soir en rentrant du boulot \\
Il s'asseyait sans dire un mot \\
Il était du genre silencieux \\
Mon vieux. \\

Les dimanches étaient monotones \\
On n'recevait jamais personne \\
Ça n'le rendait pas malheureux \\
Je crois, mon vieux. \\

Dans son vieux pardessus râpé \\
Les jours de paye quand il rentrait \\
On l'entendait gueuler un peu \\
Mon vieux. \\

Nous, on connaissait la chanson \\
Tout y passait, bourgeois, patrons, \\
La gauche, la droite, même le bon Dieu \\
Avec mon vieux. \\

Chez nous y'avait pas la télé \\
C'est dehors que j'allais chercher \\
Pendant quelques heures l'évasion \\
Tu sais, c'est con! \\

Dire que j'ai passé des années \\
A côté de lui sans le regarder \\
On a à peine ouvert les yeux \\
Nous deux. \\

J'aurais pu c'était pas malin \\
Faire avec lui un bout d'chemin \\
Ça l'aurait pt'être rendu heureux \\
Mon vieux. \\

Mais quand on a juste quinze ans \\
On n'a pas le cœur assez grand \\
Pour y loger toutes ces choses-là \\
Tu vois. \\

Maintenant qu'il est loin d'ici \\
En pensant à tout ça, j'me dis \\
J'aim'rais bien qu'il soit près de moi \\
Papa… \\

\end{multicols}

\newpage
\normalsize
\vbox{
\begin{minipage}[t][0.4\textheight][t]{\textwidth}
\vspace{0.02\textheight}
\h*{J’te le dis quand-même}
\normalsize
\begin{multicols}{2}
On aurait pu se dire tout ça \\
Ailleurs qu'au café d'en bas, \\
Que t'allais p't êt' partir \\
Et p't êt' même pas rev'nir, \\
Mais en tout cas, c' qui est sûr, \\
C'est qu'on pouvait en rire. \\

Alors on va s' quitter comme ça, \\
Comme des cons d'vant l' café d'en bas. \\
Comme dans une série B, \\
On est tous les deux mauvais. \\
On s'est moqué tellement d' fois \\
Des gens qui faisaient ça. \\

Mais j' trouve pas d' refrain à notre histoire. \\
Tous les mots qui m' viennent sont dérisoires. \\
J' sais bien qu' j' l'ai trop dit, \\
Mais j' te l' dis quand même... je t'aime. \\

J' voulais quand même te dire merci \\
Pour tout le mal qu'on s'est pas dit. \\
Certains rigolent déjà. \\
J' m'en fous, j' les aimais pas. \\
On avait l'air trop bien. \\
Y en a qui n' supportent pas. \\

Mais j' trouve pas d' refrain à notre histoire. \\
Tous les mots qui m' viennent sont dérisoires. \\
J' sais bien qu' j' l' ai trop dit, \\
Mais j' te l' dis quand même... je t'aime. \\
\end{multicols}
\end{minipage}

\vspace{0.12\textheight}
\begin{minipage}[t][0.55\textheight][t]{\textwidth}
\h*{Diego, libre dans sa tête}
\begin{multicols}{2}
\normalsize
Derrière des barreaux \\
Pour quelques mots \\
Qu'il pensait si fort \\
Dehors il fait chaud \\
Des milliers d'oiseaux \\
S'envolent sans effort \\

Quel est ce pays \\
Où frappe la nuit \\
La loi du plus fort ? \\

\columnbreak

Diego, libre dans sa tête \\
Derrière sa fenêtre \\
S'endort peut-etre... \\

Et moi qui danse ma vie \\
Qui chante et qui rit \\
Je pense à lui \\

Diego, libre dans sa tête \\
Derrière sa fenêtre \\
Déjà mort peut-être... \\
\end{multicols}
\end{minipage}
}

\end{document}
