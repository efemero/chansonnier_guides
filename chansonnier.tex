% Auth: Nicklas Vraa
% Docs: https://github.com/NicklasVraa/LiX
% Everything you need to know about this template is found in on the github repository above. Stars are very appreciated.

\documentclass{novel}

\usepackage{tocloft}

\lang      {french}
\title     {Chansonnier}
\subtitle  {Guides 31\textsuperscript{ème} unité Saint-François de Gembloux\\ Compagnie des Quatre Vents}
\cover     {resources/trefles_front.jpg}{resources/trefles_back.jpg}
\author    {}
\license   {CC}{by-sa}{4.0}{La licence ne se rapporte qu’à la structure du chansonnier, les différentes chansons gardent leur copyright original.}
%\edition   {1}{2024}
\size{b5}
\margins{15mm}

%\note{Lorem ipsum dolor sit amet, consectetur adipiscing elit. Praesent porttitor est arcu, sed euismod metus imperdiet iaculis. Quisque vestibulum molestie nulla, non consectetur tellus mollis a. Nunc commodo magna a elit commodo dignissim.}


\begin{document}
\pagenumbering{gobble}

\renewcommand{\cftchapleader}{\cftdotfill{\cftdotsep}}
\toc


\h*{Chant de rassemblement}
\setcounter{page}{1}
\pagenumbering{arabic}
\large

Nous sommes les guides de Gembloux, guides de Gembloux\\
Qui sommes toujours prêtes à tout, oui prêtes à tout\\
On est ici pour s’amuser, pour s’amuser,\\
Et passer une super année, super année\\
Année, année, année, année, année.\\

Se découvrir et s’enrichir, et s’enrichir,\\
Se retrouver avec plaisir, avec plaisir\\
Penser déjà au prochain camp, au prochain camp,\\
Qui sera bien c’est évident, c’est évident,\\
Au camp, au camp, au camp, au camp, au camp.\\


Animées par not’ super staff, not’ super staff,\\
On n’a pas d’autre rime en “aff”, d’autre rime en “aff”,\\
On ne se doit plus qu’d’apporter, plus qu’d’apporter,\\
Notre bonne humeur et notre gaité et notre gaité,\\
Gaité, gaité, gaité, gaité, gaité.\\


La chanson va se terminer, se terminer,\\
Mais il faut pas vous inquiéter, vous inquiéter,\\
On va vous la chanter encore, chanter encore,\\
Et la recommencer plus fort, encore plus FORT,\\
PLUS FORT, PLUS FORT, PLUS FORT, PLUS FORT, PLUS FORT !

\newpage
\normalsize{
\vbox{
    \begin{minipage}[t][0.4\textheight][t]{\textwidth}
    \vspace{0.05\textheight}
        \h*{Cantique des patrouilles}

\begin{multicols}{2}
Seigneur, rassemblées près des tentes\\
Pour saluer la fin du jour\\
Tes guides laissent leur voix chantantes\\
Monter vers Toi, pleines d'amour\\
Tu dois aimer l'humble prière\\
Qui de ce camp s'en va monter\\
Ô Toi qui n'avais sur la terre\\
Pas de maison pour t'abriter\\

\begin{bfseries}
[Refrain:]\\
Nous venons toutes les patrouilles\\
Te prier pour Te servir mieux\\
Vois au bois silencieux\\
Tes guides qui s'agenouillent\\
Bénis-les, ô Jésus dans les cieux\\
\end{bfseries}
\columnbreak

Merci de ce jour d'existence\\
Où ta bonté nous conserva\\
Merci de ta sainte présence\\
Qui de tout mal nous préserva\\
Merci du bien fait par les guides\\
Merci des conseils reçus\\
Merci de l'amour qui nous groupe\\
Comme des sœurs, ô Jésus.\\

\textbf{[Refrain]}\\

\end{multicols}
    \end{minipage}

    \nointerlineskip
    \begin{minipage}[b][0.60\textheight][t]{\textwidth}
    \vspace{0.1\textheight}

\h*{Chant de la promesse}
\begin{multicols}{2}
\begin{enumerate}
\item Devant tous je m'engage\\
Sur mon honneur\\
Et je te fais hommage\\
De moi, Seigneur.\\

\begin{bfseries}
[Refrain]:\\
Je veux t'aimer sans cesse\\
De plus en plus\\
Protège ma Promesse\\
Seigneur Jésus.\\
\end{bfseries}

\item Je jure de te suivre\\
En fier chrétien\\
Et tout entier je livre\\
Mon cœur au tien\\

\item Fidèle à ma Patrie\\
Je le serai\\
Tous les jours de ma vie\\
Je servirai.\\


\item Je suis de tes apôtres\\
Et chaque jour\\
Je veux aider les autres\\
Pour ton amour\\

\item Ta Règle a sur nous-mêmes\\
Un droit sacré.\\
Je suis faible tu m'aimes\\
Je maintiendrai.\\
\end{enumerate}
\end{multicols}
    \end{minipage}
}
}
\newpage
\normalsize

\h*{Il est libre Max}
Il met de la magie, mine de rien, dans tout ce qu'il fait\\
Il a le sourire facile, même pour les imbéciles\\
Il s'amuse bien, il n'tombe jamais dans les pièges\\
Il n'se laisse pas étourdir par les néons des manèges\\
Il vit sa vie sans s'occuper des grimaces\\
Que font autour de lui les poissons dans la nasse\\

Il est libre Max ! Il est libre Max !\\
Y'en a même qui disent qu'ils l'ont vu voler\\

Il travaille un p'tit peu quand son corps est d'accord\\
Pour lui faut pas s'en faire, il sait doser son effort\\
Dans l'panier de crabes, il joue pas les homards\\
Il n'cherche pas à tout prix à faire des bulles dans la mare\\

Il r'garde autour de lui avec les yeux de l'amour\\
Avant qu't'aies rien pu dire, il t'aime déjà au départ\\
Il n'fait pas de bruit, il n'joue pas du tambour\\
Mais la statue de marbre lui sourit dans la cour\\

Et bien sûr toutes les filles lui font les yeux de velours\\
Lui, pour leur faire plaisir, il leur raconte des histoires\\
Il les emmène par delà les labours\\
Chevaucher des licornes à la tombée du soir\\

Comme il n'a pas d'argent pour faire le grand voyageur\\
Il va parler souvent aux habitants de son cœur\\
Qu'est ce qu'ils s'racontent, c'est ça qu'il faudrait savoir\\
Pour avoir comme lui autant d'amour dans le regard\\

Il est libre Max ! Il est libre Max !\\
Y'en a même qui disent qu'ils l'ont vu voler

\newpage
\large
\h*{The lion sleeps tonight}

Ah-wee-ooh-wee-ooo, wee--ee-ee-ee-oo, we-um-um-a-way\\
Wee-ooh-wee-ooo, wee--ee-ee-ee-oo, we-um-um-a-way\\

Awimobawe, awimbawe, awimbawe, awimbawe, awimobawe, awimbawe, awimbawe \\

In the jungle the mighty jungle the lion sleeps tonight\\
In the jungle the quiet jungle the lion sleeps tonight \\

Woo-oo-oo-ooo, we-um-um-a-way\\
Awimobawe, awimbawe, awimbawe, awimbawe, awimobawe, awimbawe, awimbawe \\

Near the village the peaceful village the lion sleeps tonight\\
Near the village the quiet village the lion sleeps tonight \\

Woo-oo-oo-ooo, we-um-um-a-way\\
Awimobawe, awimbawe, awimbawe, awimbawe, awimobawe, awimbawe, awimbawe \\

Hush my darling, don't fear my darling. The lion sleeps tonight.\\
Hush my darling, don't fear my darling. The lion sleeps tonight\\

Awimobawe, awimbawe, awimbawe, awimbawe, awimobawe, awimbawe, awimbawe

\newpage
\small{

\vbox{
    \begin{minipage}[t][0.5\textheight][t]{\textwidth}
    \vspace{0.00\textheight}
\h*{Tous les cris, les S.O.S.}

\begin{multicols}{2}

Comme un fou va jeter à la mer\\
Des bouteilles vides et puis espère\\
Qu'on pourra lire à travers, S.O.S. écrit avec de l'air\\
Pour te dire que je me sens seul\\
Je dessine à l'encre vide, un désert\\

Et je cours, je me raccroche à la vie\\
Je me saoule avec le bruit des corps qui m'entourent\\
Comme des lianes nouées de tresses\\
Sans comprendre la détresse, des mots que j'envoie\\

Difficile d'appeler au secours\\
Quand tant de drames nous oppressent\\
Et les larmes nouées de stress\\
Étouffent un peu plus les cris d'amour\\
De ceux qui sont dans la faiblesse\\
Et dans un dernier espoir, disparaissent\\

Et je cours, je me raccroche à la vie\\
Je me saoule avec le bruit des corps qui m'entourent\\
Comme des lianes nouées de tresses\\
Sans comprendre la détresse, des mots que j'envoie\\

\begin{bfseries}
[Refrain]:\\
Tous les cris les S.O.S., partent dans les airs\\
Dans l'eau laissent une trace, dont les écumes font la beauté\\
Pris dans leur vaisseau de verre, les messages luttent\\
Mais les vagues les ramènent\\
En pierres d'étoiles sur les rochers\\
J'ai ramassé les bouts de verre, j'ai recollé tous les morceaux\\
Tout était clair comme de l'eau\\
Contre le passé y a rien à faire, il faudrait changer les héros\\
Dans un monde où le plus beau, reste à faire\\
\end{bfseries}

Et je cours, je me raccroche à la vie\\
Je me saoule avec le bruit des corps qui m'entourent\\
Comme des lianes nouées de tresses\\
Sans comprendre la détresse, des mots que j'envoie\\

\textbf{[Refrain]}

\end{multicols}
\end{minipage}

\nointerlineskip
\begin{minipage}[b][0.55\textheight][t]{\textwidth}
\vspace{0.15\textheight}
\h*{Personne}

\begin{multicols}{2}
J'avais perdu l'habitude, des clés de la solitude\\
J'avais perdu l'amer et les déserts arides\\
Même la chaleur des pull-overs\\
J'avais perdu l'enfer, au paradis...\\

J'avais oublié les refrains, qui nous rappellent à l'ordre\\
Et ton foutu désordre, ce désordre essentiel\\
Mais si confidentiel, l'existence et les roses se fanent\\
Même un lundi, au paradis...\\

\begin{bfseries}
[Refrain]: (2×)\\
Persooonne, ne te remplace\\
Non personne, ne te remplace\\
\end{bfseries}
\columnbreak

C'est un enfer à vivre, mais comment vivre avec\\
Mes envies insensées\\
Car ton armoire est vide, mes rêves me dévorent\\
Et mes draps sont glacés, toutes les nuits...\\

On a plus goût à rien, mais tant besoin de tout\\
C'qui pourrait remplacer un être indélébile\\
On cherche en vain le double, on serait prêt à tout\\
Pour revoir le jour, toutes les nuits...\\

\textbf{[Refrain] (3×)}

\end{multicols}
\end{minipage}
}
}

\newpage
\footnotesize{

\vbox{
    \begin{minipage}[t][0.4\textheight][t]{\textwidth}
\h*{I will survive}

\begin{multicols}{2}

At first I was afraid, I was petrified\\
Kept thinkin' I could never live without you by my side\\
But then I spent so many nights thinkin' how you did me wrong\\
And I grew strong and I learned how to get along\\

And now you're back from outer space\\
And I find you here with that sad look upon your face\\
I should have changed that stupid lock\\
I should have made you leave your key\\
If I'd have known for just one second you'd be back to bother me\\

\begin{bfseries}
[Refrain]:
Go on now, go, walk out the door, just turn around now\\
Cause you're not welcome anymore\\
You’re the one who tried to hurt me with goodbye\\
Did you think I'd crumble, did you think I'd lay down and die?\\
Oh no I will survive\\
Oh, as long as I know how to love I know I'll stay alive\\
I've got all my life to live, I've got all my love to give\\
And I'll survive, I will survive\\
Hey hey\\
\end{bfseries}

It took all the strength I had not to fall apart\\
Kept trying' hard to mend the pieces of my broken heart\\
And I spent oh so many nights, just feeling sorry for myself\\
I used to cry\\
But now I hold my head up high and you see in me somebody new\\
Not that chained up little person still in love with you\\
And so you feel like droppin' in, and just expect me to be free\\
But now I'm savin' all my lovin' for someone who's lovin' me\\

\textbf{[Refrain] (2×)}

\end{multicols}
\end{minipage}

\nointerlineskip
\begin{minipage}[b][0.55\textheight][t]{\textwidth}
\vspace{0.1\textheight}
\h*{Hotel California}

\begin{multicols}{2}
On a dark desert highway, cool wind in my hair\\
Warm smell of colitas rising up through the air\\
Up ahead in the distance, I saw a shimmering light\\
My head grew heavy, and my sight grew dimmer\\
I had to stop for the night\\
There she stood in the doorway;\\
I heard the mission bell\\
And I was thinking to myself,\\
'This could be Heaven or this could be Hell'\\
Then she lit up a candle and she showed me the way\\
There were voices down the corridor, I thought I heard them say...\\

Welcome to the Hotel California\\
Such a lovely place (such a lovely flace)\\
Plenty of room at the Hotel California\\
Any time of year, you can find it here\\

Her mind is Tiffany-twisted, She got the Mercedes Benz\\
She's got a lot of pretty, pretty boys, that she calls friends\\
How they dance in the courtyard, sweet summer sweat.\\
Some dance to remember, some dance to forget\\
So I called up the Captain, 'Please bring me my wine'\\
He said, 'We haven't had that spirit here since 1969'\\
And still those voices are calling from far away\\
Wake you up in the middle of the night\\
Just to hear them say...\\

Welcome to the Hotel California\\
Such a lovely Place (such a lovely face)\\
They livin' it up at the Hotel California\\
What a nice surprise, bring your alibis\\
Mirrors on the ceiling, the pink champagne on ice\\
And she said 'We are all just prisoners here, of our own device'\\
And in the master's chambers, they gathered for the feast\\
They stab it with their steely knives, but they just can't kill the beast\\
Last thing I remember, I was running for the door\\
I had to find the passage back to the place I was before\\
'Relax' said the nightman, We are programed to receive.\\
You can check out any time you like, but you can never leave\\

\end{multicols}
\end{minipage}
}
}

\newpage
\normalsize

\h*{Pour que tu m’aimes encore}
% \begin{multicols}{2}

J'ai compris tous les mots, j'ai bien compris, merci\\
Raisonnable et nouveau, c'est ainsi par ici\\
Que les choses ont changé, que les fleurs ont fané\\
Que le temps d'avant, c'était le temps d'avant\\
Que si tout zappe et lasse, les amours aussi passent\\

Il faut que tu saches\\

\begin{bfseries}
[Refrain]:\\
J'irai chercher ton cœur si tu l'emportes ailleurs\\
Même si dans tes danses d'autres dansent tes heures\\
J'irai chercher ton âme dans les froids dans les flammes\\
Je te jetterai des sorts pour que tu m'aimes encore\\
\end{bfseries}

Fallait pas commencer m'attirer, me toucher\\
Fallait pas tant donner moi je sais pas jouer\\
On me dit qu'aujourd'hui, on me dit que les autres font ainsi\\
Je ne suis pas les autres\\
Avant que l'on s'attache, avant que l'on se gâche\\
% \columnbreak

Je veux que tu saches\\

\textbf{[Refrain]}\\

Je trouverai des langages pour chanter tes louanges\\
Je ferai nos bagages pour d'infinies vendanges\\
Les formules magiques des marabouts d'Afrique\\
J'les dirais sans remords pour que tu m'aimes encore\\

Je m'inventerai reine pour que tu me retiennes\\
Je me ferai nouvelle pour que le feu reprenne\\
Je deviendrai ces autres qui te donnent du plaisir\\
Vos jeux seront les nôtres, si tel est ton désir\\

Plus brillante plus belle pour une autre étincelle\\
Je me changerai en or pour que tu m'aimes encore.
% \end{multicols}

\newpage
\large

\h*{Mon frère}
\begin{multicols}{2}

Toi le frère que je n'ai jamais eu\\
Sais-tu si tu avais vécu\\
Ce que nous aurions fait ensemble\\
Un an après moi, tu serais né\\
Alors on n'se s'rait plus quittés\\
Comme deux amis qui se ressemblent\\
On aurait appris l'argot par cœur\\
J'aurais été ton professeur\\
A mon école buissonnière\\
Sur qu'un jour on se serait battu\\
Pour peu qu'alors on ait connu\\
Ensemble la même première\\

\begin{bfseries}
[Refrain]:\\
Mais tu n'es pas la\\
A qui la faute\\
Pas à mon père\\
Pas à ma mère\\
Tu aurais pu chanter cela\\
\end{bfseries}

\columnbreak
Toi le frère que je n'ai jamais eu\\
Si tu savais ce que j'ai bu\\
De mes chagrins en solitaire\\
Si tu m'avais pas fait faux bond\\
Tu aurais fini mes chansons\\
Je t'aurais appris à en faire\\
Si la vie s'était comportée mieux\\
Elle aurait divisé en deux\\
Les paires de gants, les paires de claques\\
Elle aurait sûrement partagé\\
Les mots d'amour et les pavés\\
Les filles et les coups de matraque\\

\textbf{[Refrain]}\\

Toi le frère que je n'aurais jamais\\
Je suis moins seul de t'avoir fait\\
Pour un instant, pour une fille\\
Je t'ai dérangé, tu me pardonnes\\
Ici quand tout vous abandonne\\
On se fabrique une famille\\
\end{multicols}

\newpage
\large

\h*{Mon fils, ma bataille}

\begin{multicols}{2}
Ça fait longtemps que t'es partie, maintenant\\
Je t'écoute démonter ma vie, en pleurant\\
Si j'avais su qu'un matin\\
Je serai là, sali, jugé, sur un banc\\
Par l'ombre d'un corps\\
Que j'ai serré si souvent\\
Pour un enfant\\

Tu leur dis que mon métier, c'est du vent\\
Qu'on ne sait pas ce que je serai\\
Dans un an\\
S'ils savaient que pour toi\\
Avant de tous les chanteurs j'étais le plus grand\\
Et que c'est pour ça\\
Que tu voulais un enfant\\
Devenu grand\\

\columnbreak
\begin{bfseries}
[Refrain]:\\
Les juges et les lois\\
Ça m'fait pas peur\\
C'est mon fils ma bataille\\
Fallait pas qu'elle s'en aille\\
Je vais tout casser\\
Si vous touchez\\
Au fruit de mes entrailles\\
Fallait pas qu'elle s'en aille\\
\end{bfseries}

Bien sûr c'est elle qui l'a porté\\
Et pourtant\\
C'est moi qui lui construis sa vie lentement\\
Tout ce qu'elle peut dire sur moi\\
N'est rien à côté du sourire qu'il me tend\\
L'absence a ses torts\\
Que rien ne défend\\
C'est mon enfant\\

\textbf{[Refrain] (2×)}
\end{multicols}

\newpage
\h*{Le lundi au soleil}

\begin{multicols}{2}
Regarde ta montre\\
Il est déjà huit heures\\
Embrassons-nous tendrement\\
Un taxi t'emporte\\
Tu t'en vas, mon cœur\\
Parmi ces milliers de gens\\
C'est une journée idéale\\
Pour marcher dans la forêt\\
On trouverait plus normal\\
D'aller se coucher\\
Seuls dans les genêts\\

\begin{bfseries}
[Refrain:]\\
Le lundi au soleil\\
C'est une chose qu'on n'aura jamais\\
Chaque fois c'est pareil\\
C'est quand on est derrière les carreaux\\
Quand on travaille que le ciel est beau\\
Qu'il doit faire beau sur les routes\\
Le lundi au soleil\\

Le lundi au soleil\\
On pourrait le passer à s'aimer\\
Le lundi au soleil\\
On serait mieux dans l'odeur des foins\\
On aimerait mieux cueillir le raisin\\
Ou simplement ne rien faire\\
Le lundi au soleil\\
\end{bfseries}

Toi, tu es à... l'autre bout\\
De cette ville\\
Là-bas, comme chaque jour\\
Les dernières heures\\
Sont les plus difficiles\\
J'ai besoin de ton amour\\
Et puis dans la foule au loin\\
Je te vois, tu me souris\\
Les néons des magasins\\
Sont tous allumés\\
C'est déjà la nuit\\

\textbf{[Refrain]}\\
\end{multicols}

\newpage
\large

\h*{It’s not because you are}

\begin{multicols}{2}
When I have rencontred you,\\
You was a jeune fille au pair,\\
And I put a spell on you,\\
And you roule a pelle to me.\\

Together we go partout\\
On my mob il was super\\
It was friday on my mind,\\
It was story d'amour.\\

\begin{bfseries}
[Refrain:]\\
It is not because you are,\\
I love you because I do\\
C'est pas parc' que you are me qu'I am you.\\
\end{bfseries}

You was really beautiful\\
In the middle of the foule.\\
Don't let me misunderstood,\\
Don't let me sinon I boude.\\

My loving, my marshmallow,\\
You are belle and I are beau\\
You give me all what You have\\
I say thank you, you are bien brave.\\

\textbf{[Refrain]}\\

I wanted marry with you,\\
And make love very beaucoup,\\
To have a max of children,\\
Just like Stone and Charden.\\

But one day that must arrive,\\
Together we disputed.\\
For a stupid story of fric,\\
We decide to divorced.\\

\textbf{[Refrain]}\\

You chialed comme une madeleine,\\
Not me, I have my dignité.\\
You tell me : you are a sale mec !\\
I tell you : poil to the bec !\\

That's comme ça that you thank me\\
To have learning you english ?\\
Eh ! That's not you qui m'a appris,\\
My grand father was rosbeef !\\

\textbf{[Refrain]}\\
\end{multicols}

\newpage
\normalsize

\h*{Hier encore}

\begin{multicols}{2}
Hier encore \\
J'avais vingt ans \\
Je caressais le temps \\
Et jouais de la vie \\
Comme on joue de l'amour \\
Et je vivais la nuit \\
Sans compter sur mes jours \\
Qui fuyaient dans le temps \\

J'ai fait tant de projets \\
Qui sont restés en l'air \\
J'ai fondé tant d'espoirs \\
Qui se sont envolés \\
Que je reste perdu \\
Ne sachant où aller \\
Les yeux cherchant le ciel \\
Mais le cœur mis en terre \\

Hier encore \\
J'avais vingt ans \\
Je gaspillais le temps \\
En croyant l'arrêter \\
Et pour le retenir \\
Même le devancer \\
Je n'ai fait que courir \\
Et me suis essoufflé \\

Ignorant le passé \\
Conjuguant au futur \\
Je précédais de moi \\
Toute conversation \\
Et donnais mon avis \\
Que je voulais le bon \\
Pour critiquer le monde \\
Avec désinvolture \\

Hier encore \\
J'avais vingt ans \\
Mais j'ai perdu mon temps \\
A faire des folies \\
Qui ne me laissent au fond \\
Rien de vraiment précis \\
Que quelques rides au front \\
Et la peur de l'ennui \\

Car mes amours sont mortes \\
Avant que d'exister \\
Mes amis sont partis \\
Et ne reviendront pas \\
Par ma faute j'ai fait \\
Le vide autour de moi \\
Et j'ai gâché ma vie \\
Et mes jeunes années \\

Du meilleur et du pire \\
En jetant le meilleur \\
J'ai figé mes sourires \\
Et j'ai glacé mes pleurs \\
Où sont-ils à présent \\
A présent mes vingt ans? \\
\end{multicols}

\newpage
\large

\h*{Foule sentimentale}

\begin{multicols}{2}
Oh la la la vie en rose \\
Le rose qu'on nous propose \\
D'avoir les quantités d'choses \\
Qui donnent envie d'autre chose \\
Aïe, on nous fait croire \\
Que le bonheur c'est d'avoir \\
De l'avoir plein nos armoires \\
Dérisions de nous dérisoires car \\

\begin{bfseries}
[Refrain:]\\
Foule sentimentale \\
On a soif d'idéal \\
Attirée par les étoiles, les voiles \\
Que des choses pas commerciales \\
Foule sentimentale \\
Il faut voir comme on nous parle \\
Comme on nous parle \\
\end{bfseries}

\columnbreak
Il se dégage \\
De ces cartons d'emballage \\
Des gens lavés, hors d'usage \\
Et tristes et sans aucun avantage \\
On nous inflige \\
Des désirs qui nous affligent \\
On nous prend faut pas déconner dès qu'on est né \\
Pour des cons alors qu'on est \\
Des \\

\textbf{[Refrain]} \\

On nous Claudia Schieffer \\
On nous Paul-Loup Sulitzer \\
Oh le mal qu'on peut nous faire \\
Et qui ravagea la moukère \\
Du ciel dévale \\
Un désir qui nous emballe \\
Pour demain nos enfants pâles \\
Un mieux, un rêve, un cheval \\

\textbf{[Refrain]} \\
\end{multicols}

\newpage
\large

\h*{L’été indien}

Tu sais, je n'ai jamais été aussi heureux que ce matin-là \\
Nous marchions sur une plage un peu comme celle-ci \\
C'était l'automne, un automne où il faisait beau \\
Une saison qui n'existe que dans le Nord de l'Amérique \\
Là-bas on l'appelle l'été indien \\
Mais c'était tout simplement le nôtre \\
Avec ta robe longue tu ressemblais \\
A une aquarelle de Marie Laurencin \\
Et je me souviens, je me souviens très bien \\
De ce que je t'ai dit ce matin-là \\
Il y a un an, y a un siècle, y a une éternité \\

\begin{bfseries}
[Refrain:]\\
On ira où tu voudras, quand tu voudras \\
Et on s'aimera encore, lorsque l'amour sera mort \\
Toute la vie sera pareille à ce matin \\
Aux couleurs de l'été indien \\
\end{bfseries}

Aujourd'hui je suis très loin de ce matin d'automne \\
Mais c'est comme si j'y étais. Je pense à toi. \\
Où es-tu? Que fais-tu? Est-ce que j'existe encore pour toi? \\
Je regarde cette vague qui n'atteindra jamais la dune \\
Tu vois, comme elle je reviens en arrière \\
Comme elle je me couche sur le sable \\
Et je me souviens, je me souviens des marées hautes \\
Du soleil et du bonheur qui passaient sur la mer \\
Il y a une éternité, un siècle, il y a un an \\

\textbf{[Refrain]} 

\newpage
\small
\h*{Dès que le vent soufflera}

\begin{multicols}{3}

C'est pas l'homme qui prend la mer \\
C'est la mer qui prend l'homme, Tatatin \\
Moi la mer elle m'a pris \\
Je m' souviens un Mardi \\
J'ai troqué mes santiags \\
Et mon cuir un peu zone \\
Contre une paire de docksides \\
Et un vieux ciré jaune \\
J'ai déserté les crasses \\
Qui m' disaient «~Sois prudent~» \\
La mer c'est dégueulasse \\
Les poissons baisent dedans \\

\begin{bfseries}
[Refrain:]\\
Dès que le vent soufflera \\
Je repartira \\
Dès que les vents \\
tourneront \\
Nous nous en allerons \\
\end{bfseries}

C'est pas l'homme qui prend la mer \\
C'est la mer qui prend l'homme \\
Moi la mer elle m'a pris \\
Au dépourvu tans pis \\
J'ai eu si mal au cœur \\
Sur la mer en furie \\
Qu' j'ai vomi mon quatre heures \\
Et mon minuit aussi \\
J' me suis cogné partout \\
J'ai dormi dans des draps mouillés \\
Ça m'a coûté ses sous \\
C'est d' la plaisance, c'est le pied \\

\textbf{[Refrain]} \\

Ho ho ho ho ho hissez haut \\
ho ho ho \\

C'est pas l'homme quiprend la mer \\
C'est la mer qui prend l'homme \\
Mais elle prend pas la femme \\
Qui préfère la campagne \\
La mienne m'attend au port \\
Au bout de la jetée \\
L'horizon est bien mort \\
Dans ses yeux délavés \\
Assise sur une bitte \\
D'amarrage, elle pleure \\
Son homme qui la quitte \\
La mer c'est son malheur \\

\textbf{[Refrain]} \\

C'est pas l'homme qui prend la mer \\
C'est la mer qui prends l'homme \\
Moi la mer elle m'a pris \\
Comme on prend un taxi \\
Je ferai le tour du monde \\
Pour voir à chaque étape \\
Si tous les gars du monde \\
Veulent bien m' lâcher la grappe \\
J'irais aux quatre vents \\
Foutre un peu le boxon \\
Jamais les océans \\
N'oublieront mon prénom \\

\textbf{[Refrain]} \\

Ho ho ho ho ho hissez haut \\
ho ho ho \\

C'est pas l'homme qui prend la mer \\
C'est la mer qui prends l'homme \\
Moi la mer elle m'a pris \\
Et mon bateau aussi \\
Il est fier mon navire \\
Il est est beau mon bateau \\
C'est un fameux trois mats \\
Fin comme un oiseau \emph{[Hissez haut]} \\
Tabarly, Pageot \\
Kersauson ou Riguidel \\
Naviguent pas sur des cageots \\
Ni sur des poubelles \\

\textbf{[Refrain]} \\

C'est pas l'homme qui prend la mer \\
C'est la mer qui prends l'homme \\
Moi la mer elle m'a pris \\
Je m' souviens un Vendredi \\
Ne pleure plus ma mère \\
Ton fils est matelot \\
Ne pleure plus mon père \\
Je vis au fil de l'eau \\
Regardez votre enfant \\
Il est parti marin \\
Je sais c'est pas marrant \\
Mais c'était mon destin \\

\textbf{[Refrain] (x3)} \\

Dès que le vent soufflera \\
Nous repartira \\
Dès que les vents \\
tourneront \\
Je me n'en allerons \\
\end{multicols}


\newpage
\normalsize

\h*{Chanson pour Pierrot}

\begin{multicols}{2}
T'es pas né dans la rue \\
T'es pas né dans l' ruisseau \\
T'es pas un enfant perdu \\
Pas un enfant d' salaud, \\
Vu qu' t'es né dans ma tête \\
Et qu' tu vis dans ma peau \\
J'ai construit ta planète \\
Au fond de mon cerveau. \\

\begin{bfseries}
[Refrain:]\\
Pierrot, mon gosse, mon frangin, mon poteau, \\
Mon copain tu m' tiens chaud. \\
Pierrot. \\
\end{bfseries}

Depuis l' temps que j' te rêve, \\
Depuis l' temps que j' t'invente, \\
De pas te voir j'en crève \\
Et j' te sens dans mon ventre. \\
Le jour où tu ramène, \\
J'arrête de boire : promis, \\
Au moins toute une semaine, \\
Ce s'ra dur, mais tant pis. \\

\textbf{[Refrain]} \\

Qu' tu sois fils de princesse, \\
Ou qu' tu sois fils de rien, \\
Tu s'ras fils de tendresse, \\
Tu s'ras pas pas orphelin. \\
Mais j' connais pas ta mère : \\
Je la cherche en vain. \\
Je connais qu' la misère \\
D'être tout seul sur le ch'min. \\

\textbf{[Refrain]} \\

Dans un coin de ma tête \\
Y a déjà ton trousseau : \\
Un jean, une mobylette \\
Une paire de Santiago. \\
T'iras pas à l'école, \\
J' t'apprendrai les gros mots. \\
On jouera au football, \\
On ira au bistrot. \\

\textbf{[Refrain]} \\

Tu t' lav'ras pas les pognes \\
Avant d' venir à table. \\
Et tu m' trait'ras d'ivrogne \\
Quand j' piquerai ton cartable. \\
J' t'apprendrai des chansons \\
Tu les trouveras débiles. \\
T'auras p't' être bien raison \\
Mais j' s'rai vexé quand même. \\

\textbf{[Refrain]} \\

Allez viens mon Pierrot, \\
Tu s'ras l' chef de ma bande. \\
J' te r'filerai mon couteau, \\
J' t'apprendrai la truande. \\
Allez viens mon copain, \\
J' t'ai trouvé une maman : \\
Tous les trois ça s'ra bien \\
Allez viens, je t'attends. \\

\textbf{[Refrain]} \\

\end{multicols}

\newpage
\normalsize

\h*{La bohème}

\begin{multicols}{2}
Je vous parle d'un temps \\
Que les moins de vingt ans \\
Ne peuvent pas connaître \\
Montmartre en ce temps-là \\
Accrochait ses lilas \\
Jusque sous nos fenêtres \\
Et si l'humble garni \\
Qui nous servait de nid \\
Ne payait pas de mine \\
C'est là qu'on s'est connu \\
Moi qui criait famine \\
Et toi qui posais nue \\

La bohème, la bohème \\
Ça voulait dire on est heureux \\
La bohème, la bohème \\
Nous ne mangions qu'un jour sur \\
deux \\

Dans les cafés voisins \\
Nous étions quelques-uns \\
Qui attendions la gloire \\
Et bien que miséreux \\
Avec le ventre creux \\
Nous ne cessions d'y croire \\
Et quand quelque bistro \\
Contre un bon repas chaud \\
Nous prenait une toile \\
Nous récitions des vers \\
Groupés autour du poêle \\
En oubliant l'hiver \\

La bohème, la bohème \\
Ça voulait dire tu es jolie \\
La bohème, la bohème \\
Et nous avions tous du génie \\

Souvent il m'arrivait \\
Devant mon chevalet \\
De passer des nuits blanches \\
Retouchant le dessin \\
De la ligne d'un sein \\
Du galbe d'une hanche \\
Et ce n'est qu'au matin \\
Qu'on s'asseyait enfin \\
Devant un café-crème \\
Epuisés mais ravis \\
Fallait-il que l'on s'aime \\
Et qu'on aime la vie \\

La bohème, la bohème \\
Ça voulait dire on a vingt ans \\
La bohème, la bohème \\
Et nous vivions de l'air du temps \\

Quand au hasard des jours \\
Je m'en vais faire un tour \\
A mon ancienne adresse \\
Je ne reconnais plus \\
Ni les murs, ni les rues \\
Qui ont vu ma jeunesse \\
En haut d'un escalier \\
Je cherche l'atelier \\
Dont plus rien ne subsiste \\
Dans son nouveau décor \\
Montmartre semble triste \\
Et les lilas sont morts \\

La bohème, la bohème \\
On était jeunes, on était fous \\
La bohème, la bohème \\
Ça ne veut plus rien dire du tout \\
\end{multicols}

\newpage
\normalsize

\h*{L’Aziza}
Petite rue de Casbah au milieu de Casa \\
Petite brune enroulée d'un drap court autour de moi \\
Ses yeux remplis de pourquoi cherchent une réponse en moi \\
Elle veut vraiment que rien ne soit sûr dans tout ce qu'elle croit \\

\begin{bfseries}
[Refrain:]\\
Ta couleur et tes mots tout me va \\
Que tu vives ici ou là-bas \\
Danse avec moi \\
Si tu crois que ta vie est là \\
Ce n'est pas un problème pour moi \\
L'Aziza, \\
Je te veux si tu veux de moi \\
\end{bfseries}

Et quand tu marches le soir ne trembles pas a- a- \\
Laisse glisser les mauvais regards qui pèsent sur toi, \\
L'Aziza \\
Ton étoile jaune c'est ta peau tu n'as pas le choix \\
Ne la porte pas comme on porte un fardeau, ta force c'est ton droit \\

\textbf{[Refrain]} \\

L’Aziza \\
Ta couleur et tes mots tout me va \\
Danse avec moi \\
Que tu vives ici ou là-bas \\
Ce n'est pas un problème pour moi \\
L'Aziza, \\
Si tu crois que ta vie est la \\
Il n'y a pas de loi contre ça \\
L'Aziza, fille enfant de prophète roi \\

\textbf{[Refrain]}

\newpage
\normalsize

\h*{Alexandrie, Alexandra}

\begin{multicols}{2}
Voiles sur les filles \\
Barques sur le Nil \\
Je suis dans ta vie \\
Je suis dans tes bras \\
Alexandra Alexandrie \\

Alexandrie où l'amour danse avec la nuit \\
J'ai plus d'appétit \\
Qu'un Barracuda \\
Je boirai tout le Nil si tu n'me regarde pas \\
Je boirai tout le Nil si tu n’me retiens pas \\
Alexandrie \\
Alexandra \\
Alexandrie où l'amour danse au fond des draps \\
Ce soir j'ai de la fièvre et toi tu meurs de froid \\

\begin{bfseries}
[Refrain:]\\
Les sirènes du port d'Alexandrie \\
Chantent encore la même mélodie wowo \\
La lumière du phare d'Alexandrie \\
Fait naufrager les papillons de ma jeunesse. \\
\end{bfseries}

Voiles sur les filles \\
Barques sur le Nil \\
Je suis dans ta vie \\
Je suis dans tes bras \\
Alexandra Alexandrie \\
Alexandrie où tout commence et tout finit \\
J'ai plus d'appétit \\
Qu'un Barracuda \\
Je te mangerai crue si tu n'me \\
reviens pas \\
Je te mangerai crue si tu n’me \\
retiens pas \\
Alexandrie \\
Alexandra \\

Alexandrie ce soir je danse dans tesdraps \\
Je te mangerai crue si tu n'me retiens pas \\

\textbf{[Refrain]} \\

Ah Aaah \\
Ah Aaah \\

Voiles sur les filles \\
Et barques sur le Nil \\
Alexandrie Alexandra \\
Ce soir j'ai la fièvre et tu meurs de froid \\
Ce soir je dans', je dans', je danse dans tes draps. \\
\end{multicols}

\newpage
\normalsize

\h*{Le pouvoir des fleurs}

\begin{multicols}{2}
Je m'souviens on avait des projets \\
pour la terre \\
pour les hommes comme la nature \\
faire tomber les barrières, les murs, \\
les vieux parapets d'Arthur \\
fallait voir \\
imagine notre espoir \\
on laissait nos cœurs \\
au pouvoir des fleurs \\
jasmin, lilas, \\
c'étaient nos divisions nos soldats \\
pour changer tout ça \\

\begin{bfseries}
[Refrain:]\\
changer le monde \\
changer les choses avec des \\
bouquets de roses \\
changer les femmes \\
changer les hommes \\
avec des géraniums \\
\end{bfseries}

je m'souviens, on avait des chansons, \\
des paroles \\
comme des pétales et des corolles \\
qu'écoutait en rêvant \\
la petite fille au tourne-disque folle \\
le parfum \\
imagine le parfum \\
l'Eden, le jardin, \\
c'était pour demain, \\
mais demain c'est pareil, \\
le même désir veille \\
là tout au fond des cœurs \\
tout changer en douceur \\

changer les âmes \\
changer les cœurs avec des bouquets \\
de fleurs \\
la guerre au vent \\
l'amour devant \\
grâce à des fleurs des champs \\

ah! sur la terre \\
il y a des choses à faire \\
pour les enfants, les gens, les \\
éléphants \\
ah! tant de choses à faire \\
moi pour \\
te donner du cœur \\
je t'envoie des fleurs \\

tu verras qu'on aura des foulards, \\
des chemises \\
et que voici les couleurs vives \\
et que même si l'amour est parti \\
ce n'est que partie remise \\
pour les couleurs, les accords, les \\
parfums \\
changer le vieux monde \\
pour faire un jardin \\
tu verras \\
tu verras \\
le pouvoir des fleurs \\
y a une idée pop dans mon air \\

\textbf{[Refrain] (x2)} \\

changer les... \\
Changer les cœurs... \\
\end{multicols}

\newpage
\normalsize

\h*{Petit clown}

Tu fais rire les gens \\
Tu fais rire les enfants \\
Mais le soir en pleurant \\
Tu dois revenir sur terre \\
Tu dois redevenir petit clown de misère \\
Et tu devras reporter ton masque demain \\

Petit clown de misère \\
Ces yeux que je connais \\
Ont perdu de la lumière \\
D’avoir tant et tant pleuré \\
Et ce bleu pris à la mer \\
Peu à peu s’est fané \\
Et ton cœur est fatigué \\
De faire rire et d’amuser \\

Aujourd’hui les enfants, \\
Ne prennent plus la peine \\
Même de faire semblant \\
De rire de tes scènes \\
Et ton cœur bien souvent \\
Aussi froid que la foule \\
Ne peut plus supporter \\
Les larmes sur ta joue \\

Mais un jour en riant \\
Tu quitteras cette terre \\
Mais un jour en riant \\
Tu quitteras nos misères \\
Et je sais que là-haut \\
Tu feras sourire les anges \\
Car tu auras trouvé \\
La paix que tu cherchais (bis)


\newpage
\normalsize

\h*{L’oiseau et l’enfant}

Comme un enfant aux yeux de lumière \\
Qui voit passer au loin les oiseaux \\
Comme l'oiseau bleu survolant la Terre \\
Vois comme le monde, le monde est beau \\

Beau le bateau, dansant sur les vagues \\
Ivre de vie, d'amour et de vent \\
Belle la chanson naissante des vagues \\
Abandonnée au sable blanc \\

Blanc l'innocent, le sang du poète \\
Qui en chantant, invente l'amour \\
Pour que la vie s'habille de fête \\
Et que la nuit se change en jour \\

Jour d'une vie où l'aube se lève \\
Pour réveiller la ville aux yeux lourds \\
Où les matins effeuillent les rêves \\
Pour nous donner un monde d'amour \\

L'amour c'est toi, l'amour c'est moi \\
L'oiseau c'est toi, l'enfant c'est moi. \\

Comme un enfant aux yeux de lumière \\
Qui voit passer au loin les oiseaux \\
Comme l'oiseau bleu survolant la terre \\
Nous trouverons ce monde d'amour \\

L'amour c'est toi, l'amour c'est moi \\
L'oiseau c'est toi, l'enfant c'est moi \\

L'amour c'est toi, l'amour c'est moi \\
L'oiseau c'est toi, l'enfant c'est moi.


\newpage
\normalsize

\h*{Non, non, rien n’a changé}

\begin{multicols}{2}
C'est l'histoire d'une trêve \\
Que j'avais demandée \\
C'est l'histoire d'un soleil \\
Que j'avais espéré \\
C'est l'histoire d'un amour \\
Que je croyais vivant \\
C'est l'histoire d'un beau jour \\
Que moi, petit enfant \\
Je voulais très heureux \\
Pour toute la planète \\
Je voulais, j'espérais \\
Que la paix règne en maître \\
En ce soir de Noël \\
Mais tout a continué \\
Mais tout a continué \\
Mais tout a continué \\

\begin{bfseries}
[Refrain:]\\
Non, non, rien n'a changé \\
Tout, tout a continué \\
Non, non, rien n'a changé \\
Tout, tout a continué \\
Héhé! Héhé! \small \\
\end{bfseries}

\normalsize
Et pourtant bien des gens \\
Ont chanté avec nous \\
Et pourtant bien des gens \\
Se sont mis à genoux \\
Pour prier, oui pour prier \\
Pour prier, oui pour prier \\
Mais j'ai vu tous les jours \\
A la télévision \\
Même le soir de Noël \\
Des fusils, des canons \\
J'ai pleuré, oui j'ai pleuré \\
J'ai pleuré \\
Qui pourra m'expliquer que... \\



\textbf{[Refrain]} \\



Moi je pense à l'enfant \\
Entouré des soldats \\
Moi je pense à l'enfant \\
Qui demande pourquoi \\
Tout le temps, oui tout le temps \\
Tout le temps, oui tout le temps \\
Moi je pense à tout ça \\
Mais je ne devrais pas \\
Toutes ces choses-là \\
Ne me regardent pas \\
Et pourtant, oui et pourtant \\
Et pourtant, je chante, je chante... \\

\textbf{[Refrain]} \\

C'est l'histoire d'une trêve \\
Que j'avais demandée \\
C'est l'histoire d'un soleil \\
Que j'avais espéré \\
C'est l'histoire d'un amour \\
Que je croyais vivant \\
C'est l'histoire d'un beau jour \\
Que moi, petit enfant \\
Je voulais très heureux \\
Pour toute la planète \\
Je voulais, j'espérais \\
Que la paix règne en maître \\
En ce soir de Noël \\
Mais tout a continué \\
Mais tout a continué \\
Mais tout a continué \\

\textbf{[Refrain]} \\

Héhé! Héhé!
\end{multicols}

\newpage
\Large

\h*{Liberté}

Ne nous parlez plus de héros \\
Ne nous parlez plus de révolutions \\
Dites-nous combien il reste encore \\
Vous laissez derrière vous des rêves pillés \\
Des mondes gaspillés, des soleils brûlés, \\
Laissez-nous créer une arme d’amour \\
Une bombe à lumière, un fusil à fleurs \\
Une vie sans barrière. \\
Laissez-nous rêver d’un enfant président \\
D’un roi sans couronne, d’un Jésus indien \\
D’un Dieu qui pardonne même ceux qui l’oublient. \\
Ne nous parlez plus de héros \\
Ne nous parlez plus de révolutions \\
Dites-nous combien il reste encore \\
Vous laissez derrière vous des mères matraquées, \\
Des lunes piétinées, des hommes qui mourraient \\
Pour la Liberté.

\newpage
\large

\h*{Rêve ta vie}

\begin{bfseries}
[Refrain:]\\
Rêve, rêve, rêve ta vie (bis) \\
Dans ton sommeil \\
Et au réveil \\
Vis ton rêve ! \\
\end{bfseries}

Rêve un matin d'été, un grand champ de blé \\
Un ami trouvé en chemin \\
Rêve un ruisseau d'eau claire, près d'une clairière \\
Un boulanger qui donne son pain \\
Ferme les yeux et dis-toi que c'est arrivé \\
Il faut y croire très fort et ne pas en douter \\

\textbf{[Refrain]} \\

Rêve un grand feu de bois, quand dehors il fait froid \\
Un ennemi qui tend la main \\
Rêve que tu cours dans les champs, c'est déjà le printemps \\
La récolte est bonne au Pakistan \\
Ferme les yeux et dis-toi que c'est arrivé \\
Il faut y croire très fort et ne pas en douter

\newpage
\normalsize

\h*{A nos actes manqués}

A tous mes loupés, mes ratés, mes vrais soleils \\
Tous les chemins qui me sont passés à côté \\
A tous mes bateaux manqués, mes mauvais sommeils \\
A tous ceux que je n'ai pas été \\

Aux malentendus, aux mensonges, à nos silences \\
A tous ces moments que j'avais cru partager \\
Aux phrases qu'on dit trop vite et sans qu'on les pense \\
A celles que je n'ai pas osées \\
A nos actes manqués \\

Aux années perdues à tenter de ressembler \\
A tous les murs que je n'aurais pas su briser \\
A tout c'que j'ai pas vu tout près, juste à côté \\
Tout c'que j'aurais mieux fait d'ignorer \\

Au monde, à ses douleurs qui ne me touchent plus \\
Aux notes, aux solos que je n'ai pas inventés \\
Tous ces mots que d'autres ont fait rimer et qui me tuent \\
Comme autant d'enfants jamais portés \\
A nos actes manqués \\

Aux amours échouées de s'être trop aimé \\
Visages et dentelles croisés justes frôlés \\
Aux trahisons que j'ai pas vraiment regrettées \\
Aux vivants qu'il aurait fallu tuer \\

A tout ce qui nous arrive enfin, mais trop tard \\
A tous les masques qu'il aura fallu porter \\
A nos faiblesses, à nos oublis, nos désespoirs \\
Aux peurs impossibles à échanger \\

A nos actes manqués


\newpage
\normalsize

\h*{Je te donne}

I can give you a voice, bred with rythms and soul \\
the heart of a Welsh boy who's lost his home \\
put it in harmony , let the words ring \\
carry your thoughts in the song we sing \\
Je te donne mes notes , je te donne mes mots \\
quand ta voix les emporte a ton propre tempo \\
une épaule fragile et solide a la fois \\
ce que j'imagine et ce que je crois . \\

Je te donne toutes mes differences, \\
tous ces défauts qui sont autant de chances \\
on sera jamais des standards, des gens bien comme il faut \\
je te donne ce que j'ai ce que je vaux \\

I can give you the force of my ancestral pride \\
the will to go on when i'm hurt deep inside \\
whatever the feeling, whatever the way \\
it helps me to go on from day to day \\
je te donne nos doutes et notre indicible espoir \\
les questions que les routes ont laissées dans l'histoire \\
nos filles sont brunes et l'on parle un peu fort \\
et l'humour et l'amour sont nos trésors \\

Je te donne , donne , donne ce que je suis \\

I can give you my voice, bred with rythm and soul, \\
je te donne mes notes , je te donne ma voix \\
the songs that i love, and the stories i've told \\
ce que j'imagine et ce que je crois \\
i can make you feel good even when i'm down \\
les raisons qui me portent et ce stupide espoir \\
my force is a platform that you can climb on \\
une épaule fragile et forte a la fois \\

\begin{bfseries}
[Refrain] (x5):\\
je te donne, je te donne tout ce que je vaux , ce que je suis, mes dons, \\
mes défauts, mes plus belles chances, mes différences
\end{bfseries}

\newpage
\normalsize

\h*{C’est écrit}

\begin{multicols}{2}
Elle te fera changer la course des nuages, \\
Balayer tes projets, vieillir bien avant l'âge, \\
Tu la perdras cent fois dans les vapeurs des ports, \\
C'est écrit... \\
Elle rentrera blessée dans les parfums d'un autre, \\
Tu t'entendras hurler «~que les diables l'emportent~» \\
Elle voudra que tu pardonnes, et tu pardonneras, \\
C'est écrit... \\
Elle n'en sort plus de ta mémoire \\
Ni la nuit, ni le jour, \\
Elle danse derrière les brouillards \\
Et toi, tu cherches et tu cours. \\
Tu prieras jusqu'aux heures ou personne n'écoute, \\
Tu videras tous les bars qu'elle mettra sur ta route, \\
T'en passeras des nuits à regarderdehors. \\
C'est écrit... \\
Elle n'en sort plus de ta mémoire \\
Ni la nuit, ni le jour, \\
Elle danse derrière les brouillards \\
Et toi, tu cherches et tu cours, \\
Mais y a pas d'amours sans histoires. \\
Et tu rêves, tu rêves... \\
Qu'est-ce qu'elle aime, qu'est-ce qu'elle veut ? \\
Et ses ombres qu'elle te dessine autour des yeux ? \\
Qu'est-ce qu'elle aime ? \\
Qu'est-ce qu'elle rêve, qui elle voit ? \\
Et ces cordes qu'elle t'enroule autour des bras ? \\
Qu'est-ce qu'elle aime ? \\
Je t'écouterai me dire ses soupirs, ses dentelles, \\
Qu'à bien y réfléchir, elle n'est plus vraiment belle, \\
Que t'es déjà passé par des moments plus forts, \\
Depuis... \\
Elle n'en sort plus de ta mémoire \\
Ni la nuit, ni le jour, \\
Elle danse derrière les brouillards \\
Et toi, tu cherches et tu cours, \\
Mais y a pas d'amours sans histoires. \\
Oh tu rêves, tu rêves... \\
Elle n'en sort plus de ta mémoire \\
Elle danse derrière les brouillards \\
Et moi j'ai vécu la même histoire \\
Depuis je compte les jours \\
Depuis je compte les nuits
\end{multicols}

\newpage
\small

\h*{Place des grands hommes}

\begin{multicols}{2}
\begin{bfseries}
[Refrain:] \\
On s'était dit rendez-vous dans 10 ans \\
Même jour, même heure, mêmes port \\
On verra quand on aura 30 ans \\
Sur les marches de la place des grands hommes \\
\end{bfseries}

Le jour est venu et moi aussi \\
Mais j' veux pas être le premier. \\
Si on avait plus rien à se dire et si et si... \\

Je fais des détours dans le quartier. \\
C'est fou c'qu'un crépuscule de printemps. \\
Rappelle le même crépuscule qu'il y a 10 ans, \\
Trottoirs usés par les regards baissés. \\
Qu'est-ce que j'ai fait de ces années ? \\

J'ai pas flotté tranquille sur l'eau, \\
Je n'ai pas nagé le vent dans le dos. \\
Dernière ligne droite, la rue Soufflot, \\
Combien seront là 4, 3, 2, 1... 0 ? \\

\textbf{[Refrain]} \\

J'avais eu si souvent envie d'elle. \\
La belle Séverine me regardera-t-elle ? \\
Eric voulait explorer le subconscient. \\
Remonte-t-il à la surface de temps en temps ? \\
J'ai un peu peur de traverser l' miroir. \\
Si j'y allais pas... J' me serais trompé d'un soir. \\
Devant une vitrine d'antiquités, \\
J'imagine les retrouvailles de l'amitié. \\
«~T'as pas changé, qu'est-ce que tudeviens ? \\
Tu t'es mariée, t'as trois gamins. \\
T'as réussi, tu fais médecin ? \\
Et toi Pascale, tu t' marres toujourspour rien ?~» \\

\textbf{[Refrain]} \\

J'ai connu des marées hautes et des marées basses, \\
Comme vous, comme vous, comme vous. \\
J'ai rencontré des tempêtes et des bourrasques, \\
Comme vous, comme vous, comme vous. \\
Chaque amour morte à une nouvelle a fait place, \\
Et vous, et vous...et vous ? \\
Et toi Marco qui ambitionnait simplement d'être heureux dans la vie, \\
As-tu réussi ton pari ? \\
Et toi François, et toi Laurence, et toi Marion, \\
Et toi Gégé...et toi Bruno, et toi Evelyne ? \\

\textbf{[Refrain]} \\

Et bien c'est formidable les copains! \\
On s'est tout dit, on s' serre la main ! \\
On ne peut pas mettre 10 ans sur table \\
Comme on étale ses lettres au Scrabble. \\
Dans la vitrine je vois le reflet \\

D'une lycéenne derrière moi. \\
Si elle part à gauche, je la suivrai. \\
Si c'est à droite... Attendez-moi ! \\
Attendez-moi ! Attendez-moi ! \\
Attendez-moi ! \\

On s'était dit rendez-vous dans 10 ans, \\
Même jour, même heure, mêmes pommes. \\
On verra quand on aura 30 ans \\
Si on est d'venus des grands hommes... \\
Des grands hommes... des grands hommes... \\
Tiens si on s' donnait rendez-vous dans 10 ans... \\

\end{multicols}

\newpage
\small

\h*{Mistral gagnant}
A m'asseoir sur un banc cinq minutes avec toi \\
Et regarder les gens tant qu'y en a \\
Te parler du bon temps qui est mort ou qui r'viendra \\
En serrant dans ma main tes p'tits doigts \\
Pis donner à bouffer à des pigeons idiots \\
Leur filer des coups d' pieds pour de faux \\
Et entendre ton rire qui lézarde les murs \\
Qui sait surtout guérir mes blessures \\

Te raconter un peu comment j'étais mino \\
Les bonbecs fabuleux, qu'on piquait chez l' marchand \\
Car-en-sac et Minto, caramel à un franc \\
Et les mistrals gagnants \\

Par marcher sous la pluie cinq minutes avec toi \\
Et regarder la vie tant qu'y en a \\
Te raconter la Terre en te bouffant des yeux \\
Te parler de ta mère un p'tit peu \\
Et sauter dans les flaques pour la faire râler \\
Bousiller nos godasses et s' marrer \\
Et entendre ton rire comme on entend la mer \\
S'arrêter, r'partir en arrière \\
Te raconter surtout les carambars d'antan et les cocos bohères \\
Et les vrais roudoudous qui nous coupaient les lèvres \\
Et nous niquaient les dents \\
Et les mistrals gagnants \\

A m'asseoir sur un banc cinq minutes avec toi \\
Et r'garder le soleil qui s'en va \\
Te parler du bon temps qui est mort et je m'en fous \\
Te dire que les méchants c'est pas nous \\
Que si moi je suis barge, ce n'est que de tes yeux \\
Car ils ont l'avantage d'être deux \\
Et entendre ton rire s'envoler aussi haut \\
Que s'envolent les cris des oiseaux \\

Te raconter enfin qu'il faut aimer la vie \\
Et l'aimer même si \\
le temps est assassin \\
Et emporte avec lui les rires des enfants \\
Et les mistrals gagnants \\
Et les mistrals gagnants

\newpage
\normalsize

\vbox{
\begin{minipage}[t][0.5\textheight][t]{\textwidth}
% \vspace{0.02\textheight}
\h*{La bille de verre}
\small
\begin{multicols}{2}
Un bateau de bois \\
Emporte papa \\
Tout au bout d'la terre. \\
Il verra la Chine \\
Et les îles opalines \\
Où les gens vivent nus. \\
Moi, j'deviendrai un homme, \\
Mes notes seront bonnes, \\
Il sera fier de moi. \\
Il me rapportera une bille de verre \\
Et un ver à soie (bis). \\

Si la nuit m'fait peur \\
J'lui dirai que mon cœur \\
Est au bout d'la terre \\
Où les enfants des rois \\
Ont des sabres qui coupent \\
Et des chevaux vivants. \\
Moi, je ferai l'grand, \\
Je défendrai maman \\
Contre les voleurs. \\
Il me rapportera une bille de verre \\
Et un ver à soie (bis). \\

\columnbreak

Plus tard il y aura les caresses des femmes, \\
Les secrets qui planent \\
Aux oreilles des grands, \\
Les départs à minuit, les tempêtes, les \\
drames, \\
L'océan... \\
Et quelqu'un qui attend une bille de verre \\
Et un ver à soie (bis). \\

Un bateau de bois \\
Emporte papa \\
Tout au bout d'la terre. \\
Il chass'ra le fauve \\
Au fond des jungles mauves \\
Où le jour n'entre pas. \\
Je cach'rai ma peine. \\
J'attendrai qu'il revienne. \\
Il sera fier de moi. \\
Il me rapportera une bille de verre \\
Et un ver à soie (bis). \\

Il me rapportera une bille de verre... (x2) \\
Il me rapportera une bille de verre \\
Et un ver à soie. \\

\end{multicols}
\end{minipage}
\vspace{0.04\textheight}

\begin{minipage}[b][0.52\textheight][t]{\textwidth}
\h*{Il changeait la vie}
\begin{multicols}{2}
\small
C'était un cordonnier, sans rien d'particulier \\
Dans un village dont le nom m'a échappé \\
Il faisait des souliers si jolis, si légers \\
Que nos vies semblaient un peu moins lourdes à porter \footnotesize \\

\small
Il y mettait du temps, du talent et du cœur \\
Ainsi passait sa vie au milieu de nos heures \\
Et loin des beaux discours, des grandes théories \\
A sa tâche chaque jour, on pouvait dire de lui \\
Il changeait la vie \footnotesize \\

\small
C'était un professeur, un simple professeur \\
Qui pensait que savoir était un grand trésor \\
Que tous les moins que rien n'avaient pour s'en sortir \\
Que l'école et le droit qu'a chacun de s'instruire \footnotesize \\

\small
Il y mettait du temps, du talent et du cœur \\
Ainsi passait sa vie au milieu de nos heures \\
Et loin des beaux discours, des grandes théories \\
A sa tâche chaque jour, on pouvait dire de lui \\
Il changeait la vie \footnotesize \\

\small
C'était un p'tit bonhomme, rien qu'un tout p'tit bonhomme \\
Malhabile et rêveur, un peu loupé en somme \\
Se croyait inutile, banni des autres hommes \footnotesize \\

\small
Il pleurait sur son saxophone \footnotesize \\

\small
Il y mit tant de temps, de larmes et de douleur \\
Les rêves de sa vie, les prisons de son cœur \\
Et loin des beaux discours, des grandes théories \\
Inspiré jour après jour de son souffle et de ses cris \\
Il changeait la vie (x5)
\end{multicols}
\end{minipage}
}


\newpage
\normalsize

\vbox{
\begin{minipage}[t][0.45\textheight][t]{\textwidth}
\vspace{0.02\textheight}
\h*{Cœur de loup}
\small
\begin{multicols}{2}
Cœur de Loup
Pas le temps de tout lui dire \\
Pas le temps de tout lui taire \\
Juste assez pour tenter la satyre \\
Qu'elle sente que j'veux lui plaire \\
Sous le pli de l'emballage \\
La lubie de faufiler \\
La folie de rester sage si elle veut \\
De n'pas l'embrasser \\
Quand d'un coup d'aile se déplume \\
Mon œillet luit fait de l'œil \\
Même hululer sous la lune ne m'fait pas peur \\
Pourvu qu'elle veuille \\

Je n'ai qu'une seule envie \\
Me laisser tenter \\
La victime est si belle \\
Et le crime est si gai \\
Cœur de loup, peur du lit, séduis-là, sans délais \\
Suis le swing, c’est le coup de gong du kingbong \\

Pas besoin de beaucoup \\
Mais pas de peu non plus \\
Par le biais d'un billet fou \\
Lui faire savoir que j'n'en peux plus \\
C'est le cas du kamikaze \\
C'est l'ABC du condamné \\
Le légionnaire qui veut l'avantage des voyages \\
Sans s'engager \\
Elle est si frêle esquive \\
Sous mes bordées d'amour \\
Je suppose qu'elle suppose \\
Que je l'aimerai toujours \\
Le doigts sur l'aventure \\
Le pied dans l'inventaire \\
Même si l'affaire n'est pas sûre \\
Ne pas s'enfuir \\
Ne pas s'en faire \\

Je n'ai qu'une seule envie \\
Me laisser tenter \\
La victime est si belle \\
Et le crime est si gai \\

Pas le temps de mentir \\
Ni de quitter la scène \\
YEP ! Elle aura beau rougir \\
De toute façon il faut qu'elle m'aime \\
Je n'ai qu'un seule envie \\
Me laisser tenter… 
\end{multicols}
\end{minipage}
\vspace{0.02\textheight}

\begin{minipage}[b][0.7\textheight][t]{\textwidth}
\h*{L’encre de tes yeux}
\begin{multicols}{2}
\small
Puisqu'on ne vivra jamais tous les deux \\
Puisqu'on est fou, puisqu'on est seuls \\
Puisqu'ils sont si nombreux \\
Même la morale parle pour eux \\

J'aimerais quand même te dire \\
Tout ce que j'ai pu écrire \\
Je l'ai puisé à l'encre de tes yeux. \\

Je n'avais pas vu que tu portais des chaînes \\
À trop vouloir te regarder, \\
J'en oubliais les miennes \\
On rêvait de Venise et de liberté \\
J'aimerais quand même te dire \\
Tout ce que j'ai pu écrire \\
C'est ton sourire qui me l'a dicté. \\

Tu viendras longtemps marcher dans mes rêves \\
Tu viendras toujours du côté \\
Où le soleil se lève \\
Et si malgré ça j'arrive à t'oublier \\
J'aimerais quand même te dire \\
Tout ce que j'ai pu écrire \\
Aura longtemps le parfum des regrets. \\
\end{multicols}
\end{minipage}
}

\newpage
\normalsize

\h*{Étoile des neiges}
\begin{multicols}{2}
Dans un coin perdu de montagne \\
Un tout petit savoyard \\
Chantait son amour \\
Dans le calme du soir \\
Près de sa bergère \\
Au doux regard. \\

\begin{bfseries}
[Refrain 1:] \\
Étoile des neiges \\
Mon coeur amoureux \\
S'est pris au piège \\
De tes grands yeux \\
Je te donne en gage \\
Cette croix d'argent \\
Et de t'aimer toute ma vie \\
Je fais serment. \\
\end{bfseries}

Hélas soupirait la bergère \\
Que répondront nos parents \\
Comment ferons-nous \\
Nous n'avons pas d'argent \\
Pour nous marier \\
Dès le printemps ? \\

\begin{bfseries}
[Refrain 2:] \\
Étoile des neiges \\
Sèche tes beaux yeux \\
Le ciel protège \\
Les amoureux \\
Je pars en voyage \\
Pour qu'à mon retour \\
À tout jamais plus rien \\
N'empêche notre amour. \\
\end{bfseries}

Alors il partit vers la ville \\
Et ramoneur il se fit \\
Sur tous les chemins \\
Dans le vent et la pluie \\
Comme un petit diable \\
Noir de suie. \\


\begin{bfseries}
[Refrain 3:] \\
Étoile des neiges \\
Sèche tes beaux yeux \\
Le ciel protège \\
Ton amoureux \\
Ne perds pas courage \\
Il te reviendra \\
Et tu seras bientôt \\
Encore entre ses bras. \\
\end{bfseries}

Et quand les beaux jours refleurirent \\
Il s'en revint au hameau \\
Et sa fiancée \\
L'attendait tout là-haut \\
Parmi les clochettes \\
Des troupeaux. \\

\begin{bfseries}
[Refrain 4:] \\
Étoile des neiges \\
Tes garçons d'honneur \\
Vont en cortège \\
Portant des fleurs \\
Par un mariage \\
Finit mon histoire \\
De la bergère et de son petit savoyard \\
\end{bfseries}
\end{multicols}

\newpage
\normalsize

\vbox{
\begin{minipage}[t][0.44\textheight][t]{\textwidth}
\h*{Mon mec à moi}
\small
\begin{multicols}{2}
Il joue avec mon cœur, \\
il triche avec ma vie, \\
il dit des mots menteurs, \\
et moi, je crois tout c' qu'il dit \\
Les chansons qu'il me chante, \\
les rêves qu'il fait pour deux, \\
c'est comme les bonbons menthe, \\
ça fait du bien quand il pleut. \\
Je m' raconte des histoires, \\
en écoutant sa voix, \\
c'est pas vrai ces histoires, \\
mais moi j'y crois. \\

Mon mec à moi \\
il me parle d'aventures, \\
et quand elles brillent dans ses yeux, \\
j' pourrais y passer la nuit \\
Il parle d'amour \\
comme il parle des voitures \\
et moi j'l'suis où il veut, \\
tellement je crois tout c'qu'il m'dit \\
tellement je crois tout c'qu'il m'dit \\
Oh oui \\
Mon mec à moi. \\

Sa façon d'être à moi \\
sans jamais dire “ je t'aime “, \\
c'est rien qu'du cinéma, \\
mais c'est du pareil au même. \\
Ce film en noir et blanc \\
qu'il m'a joué deux cents fois, \\
c'est Gabin et Morgan \\
enfin, ça ressemble à tout ça \\
Je m'raconte des histoires, \\
des scénarios chinois, \\
c'est pas vrai ces histoires, \\
mais moi j'y crois \\

Mon mec à moi \\
il me parle d'aventures \\
et quand elles brillent dans ses yeux, \\
j'pourrais y passer la nuit \\
Il parle d'amour \\
comme il parle des voitures, \\
et moi j'l'suis où il veut, \\
tellement je crois tout c'qu'il m'dit \\
tellement je crois tout c'qu'il m'dit \\
Oh oui \\
\end{multicols}
\end{minipage}
\vspace{0.07\textheight}

\begin{minipage}[b][0.6\textheight][t]{\textwidth}
\h*{Elle est d’ailleurs}
\begin{multicols}{2}
\small
Elle a de ces lumières au fond des yeux \\
Qui rendent aveugles ou amoureux \\
Elle a des gestes de parfum \\
Qui rendent bête ou rendent chien \\
Et si lointaine dans son cœur \\
Pour moi c'est sûr, elle est d'ailleurs \\

Elle a de ces longues mains de dentellière \\
A damner l'âme d'un Werner \\
Cette silhouette vénitienne \\
Quand elle se penche à ses persiennes \\
Ce geste je le sais par cœur \\
Pour moi c'est sûr, elle est d'ailleurs \\

Et moi je suis tombé en esclavage \\
De ce sourire, de ce visage \\
Et je lui dis emmène moi \\
Et moi je suis prêt à tous les sillages \\
Vers d'autres lieux, d'autres rivages \\
Mais elle passe et ne répond pas \\

Et moi je suis tombé en esclavage \\
De ce sourire, de ce visage \\
Et je lui dis emmène moi \\
Et moi je suis prêt à tous les sillages \\
Vers d'autres lieux, d'autres rivages \\
Mais elle passe et ne répond pas \\
Les mots pour elle sont sans valeur \\
Pour moi c’est sur, elle est d’ailleurs \\

Elle a de ces manières de ne rien dire \\
Qui parlent au bout des souvenirs \\
Cette manière de traverser \\
Quand elle s'en va chez le boucher \\
Quand elle arrive à ma hauteur \\
Pour moi c'est sûr, elle est d'ailleurs \\

\end{multicols}
\end{minipage}
}
\newpage
\normalsize

\vbox{
\begin{minipage}[t][0.4\textheight][t]{\textwidth}
\h*{Belle Ile-en-Mer}
\normalsize
\begin{multicols}{2}
\begin{bfseries}
[Refrain:] \\
Belle-Ile-en-Mer, Marie-Galante, \\
Saint-Vincent, Loin Singapour \\
Seymour Ceylan, Vous c'est l'eau, \\
c'est l'eau qui vous sépare \\
Et vous laisse à part \\
\end{bfseries}

Moi des souvenirs d'enfance en France, violence \\
Manque d'indulgence par les différences que j'ai \\
Café Léger au lait mélangé \\
Séparé petit enfant tout comme vous, je connais ce sentiment \\
De solitude et d'isolement \\

\textbf{[Refrain]} \\

Comme laissé tout seul en mer, \\
Corsaire sur terre \\
Un peu solitaire l'amour je l'voyais passer, \\
Ohé Ohé, je l'voyais passer \\
Séparé petit enfant tout comme vous je connais ce sentiment \\
De solitude et d'isolement \\

\textbf{[Refrain]} \\

Karukera, Calédonie, Ouessant,
Vierges des mers, toutes seules tout l'temps
Vous c'est l'eau, c'est l'eau qui vous sépare
Et vous laisse à part
Oh oh... \\

\end{multicols}
\end{minipage}
\vspace{0.09\textheight}

\begin{minipage}[b][0.55\textheight][t]{\textwidth}
\h*{Il y a}
\small
Il y a du thym, de la bruyère et des bois de pin, rien de bien malin \\
Il y a des ruisseaux, des clairières, pas de quoi en faire un plat de ce coin \\
Il y a des odeurs de menthe et des cheminées et des feux dedans \\
Il y a des jours et des nuits lentes et l'histoire absente banalement \\

Et loin de tout, loin de moi \\
C'est là que tu te sens chez toi \\
De là que tu pars, où tu reviens chaque fois \\
Et où tout finira \\

Il y a des enfants, des grand-mères \\
Une petite église et un grand café \\
Il y a au fond du cimetière des joies, des misères et du temps passé \\
Il y a une petite école et des bancs de bois, tout comme autrefois \\
Il y a des images qui collent au bout de tes doigts \\
Et ton cœur qui bat \\

Et loin de tout, loin de moi \\
C'est là que tu te sens chez toi \\
De là que tu pars, où tu reviens chaque fois \\
Et où tout finira \\
\end{minipage}
}

\newpage
\normalsize

\h*{Marchand de cailloux}
\begin{multicols}{2}
Dis Papa, quand c'est qu'y passe \\
Le marchand d'cailloux \\
J'en voudrais dans mes godasses \\
A la place des joujoux \\

Avec mes copines en classe \\
On comprend pas tout \\
Pourquoi des gros dégueulasses \\
Font du mal partout \\
Pourquoi les enfants de Belfast \\
Et d'tous les ghettos \\
Quand y balancent un caillasse \\
On leur fait la peau \\
J'croyais qu'David et Goliath \\
Ça marchait encore \\
Les plus p'tits pouvaient s'débattrent \\
Sans être les plus morts \\

Dis Papa, quand c'est qu'y passe \\
Le marchand d'liberté \\
Il en a oublié un max \\
En f'sant sa tournée \\
Pourquoi des mômes crèvent de faim \\
Pendant qu'on étouffe \\
D'vant nos télés, comme des crétins \\
Sous des tonnes de bouffe \\

Dis Papa, quand c'est qu'y passe \\
Le marchand d'tendresse \\
S'il est sur l'trottoir d'en face \\
Dis-y qu'y traverse \\
J'peux lui en r'filer un peu \\
Pour ceux qu'en ont b'soin \\
J'en ai r'çu tellement mon vieux \\
Qu'j'peux en donner tout plein \\
J'veux partager mon Mac Do \\
Avec ceux qui ont faim \\
J'veux donner d'amour bien chaud \\
A ceux qu'on plus rien \\
Est-ce que c'est ça être coco \\
Ou être un vrai chrétien \\
Moi j'me fous de tous ces mots \\
J'veux être un vrai humain \\

Dis Papa, tous ces discours \\
Me font mal aux oreilles \\
Même ceux qui sont plein d'amour \\
C'est kif-kif-pareil \\
Ça m'fais comme des trous dans la tête \\
Ça m'pollue la vie et tout \\
Ça fait qu'je vois sur ma planète \\
Des «~Inti Fada~» partout \\

\begin{bfseries}
[Répétition] (x2)\\
Et p't'être que sur ta guitare \\
J'en jetterai aussi \\
Si tu t'sers de moi, trouillard \\
Pour chanter tes conneries 
\end{bfseries}

\end{multicols}

\newpage
\normalsize
\vbox{
\begin{minipage}[t][0.4\textheight][t]{\textwidth}
\h*{Morgane de toi}
\footnotesize
\begin{multicols}{2}
Y a un mariolle, il a au moins quatre ans \\
Y veut t' piquer ta pelle et ton seau \\
Ta couche culotte avec tes bonbecs dedans \\
Lolita, défend-toi, fous-y un coup d' râteau dans l' dos \\
Attend un peu avant de t'faire emmerder \\
Par ces p'tits machos qui pensent qu'à une chose \\
Jouer au docteur non conventionné \\
J'y ai joué aussi, je sais de quoi j' cause \\
J' les connais bien les play-boys des bacs à sable \\
J' draguais leurs mères avant d' connaître la tienne \\
Si tu les écoutes y t' feront porter leurs cartables \\
'Reusement qu' j' suis là, que j' te regarde et que j' t'aime \\

\begin{bfseries}
[Refrain:] \\
Lola \\
J' suis qu'un fantôme quand \\
tu vas où j' suis pas \\
Tu sais ma môme \\
Que j' suis morgane de toi \\
\end{bfseries}

Comme j'en ai marre de m' faire tatouer des machins \\
Qui m' font comme une bande dessinée sur la peau \\
J'ai écrit ton nom avec des clous dorés \\
Un par un, plantés dans le cuir de mon blouson dans l' dos \\
T'es la seule gonzesse que j'peux tenir dans mes bras \\
Sans m' démettre une épaule, sans plier sous ton poids \\
Tu pèses moins lourd qu'un moineau qui mange pas \\
Déploie jamais tes ailes, Lolita t'envole pas \\
Avec tes miches de rat qu'on dirait des noisettes \\
Et ta peau plus sucrée qu'un pain au chocolat \\
Tu risques de donner faim a un tas de p'tits mecs \\
Quand t'iras à l'école, si jamais t'y vas \\

\textbf{[Refrain]} \\

Qu'est-ce qu' tu m' racontes tu veux un p'tit frangin \\
Tu veux qu' j' t'achète un ami Pierrot \\
Eh les bébés ça s' trouve pas dans les magasins \\
Puis j' crois pas que ta mère voudra qu' j' lui fasse un p'tit dans l' dos \\
Ben quoi Lola on est pas bien ensemble \\
Tu crois pas qu'on est déjà bien assez nombreux \\
T'entends pas c' bruit, c'est le monde qui tremble \\
Sous les cris des enfants qui sont malheureux \\
Allez viens avec moi, j't'embarque dans ma galère \\
Dans mon arche y a d' la place pour tous les marmots \\
Avant qu' ce monde devienne un grand cimetière \\
Faut profiter un peu du vent qu'on a dans l' dos \\

\textbf{[Refrain] (x2)}

\end{multicols}
\end{minipage}

\vspace{0.09\textheight}
\begin{minipage}[t][0.55\textheight][t]{\textwidth}
\h*{Elisa}
\begin{multicols}{2}
\footnotesize
Elisa, Elisa \\
Elisa saute-moi au cou \\
Elisa, Elisa \\
Elisa cherche-moi des poux, \\
Enfonce bien tes ongles, \\
Et tes doigts délicats \\
Dans la jungle \\
De mes cheveux Lisa \\

Elisa, Elisa \\
Elisa saute-moi au cou \\
Elisa, Elisa \\
Elisa cherche-moi des poux, \\
Fais-moi quelques anglaises, \\
Et la raie au milieu \\
On a treize \\
Quatorze ans à nous deux \\

Elisa, Elisa \\
Elisa les autr's on s'en fout, \\
Elisa, Elisa \\
Elisa rien que toi, moi, nous \\
Tes vingt ans, mes quarante \\
Si tu crois que cela \\
Me tourmente \\
Ah non vraiment Lisa \\

Elisa, Elisa \\
Elisa saute-moi au cou \\
Elisa, Elisa \\
Elisa cherche-moi des poux, \\
Enfonce bien tes ongles, \\
Et tes doigts délicats \\
Dans la jungle \\
De mes cheveux Lisa \\
\end{multicols}
\end{minipage}
}

\newpage
\large

\h*{Elle écoute pousser les fleurs}

\begin{multicols}{2}
Elle écoute pousser les fleurs \\
Au milieu du bruit des moteurs \\
Avec de l'eau de pluie \\
Et du parfum d'encens \\
Elle voyage de temps en temps \\
Elle n'a jamais rien entendu \\
Des chiens qui aboient dans la rue \\
Elle fait du pain doré \\
Tous les jours à quatre heures \\
Elle mène sa vie en couleur \\

Elle collectionne \\
Les odeurs de l'automne \\
Et les brindilles de bois mort \\
Quand l'hiver arrive \\
Elle ferme ses livres \\

Et puis doucement \\
Elle s'endort sur des tapis de laine \\
Au milieu des poupées indiennes \\
Sur les ailes en duvet \\
De ses deux pigeons blancs \\
Jusqu'aux premiers jours du printemps \\

Elle dit qu'elle va faire \\
Le tour de la terre \\
Qu'elle sera rentrée pour dîner \\
Les instants fragiles \\
Les mots inutiles \\

Elle sait tout cela \\
Quand elle écoute pousser les fleurs \\
Au milieu du bruit des moteurs \\
Quand les autres s'emportent \\
Quand j'arrive à m'enfuir \\
C'est chez elle que je vais dormir \\

Et c'est vrai que j'ai peur de lui faire un enfant... \\
\end{multicols}

\newpage
\normalsize

\h*{Quelque chose de Tennessee}
On a tous en nous quelque chose de Tennessee \\
Cette volonté de prolonger la nuit \\
Ce désir fou de vivre une autre vie \\
Ce rêve en nous avec ses mots à lui \\
Quelque chose en nous de Tennessee \\

Quelque chose de Tennessee \\
Cette force qui nous pousse vers l'infini \\
Y a peu d'amour avec tell'ment d'envie \\
Si peu d'amour avec tell'ment de bruit \\
Quelque chose de Tennessee \\

Ainsi vivait Tennessee \\
Le cœur en fièvre et le corps démoli \\
Avec cette formidable envie de vie \\
Ce rêve en nous c'était son cri à lui \\
Quelque chose de Tennessee \\

Comme une étoile qui s'éteint dans la nuit \\
A l'heure où d'autres s'aiment à la folie \\
Sans un éclat de voix et sans un bruit \\
Sans un seul amour, sans un seul ami \\
Ainsi disparut Tennessee \\

A certaines heures de la nuit \\
Quand le cœur de la ville s'est endormi \\
Il flotte un sentiment comme une envie \\
Ce rêve en nous, avec ses mots à lui \\
Quelque chose de Tennessee \\
Oh oui Tennessee \\
Y a quelque chose en nous de Tennessee... \\

\newpage
\large

\h*{Une chanson douce}
\begin{multicols}{2}
Une chanson douce \\
Que me chantait ma maman, \\
En suçant mon pouce \\
J'écoutais en m'endormant. \\
Cette chanson douce, \\
Je veux la chanter pour toi \\
Car ta peau est douce \\
Comme la mousse des bois. \\

La petite biche est aux abois. \\
Dans le bois, se cache le loup, \\
Ouh, ouh, ouh ouh ! \\
Mais le brave chevalier passa. \\
Il prit la biche dans ses bras. \\
La, la, la, la. \\

La petite biche, \\
Ce sera toi, si tu veux. \\
Le loup, on s'en fiche. \\
Contre lui, nous serons deux. \\
Une chanson douce \\
Pour tous les petits enfants \\
Une chanson douce \\
Que me chantait ma maman. \\

O le joli conte que voilà, \\
La biche, en femme, se changea, \\
La, la, la, la \\
Et dans les bras du beau chevalier, \\
Belle princesse elle est restée, \\
A tout jamais \\

La belle princesse \\
Avait tes jolis cheveux, \\
La même caresse \\
Se lit au fond de tes yeux. \\
Cette chanson douce \\
Que me chantait ma maman \\
En suçant mon pouce \\
Je l’écoutais en m’endormant. \\
\end{multicols}

\newpage
\normalsize

\h*{Les murs de poussière}
\begin{multicols}{2}
Il rêvait d'une ville étrangère \\
Une ville de filles et de jeux \\
Il voulait vivre d'autres manières \\
Dans un autre milieu \\
Il rêvait sur son chemin de pierres \\
"Je partirai demain, si je veux \\
J'ai la force qu'il faut pour le faire \\
Et j'irai trouver mieux" \\

Il voulait trouver mieux \\
Que son lopin de terre \\
Que son vieil arbre tordu au milieu \\
Trouver mieux que la douce lumière \\
du soir \\
Près du feu \\
Qui réchauffait son père \\
Et la troupe entière de ses aïeux \\
Le soleil sur les murs de poussière \\
Il voulait trouver mieux... \\

Il a fait tout le tour de la terre \\
Il a même demandé à Dieu \\
Il a fait tout l'amour de la terre \\
Il n'a pas trouvé mieux \\
Il a croisé les rois de naguère \\
Tout drapés de diamants et de feu \\
Mais dans les châteaux des rois de \\
naguère \\
Il n'a pas trouvé mieux... \\

Il n’a pas trouvé mieux \\
Que son lopin de terre \\
Que son vieil arbre tordu au milieu \\
Trouver mieux que la douce lumière \\
du soir \\
Près du feu \\
Qui réchauffait son père \\
Et la troupe entière de ses aïeux \\
Le soleil sur les murs de poussière \\
Il n'a pas trouvé mieux... \\

Il a dit "Je retourne en arrière \\
Je n'ai pas trouvé ce que je veux" \\
Il a dit "Je retourne en arrière" \\
Il s'est brûlé les yeux \\
Il s’est brûlé les yeux \\
Sur son lopin de terre \\
Sur son vieil arbre tordu au milieu \\
Aux reflets de la douce lumière \\
Du soir près du feu \\
Qui réchauffait son père \\
Et la troupe entière de ses aïeux \\
Au soleil sur les murs de poussière \\
Il s’est brûlé les yeux. \\
\end{multicols}

\newpage
\large

\h*{Petite Marie}
Petite Marie, je parle de toi parc'qu'avec, ta petite voix, tes petite manies \\
Tu as versé sur ma vie des milliers de roses \\

Petite furie, je me bats pour toi pour que dans dix mille ans de ça \\
On se retrouve à l'abri sous un ciel aussi joli que de millier de roses \\

\begin{bfseries}
[Refrain:]\\
Je viens du ciel et les étoiles entre elles ne parlent que de toi \\
D'un musicien qui fait jouer ses mains sur un morceau de bois \\
De leur amour plus bleu que le ciel autour \\
\end{bfseries}

Petite Marie, je t'attends transi sous une tuile de ton toit \\
Le vent de la nuit froide me renvoie la ballade que j'avais écrite pour toi \\
Petite furie, tu dis que la vie c'est une bague à chaque doigt \\
Au soleil de Floride, moi mes poches sont vides et mes yeux pleurent de froid \\

\textbf{[Refrain]}\\

Dans la pénombre de ta rue, petite Marie, m'entends-tu ? \\
Je n'attends plus que toi pour partir... \\
Dans la pénombre de ta rue, petite Marie, m'entends-tu ? \\
Je n'attends plus que toi pour partir... \\

\newpage
\normalsize

\h*{Ça fait rire les oiseaux}

\begin{bfseries}
[Refrain:]\\
Ça fait rir' les oiseaux, ça fait chanter les abeilles. \\
Ça chasse les nuages et fait briller le soleil. \\
Ça fait rir' les oiseaux et danser les écureuils. \\
Ça rajoute des couleurs aux couleurs de l'arc-en-ciel. \\
Ça fait rir' les oiseaux, \\
Oh, oh, oh, rir' les oiseaux \\
\end{bfseries}

Une chanson d'amour, c'est comme un looping en avion : \\
Ça fait battre le cœur des filles et des garçons. \\
Une chanson d'amour, c'est l'oxygèn' dans la maison. \\
Tes pieds n'touch'nt plus par terre, t'es en lévitation. \\
Si y a d' la pluie dans ta vie, le soir te fait peur. \\
La musique est là pour ça. \\
Y a toujours une mélodie pour des jours meilleurs. \\
Allez, tape dans tes mains, ça porte bonheur. \\
C'est magique, un refrain qu'on reprend tous en chœur. \\

\textbf{[Refrain]}\\

T'es revenu chez toi la tête pleine de souvenirs : \\
Des soirs au clair de lune, des moments de plaisir. \\
T'es revenu chez toi et tu veux déjà repartir \\
Pour trouver l'aventure qui n'arrête pas de finir. \\
Si y a du gris dans ta nuit, des larmes dans ton cœur. \\
La musique est là pour ça. \\
Y a toujours une mélodie pour des jours meilleurs. \\
Allez, tape dans tes mains ça porte bonheur. \\
C'est magique, un refrain qu'on reprend tous en chœur. \\

\textbf{[Refrain] (x2)}

\newpage
\large

\h*{L’Amérique}

Mes amis, je dois m'en aller \\
Je n'ai plus qu'à jeter mes clés \\
Car elle m'attend depuis que je suis né \\
L'Amérique, l’Amérique \\

J'abandonne sur mon chemin \\
Tant de choses que j'aimais bien \\
Cela commence par un peu de chagrin \\
L'Amérique, l’Amérique \\

\begin{bfseries}
[Refrain:]\\
L'Amérique, l'Amérique, je veux l'avoir et je l'aurai \\
L'Amérique, l'Amérique, si c'est un rêve, je le saurai \\
Tous les sifflets des trains, toutes les sirènes des bateaux \\
M'ont chanté cent fois la chanson de l'Eldorado \\
De l'Amérique \\
\end{bfseries}

Mes amis, je vous dis adieu \\
Je devrais vous pleurer un peu \\
Pardonnez-moi si je n'ai dans les yeux \\
Que l'Amérique, l’Amérique \\

Je reviendrai je ne sais pas quand \\
Cousu d'or et brodé d'argent \\
Ou sans un sou, mais plus riche qu'avant \\
De l'Amérique \\

\textbf{[Refrain]}


\newpage
\normalsize

\h*{Mon vieux}

\begin{multicols}{2}
Dans son vieux pardessus râpé \\
Il s'en allait l'hiver, l'été \\
Dans le petit matin frileux \\
Mon vieux. \\

Y avait qu'un dimanche par semaine \\
Les autres jours, c'était la graine \\
Qu'il allait gagner comme on peut \\
Mon vieux. \\

L'été, on allait voir la mer \\
Tu vois c'était pas la misère \\
C'était pas non plus l'paradis \\
Hé oui tant pis. \\

Dans son vieux pardessus râpé \\
Il a pris pendant des années \\
L'même autobus de banlieue \\
Mon vieux. \\

L'soir en rentrant du boulot \\
Il s'asseyait sans dire un mot \\
Il était du genre silencieux \\
Mon vieux. \\

Les dimanches étaient monotones \\
On n'recevait jamais personne \\
Ça n'le rendait pas malheureux \\
Je crois, mon vieux. \\

Dans son vieux pardessus râpé \\
Les jours de paye quand il rentrait \\
On l'entendait gueuler un peu \\
Mon vieux. \\

Nous, on connaissait la chanson \\
Tout y passait, bourgeois, patrons, \\
La gauche, la droite, même le bon Dieu \\
Avec mon vieux. \\

Chez nous y'avait pas la télé \\
C'est dehors que j'allais chercher \\
Pendant quelques heures l'évasion \\
Tu sais, c'est con! \\

Dire que j'ai passé des années \\
A côté de lui sans le regarder \\
On a à peine ouvert les yeux \\
Nous deux. \\

J'aurais pu c'était pas malin \\
Faire avec lui un bout d'chemin \\
Ça l'aurait pt'être rendu heureux \\
Mon vieux. \\

Mais quand on a juste quinze ans \\
On n'a pas le cœur assez grand \\
Pour y loger toutes ces choses-là \\
Tu vois. \\

Maintenant qu'il est loin d'ici \\
En pensant à tout ça, j'me dis \\
J'aim'rais bien qu'il soit près de moi \\
Papa… \\

\end{multicols}

\newpage
\normalsize
\vbox{
\begin{minipage}[t][0.4\textheight][t]{\textwidth}
\h*{J’te le dis quand-même}
\normalsize
\begin{multicols}{2}
On aurait pu se dire tout ça \\
Ailleurs qu'au café d'en bas, \\
Que t'allais p't êt' partir \\
Et p't êt' même pas rev'nir, \\
Mais en tout cas, c' qui est sûr, \\
C'est qu'on pouvait en rire. \\

Alors on va s' quitter comme ça, \\
Comme des cons d'vant l' café d'en bas. \\
Comme dans une série B, \\
On est tous les deux mauvais. \\
On s'est moqué tellement d' fois \\
Des gens qui faisaient ça. \\

Mais j' trouve pas d' refrain à notre histoire. \\
Tous les mots qui m' viennent sont dérisoires. \\
J' sais bien qu' j' l'ai trop dit, \\
Mais j' te l' dis quand même... je t'aime. \\

J' voulais quand même te dire merci \\
Pour tout le mal qu'on s'est pas dit. \\
Certains rigolent déjà. \\
J' m'en fous, j' les aimais pas. \\
On avait l'air trop bien. \\
Y en a qui n' supportent pas. \\

Mais j' trouve pas d' refrain à notre histoire. \\
Tous les mots qui m' viennent sont dérisoires. \\
J' sais bien qu' j' l' ai trop dit, \\
Mais j' te l' dis quand même... je t'aime. \\
\end{multicols}
\end{minipage}

\vspace{0.12\textheight}
\begin{minipage}[t][0.55\textheight][t]{\textwidth}
\h*{Diego, libre dans sa tête}
\begin{multicols}{2}
\normalsize
Derrière des barreaux \\
Pour quelques mots \\
Qu'il pensait si fort \\
Dehors il fait chaud \\
Des milliers d'oiseaux \\
S'envolent sans effort \\

Quel est ce pays \\
Où frappe la nuit \\
La loi du plus fort ? \\

\columnbreak

Diego, libre dans sa tête \\
Derrière sa fenêtre \\
S'endort peut-etre... \\

Et moi qui danse ma vie \\
Qui chante et qui rit \\
Je pense à lui \\

Diego, libre dans sa tête \\
Derrière sa fenêtre \\
Déjà mort peut-être... \\
\end{multicols}
\end{minipage}
}

\newpage
\normalsize
\vbox{
\begin{minipage}[t][0.4\textheight][t]{\textwidth}
\h*{San Francisco}
\small
\begin{multicols}{2}
C'est une maison bleue adossée à la colline \\
On y vient à pied, on ne frappe pas \\
Ceux qui vivent là, ont jeté la clé \\
On se retrouve ensemble, après des années de route \\
Et l'on vient s'asseoir, autour du repas \\
Tout le monde est là, à cinq heures du soir \\
San Francisco s'embrume, San Francisco s'allume \\
San Francisco… \\
Où êtes vous, Lisette et Luc, Sylvia, attendez-moi… \\

Nageant dans le brouillard, enlacés, roulant dans l'herbe \\
On écoutera Tom à la guitare, Phil à la kena, jusqu'à la nuit noire \\
Un autre arrivera pour nous dire des nouvelles \\
D'un qui reviendra dans un an ou deux \\
Puisqu'il est heureux, on s'endormira \\
San Francisco se lève, San Francisco se lève \\
San Francisco… \\
Où êtes vous, Lisette et Luc, Sylvia, \\
attendez-moi… \\

C'est une maison bleue, accrochée à ma mémoire \\
On y vient à pied, on ne frappe pas \\
Ceux qui vivent là, ont jeté la clef. \\
Peuplée de cheveux longs, de grands lits et de musique \\
Peuplée de lumière, et peuplée de fous \\
Elle sera dernière à rester debout. \\
Si San Francisco s'effondre, si San Francisco s'effondre \\
San Francisco… \\
Où êtes vous Lisette et Luc, Sylvia, attendez-moi… \\

\end{multicols}
\end{minipage}

\vspace{0.05\textheight}
\begin{minipage}[t][0.65\textheight][t]{\textwidth}
\h*{Je l’aime à mourir}
\begin{multicols}{2}
\small
Moi je n'étais rien et voilà qu'aujourd'hui \\
Je suis le gardien du sommeil de ses nuits \\
Je l'aime à mourir \\
Vous pouvez détruire tout ce qu'il vous plaira \\
Elle n'a qu'à ouvrir l'espace de ses bras \\
Pour tout reconstruire, pour tout reconstruire \\
Je l'aime à mourir \\
Elle a gommé les chiffres des horloges du quartier \\
Elle a fait de ma vie des cocottes en papier \\
Des éclats de rire \\
Elle a bâti des ponts entre nous et le ciel \\
Et nous les traversons à chaque fois qu'elle \\
Ne veut pas dormir, ne veut pas dormir \\
Je l'aime à mourir \scriptsize \\

\small
\begin{bfseries}
[Refrain:]\\
Elle a dû faire toutes les guerres pour être si forte aujourd'hui \\
Elle a dû faire toutes les guerres de la vie, et l'amour aussi \\
\end{bfseries}

Elle vit de son mieux son rêve d'opaline \\
Elle danse au milieu des forêts qu'elle dessine \\
Je l'aime à mourir \\
Elle porte des rubans qu'elle laisse s'envoler \\
Elle me chante souvent que j'ai tort d'essayer \\
De les retenir, de les retenir \\
Je l'aime à mourir \\
Pour monter dans sa grotte cachée sous les toits \\
Je dois clouer des notes à mes sabots de bois \\
Je l'aime à mourir \\
Je dois juste m'asseoir, je ne dois pas parler \\
Je ne dois rien vouloir je dois juste essayer \\
De lui appartenir, de lui appartenir \\
Je l'aime à mourir \scriptsize \\

\small
\textbf{[Refrain]}\\

Moi je n'étais rien et voilà qu'aujourd'hui \\
Je suis le gardien Du sommeil de ses nuits \\
Je l'aime à mourir \\
Vous pouvez détruire tout ce qu'il vous plaira \\
Elle n'aura qu'à ouvrir l'espace de ses bras \\
Pour tout reconstruire, pour tout reconstruire \\
Je l'aime à mourir \\

\end{multicols}
\end{minipage}
}

\newpage
\normalsize
\vbox{
\begin{minipage}[t][0.4\textheight][t]{\textwidth}
\h*{Il faudra leur dire}
\small
\begin{multicols}{2}
\begin{bfseries}
[Refrain:]\\
Si c'est vrai qu'il y a des gens qui s'aiment \\
Si les enfants sont tous les mêmes \\
Alors il faudra leur dire \\
C'est comme des parfums qu'on respire \\
Juste un regard \\
Facile à faire \\
Un peu plus d'amour que d'ordinaire \\
\end{bfseries}

Puisqu'on vit dans la même lumière \\
Même s'il y a des couleurs qu'ils préfèrent \\
Nous on voudrait leur dire \\
C'est comme des parfums qu'on respire \\
Juste un regard \\
Facile à faire \\
Un peu plus d'amour que d'ordinaire \\

Juste un peu plus d'amour encore \\
Pour moins de larmes \\
Pour moins de vide \\
Pour moins d'hiver \\
Puisqu'on vit dans les creux d'un rêve \\
Avant que leurs mains ne touchent nos lèvres \\
Nous on voudrait leur dire \\
Les mots qu'on reçoit \\
C'est comme des parfums qu'on respire \\
Il faudra leur dire \\
Facile à faire \\
Un peu plus d'amour que d'ordinaire \\

\textbf{[Refrain] (x2)}\\

\end{multicols}
\end{minipage}

\vspace{0.08\textheight}
\begin{minipage}[t][0.45\textheight][t]{\textwidth}
\h*{Qui a le droit}
\begin{multicols}{2}
\small

On m'avait dit : "Te poses pas trop de questions. \\
Tu sais petit, c'est la vie qui t' répond. \\
A quoi ça sert de vouloir tout savoir ? \\
Regarde en l'air et voit c' que tu peux voir." \\

On m'avait dit : "Faut écouter son père." \\
Le mien a rien dit, quand il s'est fait la paire. \\
Maman m'a dit : "T'es trop p'tit pour comprendre." \\
Et j'ai grandi avec une place à prendre. \\

\begin{bfseries}
[Refrain:]\\
Qui a le droit, qui a le droit, \\
Qui a le droit d' faire ça \\
A un enfant qui croit vraiment \\
C' que disent les grands ? \\

On passe sa vie à dire merci, \\
Merci à qui, à quoi ? \\
A faire la pluie et le beau temps \\
Pour des enfants à qui l'on ment. \\
\end{bfseries}

On m'avait dit que les hommes sont tous pareils. \\
Y a plusieurs dieux, mais y' a qu'un seul soleil. \\
Oui mais, l' soleil il brille ou bien il brûle. \\
Tu meurs de soif ou bien tu bois des bulles. \\

A toi aussi, j' suis sur qu'on t'en a dit, \\
De belles histoires, tu parles... que des conneries ! \\
Alors maintenant, on s' retrouve sur la route, \\
Avec nos peurs, nos angoisses et nos doutes. \\

\textbf{[Refrain]}\\

\end{multicols}
\end{minipage}
}

\newpage
\normalsize

\h*{Déjeuner en paix}
J'abandonne sur une chaise le journal du matin \\
Les nouvelles sont mauvaises d'où qu'elles viennent \\
J'attends qu'elle se réveille et qu'elle se lève enfin \\
Je souffle sur les braises pour qu'elles prennent \\

Cette fois je ne lui annoncerai pas \\
La dernière hécatombe \\
Je garderai pour moi ce que m'inspire le monde \\
Elle m'a dit qu'elle voulait si je le permettais \\
Déjeuner en paix, déjeuner en paix \\

Je vais à la fenêtre et le ciel ce matin \\
N'est ni rose ni honnête pour la peine \\
«~Est-ce que tout va si mal ? Est-ce que rien ne va bien ? \\
L'homme est un animal~», me dit-elle \\

\begin{bfseries}
[Refrain:]\\
Elle prend son café en riant \\
Elle me regarde à peine \\
Plus rien ne la surprend sur la nature humaine \\
C'est pourquoi elle voudrait enfin si je le permets \\
Déjeuner en paix, déjeuner en paix, oh déjeuner en paix \\
\end{bfseries}

Je regarde sur la chaise le journal du matin \\
Les nouvelles sont mauvaises d'où qu'elles viennent \\
«~Crois-tu qu'il va neiger ?~» me demande-t-elle soudain \\
«~Me feras-tu un bébé pour Noël ?~» \\

\textbf{[Refrain]}\\


\newpage
\small

\h*{Les enfants du monde entier}
\begin{multicols}{2}
Pour les enfants du monde entier \\
Qui n'ont plus rien à espérer \\
Je voudrais faire une prière \\
À tous les maîtres de la terre \\

À chaque enfant qui disparaît \\
C'est l'univers qui tire un trait \\
Sur un espoir pour l'avenir \\
De pouvoir nous appartenir \\

J'ai vu des enfants s'en aller \\
Sourire aux lèvres et coeur léger \\
Vers la mort et le paradis \\
Que les adultes avaient promis \\

Mais quand ils sautaient sur les mines \\
C'était Mozart qu'on assassine \\
Si le bonheur est à ce prix \\
De quel enfer s'est-il nourri? \\

Et combien faudra-t-il payer \\
De silence et d'obscurité \\
Pour effacer dans les mémoires \\
Le souvenir de leur histoire? \\

Quel testament, quel évangile, \\
Quelle main aveugle ou imbécile \\
Peut condamner tant d'innocence \\
À tant de larmes et de souffrance? \\

La peur, la haine et la violence \\
Ont mis le feu à leur enfance \\
Leurs chemins se sont hérissés \\
De misère et de barbelés \\

Peut-on convaincre un dictateur \\
D'écouter battre un peu son cœur? \\
Peut-on souhaiter d'un président \\
Qu'il pleure aussi de temps en temps? \\

Pour les enfants du monde entier \\
Qui n'ont de voix que pour pleurer \\
Je voudrais faire une prière \\
À tous les maîtres de la terre \\

Dans vos sommeils de somnifères \\
Où vous dormez les yeux ouverts \\
Laissez souffler pour un instant \\
La magie de vos coeurs d'enfants \\

Puisque l'on sait de par le monde \\
Faire la paix pour quelques secondes \\
Au nom du Père et pour Noël \\
Que la trève soit éternelle \\

Qu'elle taise à jamais les rancoeurs \\
Et qu'elle apaise au fond des coeurs \\
La vengeance et la cruauté \\
Jusqu'au bout de l'éternité \\

Je n'ai pas l'ombre d'un pouvoir \\
Mais j'ai le cœur rempli d'espoir \\
Et de chansons pour aujourd'hui \\
Qui sont des hymnes pour la vie \\

Et des ghettos, des bidonvilles, \\
Du coeur du siècle de l'exil \\
Des voix s'élèvent un peu partout \\
Qui font chanter les gens debout \\

Vous pouvez fermer vos frontières, \\
Bloquer vos ports et vos rivières, \\
Mais les chansons voyagent à pied \\
En secret dans des coeurs fermés \\

Ce sont les mères qui les apprennent \\
À leurs enfants qui les reprennent \\
Elles finiront par éclater \\
Sous le ciel de la liberté \\

Pour les enfants du monde entier...
\end{multicols}

\newpage
\normalsize
\vbox{
\begin{minipage}[t][0.4\textheight][t]{\textwidth}
\h*{L’aigle noir}
\small
\begin{multicols}{2}
Un beau jour ou peut-être une nuit \\
Près d'un lac je m'étais endormie \\
Quand soudain, semblant crever le ciel \\
Et venant de nulle part, \\
Surgit un aigle noir. \footnotesize \\

\small
Lentement, les ailes déployées, \\
Lentement, je le vis tournoyer \\
Près de moi, dans un bruissement d'ailes, \\
Comme tombé du ciel \\
L'oiseau vint se poser. \footnotesize \\

\small
Il avait les yeux couleur rubis \\
Et des plumes couleur de la nuit \\
À son front, brillant de mille feux, \\
L'oiseau roi couronné \\
Portait un diamant bleu. \footnotesize \\

\small
De son bec, il a touché ma joue \\
Dans ma main, il a glissé son cou \\
C'est alors que je l'ai reconnu \\
Surgissant du passé \\
Il m'était revenu.

\columnbreak
Dis l'oiseau, o dis, emmène-moi \\
Retournons au pays d'autrefois \\
Comme avant, dans mes rêves d'enfant, \\
Pour cueillir en tremblant \\
Des étoiles, des étoiles. \footnotesize \\

\small
Comme avant, dans mes rêves d'enfant, \\
Comme avant, sur un nuage blanc, \\
Comme avant, allumer le soleil, \\
Être faiseur de pluie \\
Et faire des merveilles.  \footnotesize \\

\small
L'aigle noir dans un bruissement d'ailes \\
Prit son vol pour regagner le ciel \footnotesize \\

\small
Quatre plumes, couleur de la nuit, \\
Une larme, ou peut-être un rubis \\
J'avais froid, il ne me restait rien \\
L'oiseau m'avait laissée \\
Seule avec mon chagrin  \footnotesize \\

\small
Un beau jour, ou était-ce une nuit \\
Près d'un lac je m'étais endormie \\
Quand soudain, semblant crever le ciel, \\
Et venant de nulle part \\
Surgit un aigle noir.

\end{multicols}
\end{minipage}

\vspace{0.08\textheight}
\begin{minipage}[t][0.65\textheight][t]{\textwidth}
\h*{Chante la vie chante}
\begin{multicols}{2}
\small
Chante la vie chante \\
Comme si tu devais mourir demain \\
Chante comme si plus rien n'avait d'importance \\
Chante, oui chante \\
Aime la vie aime \\
Comm' un voyou comm' un fou comm' un chien \\
Comme si c'était ta dernière chance \\
Chante oui chante \\
Tu peux partir quand tu veux \\
Et tu peux dormir où tu veux \\
Rêver d'une fille \\
Prendre la Bastille \\
Ou claquer ton fric au jeu \\
Mais n'oublie pas. \\

Chante la vie chante \\
Comme si tu devais mourir demain \\
Chante comme si plus rien n'avaitd'importance \\
Chante, oui chante \\
Fête fais la fête \\
Pour un amour un ami ou un rien \\
Pour oublier qu'il pleut sur tes vacances \\
Chante oui chante \\
Et tu verras que c'est bon \\
De laisser tomber sa raison \\
Sors par les fenêtres \\
Marche sur la tête \\
Pour changer les traditions \\
Mais n'oublie pas. \\

Chante la vie chante \\
Comme si tu devais mourir demain \\
Chante comme si plus rien n'avait \\
d'importance, chante, oui chante \\
La la la… 

\end{multicols}
\end{minipage}
}

\newpage
\normalsize
\vbox{
\begin{minipage}[t][0.4\textheight][t]{\textwidth}
\h*{En cloque}
\footnotesize
\begin{multicols}{3}
Elle a mis sur l' mur \\
Au dessus du berceau \\
Une photo d'Arthur \\
Rimbaud \\
Avec ses cheveux en brosse \\
Elle trouve qu'il est beau \\
Dans la chambre du gosse \\
Bravo \\
Déjà les p'tits anges \\
Sur le papier peint \\
J' trouvais ça étrange \\
J' dis rien \\
Elle me font marrer \\
Ses idées loufoques \\
Depuis qu'elle est \\
En cloque \\

Elle s' réveille la nuit \\
Veut bouffer des fraises \\
Elle a des envies \\
Balaises \\
Moi, j' suis aux p'tits soins \\
J' me défonces en huit \\
Pour qu'elle manque de rien \\
Ma p'tite \\
C'est comme si j' pissais \\
Dans un violoncelle \\
Comme si j'existais \\
Plus pour elle \\
Je m' retrouve planté \\
Tout seul dans mon froc \\
Depuis qu'elle est \\
En cloque \\

Le soir elle tricote \\
En buvant d' la verveine \\
Moi j' démêle ses pelotes \\
De laine \\
Elle use les miroirs \\
A s' regarder dedans \\
A s' trouver bizarre \\
Tout le temps \\
J' lui dit qu'elle est belle \\
Comme un fruit trop mûr \\
Elle croit qu' je m' fous d'elle \\
C'est sûr \\
Faut bien dire s' qu'y est \\
Moi aussi j' débloque \\
Depuis qu'elle est \\
En cloque \\

Faut qu' j' retire mes grolles \\
Quand j' rentre dans la chambre \\
Du p'tit rossignol \\
Qu'elle couve \\
C'est qu' son p'tit bonhomme \\
Qu'arrive en Décembre \\
Elle le protège comme \\
Une louve \\
Même le chat pépère \\
Elle en dit du mal \\
Sous prétexte qu'il perd \\
Ses poils \\
Elle veut plus l' voir traîner \\
Autour du paddock \\
Depuis qu'elle est \\
En cloque \\

Quand j' promène mes mains \\
D' l'autre côté d' son dos \\
J' sens comme des coups de poings \\
Ça bouge \\
J' lui dis "t'es un jardin" \\
"Une fleur, un ruisseau" \\
Alors elle devient \\
Toute rouge \\
Parfois c' qu'y m' désole \\
C' qu'y fait du chagrin \\
Quand j' regarde son ventre \\
Puis l' mien \\
C'est qu' même si j' devenais \\
Pédé comme un phoque \\
Moi j' serai jamais \\
En cloque \\

\end{multicols}
\end{minipage}

\vspace{0.08\textheight}
\begin{minipage}[t][0.65\textheight][t]{\textwidth}
\h*{Comme toi}
\begin{multicols}{2}
\footnotesize

Elle avait les yeux clairs et la robe en velours \\
À côté de sa mère et la famille autour \\
Elle pose un peu distraite au doux soleil de la fin du jour \\
La photo n'est pas bonne mais l'on peut y voir \\
Le bonheur en personne et la douceur d'un soir \\
Elle aimait la musique surtout Schumann et puis Mozart \\

\begin{bfseries}
[Refrain:]\\
Comme toi comme toi comme toi comme toi \\
Comme toi comme toi comme toi comme toi \\
Comme toi que je regarde tout bas \\
Comme toi qui dort en rêvant à quoi \\
Comme toi comme toi comme toi comme toi \\
\end{bfseries}

\columnbreak

Elle allait à l'école au village d'en bas \\
Elle apprenait les livres elle apprenait les lois \\
Elle chantait les grenouilles et les princesses qui dorment au bois \\
Elle aimait sa poupée elle aimait ses amis \\
Surtout Ruth et Anna et surtout Jérémie \\
Et ils se marieraient un jour peut-être à Varsovie \\

\textbf{[Refrain]}\\

Elle s'appelait Sarah elle n'avait pas huit ans \\
Sa vie c'était douceur rêves et nuages blancs \\
Mais d'autres gens en avaient décidé autrement \\
Elle avait tes yeux clairs et elle avait ton âge \\
C'était une petite fille sans histoires et très sage \\
Mais elle n'est pas née comme toi ici et maintenant. \\

\end{multicols}
\end{minipage}
}

\newpage
\normalsize
\h*{Manu}
\begin{multicols}{2}
Eh Manu rentre chez toi \\
Y a des larmes plein ta bière \\
Le bistrot va fermer \\
Pi tu gonfles la taulière \\
J'croyais qu'un mec en cuir \\
Ça pouvait pas chialer \\
J'pensais même que souffrir \\
Ça pouvais pas t'arriver \\
J'oubliais qu'tes tatouages \\
Et ta lame de couteau \\
C'est surtout un blindage \\
Pour ton cœur d'artichaut \\

\begin{bfseries}
[Répétition 1:]\\
Eh déconne pas Manu \\
Va pas t'tailler les veines \\
Une gonzesse de perdue \\
C'est dix copains qui r'viennent \\
\end{bfseries}

On était tous maqués \\
Quand toi t'étais tous seul \\
Tu disais j'me fais chier \\
Et j'voudrais sauver ma gueule \\
T'as croisé cette nana \\
Qu'était faite pour personne \\
T'as dit elle pour moi \\
Ou alors y a maldonne \\
T'as été un peu vite \\
Pour t'tatouer son prénom \\
A l'endroit où palpite \\
Ton grand cœur de grand con \\


\begin{bfseries}
[Répétition 2:]\\
Eh déconne pas Manu \\
C't'à moi qu'tu fais d'la peine \\
Une gonzesse de perdue \\
C'est dix copains qui r'viennent \\
\end{bfseries}

J'vais dire on est des loups \\
On est fait pour vivre en bande \\
Mais surtout pas en couple \\
Ou alors pas longtemps \\
Nous autres ça fait un bail \\
Qu'on a largué nos p'tites \\
Toi t'es toujours en rade \\
Avec la tienne et tu flippes \\
Eh Manu vivre libre \\
C'est souvent vivre seul \\
Ça fait p't'être mal au bide \\
Mais c'est bon pour la gueule \\

\begin{bfseries}
[Répétition 3:]\\
Eh déconne pas Manu \\
Ça sert à rien la haine \\
Une gonzesse de perdue \\
C'est dix copains qui r'viennent \\
\end{bfseries}

Elle est plus amoureuse \\
Manu faut qu'tu t'arraches \\
Elle peut pas être heureuse \\
Dans les bras d'un apache \\
Quand tu lui dis je t'aime \\
Si elle te d'mande du feu \\
si elle a la migraine \\
Dès qu'elle est dans ton pieu \\
Dis lui qu't'es désolé \\
Qu't'as dû t'gourrer de trottoir \\
Quand tu l'as rencontrée \\
T'as dû t'tromper d'histoire \\

\textbf{[Répétition 1] [Répétition 3]}

\end{multicols}

\newpage
\large
\h*{Encore un matin}
\begin{multicols}{2}
Encore un matin \\
Un matin pour rien \\
Une argile au creux de mes mains \\
Encore un matin \\
Sans raison ni fin \\
Si rien ne trace son chemin \\

Matin pour donner ou bien matin pour prendre \\
Pour oublier ou pour apprendre \\
Matin pour aimer, maudire ou mépriser \\
Laisser tomber ou résister \\

Encore un matin \\
Qui cherche et qui doute \\
Matin perdu cherche une route \\
Encore un matin \\
Du pire ou du mieux \\
A éteindre ou mettre le feu \\
\columnbreak

\begin{bfseries}
[Refrain:]\\
Un matin, ça ne sert à rien \\
Un matin \\
Sans un coup de main \\
Ce matin \\
C'est le mien, c'est le tien \\
Un matin de rien \\
Pour en faire \\
Un rêve plus loin \\
\end{bfseries}

Encore un matin \\
Ou juge ou coupable \\
Ou bien victime ou bien capable \\
Encore un matin, ami, ennemi \\
Entre la raison et l'envie \\
Matin pour agir ou attendre la chance \\
Ou bousculer les évidences \\
Matin innocence, matin intelligence \\
C'est toi qui décide du sens \\

\textbf{[Refrain]}
\end{multicols}

\newpage
\normalsize
\h*{Le paradis blanc}
\begin{multicols}{2}
Il y a tant de vagues et de fumée \\
Qu'on arrive plus à distinguer \\
Le blanc du noir \\
Et l'énergie du désespoir \\
Le téléphone pourra sonner \\
Il n'y aura plus d'abonné \\
Et plus d'idée \\
Que le silence pour respirer \\
Recommencer là où le monde acommencé \\

Je m'en irai dormir dans le paradis blanc \\
Où les nuits sont si longues qu'on en oublie le temps \\
Tout seul avec le vent \\
Comme dans mes rêves d'enfant \\
Je m'en irai courir dans le paradis blanc \\
Loin des regards de haine \\
Et des combats de sang \\
Retrouver les baleines \\
Parler aux poissons d'argent \\
Comme, comme, comme avant \\
\columnbreak

Y a tant de vagues, et tant d'idées \\
Qu'on arrive plus à décider \\
Le faux du vrai \\
Et qui aimer ou condamner \\
Le jour où j'aurai tout donné \\
Que mes claviers seront usés \\
D'avoir osé \\
Toujours vouloir tout essayer \\
Et recommencer là où le monde a commencé \\

Je m'en irai dormir dans le paradis blanc \\
Où les manchots s'amusent dès le soleil levant \\
Et jouent en nous montrant \\
Ce que c'est d'être vivant \\
Je m'en irai dormir dans le paradis blanc \\
Où l'air reste si pur \\
Qu'on se baigne dedans \\
A jouer avec le vent \\
Comme dans mes rêves d'enfant \\
Comme, comme, comme avant \\

Parler aux poissons d'argent \\
Et jouer avec le vent \\
Comme dans mes rêves d'enfant \\
Comme avant

\end{multicols}

\newpage
\normalsize
\h*{Il jouait du piano debout}
\begin{multicols}{2}
Ne me dites pas que ce garçon était fou \\
Il ne vivait pas comme les autres, c'est tout \\
Et pour quelles raisons étranges \\
Les gens qui n'sont pas comme nous, \\
Ça nous dérange \\

Ne me dites pas que ce garçon n'valait rien \\
Il avait choisi un autre chemin \\
Et pour quelles raisons étranges \\
Les gens qui pensent autrement \\
Ça nous dérange \\
Ça nous dérange \\

\begin{bfseries}
[Refrain:]\\
Il jouait du piano debout \\
C'est peut-être un détail pour vous \\
Mais pour moi, ça veut dire beaucoup \\
Ça veut dire qu'il était libre \\
Heureux d'être là malgré tout \\
Il jouait du piano debout \\
Quand les trouillards sont à genoux \\
Et les soldats au garde à vous \\
Simplement sur ses deux pieds, \\
Il voulait être lui, vous comprenez \\
\end{bfseries}
\columnbreak

Il n'y a que pour sa musique, qu'il était patriote \\
Il s'rait mort au champ d'honneur pour quelques notes \\
Et pour quelles raisons étranges, \\
Les gens qui tiennent à leurs rêves, \\
Ça nous dérange \\

Lui et son piano, ils pleuraient quelques fois \\
Mais c'est quand les autres n'étaient pas là \\
Et pour quelles raisons bizarres, \\
Son image a marqué ma mémoire, \\
Ma mémoire.. \\

\textbf{[Refrain]}\\

Il jouait du piano debout \\
Il chantait sur des rythmes fous \\
Et pour moi ça veut dire beaucoup \\
Ça veut dire essaie de vivre \\
Essaie d'être heureux, \\
Ça vaut le coup.
\end{multicols}

\newpage
\normalsize
\h*{Hélène}

Seul sur le sable les yeux dans l'eau \\
Mon rêve était trop beau \\
L'été qui s'achève tu partiras \\
A cent mille lieux de moi \\
Comment oublier ton sourire \\
Et tellement de souvenirs \\

Nos jeux dans les vagues près du quai \\
Je n'ai vu le temps passer \\
L'amour sur la plage désertée \\
Nos corps brûlés enlacés \\
Comment t'aimer si tu t'en vas \\
Dans ton pays loin là-bas babababa \\

Hélène things you do make me crazy about you \\
Pourquoi tu pars reste ici j'ai tant besoin d'une amie \\
Hélène things you do make me crazy about you \\
Pourquoi tu pars si loin de moi \\
La où le vent te porte loin de mon coeur qui bat \\

Hélène things you do make me crazy about you \\
Pourquoi tu pars reste ici reste encore juste une nuit \\

Seul sur le sable les yeux dans l'eau \\
Mon rêve était trop beau \\
L'été qui s'achève tu partiras \\
A cent mille lieux de moi \\
Comment t'aimer si tu t'en vas \\
Dans ton pays loin là-bas \\
Dans ton pays loin là-bas \\
Dans ton pays loin de moi

\newpage
\normalsize
\h*{Lucie}
\begin{multicols}{2}
Lucie, Lucie c'est moi je sais, \\
Il y a des soirs comme ça où tout... \\
s'écroule autour de vous. \\
Sans trop savoir pourquoi toujours \\

Regarder devant soi \\
Sans jamais baisser les bras, je sais... ) \\
C'est pas le remède à tout, \\
Mais 'faut se forcer parfois... \\

Lucie, Lucie dépêche toi, on vit, \\
On ne meurt qu'une fois... \\
Et on n'a le temps de rien, \\
Que c'est déjà la fin mais... \\

\begin{bfseries}
[Refrain:]\\
C'est pas marqué dans les livres, \\
Que le plus important à vivre, \\
Est de vivre au jour le jour. \\
Le temps c'est de l'Amour... \\
\end{bfseries}

Même, si je n'ai pas le temps, \\
D'assurer mes sentiments... \\
J'ai en moi, oh de plus en plus fort, \\
Des envies d'encore... \\

Tu sais, non, je n'ai plus à cœur, \\
De réparer mes erreurs ou de, \\
Refaire c'qu'est plus à faire : \\
Revenir en arrière... \\

Lucie, Lucie t'arrête pas, on ne vit \\
Qu'une vie à la fois... \\
A peine le temps de savoir, \\
Qu'il est déjà trop tard... \\

\textbf{[Refrain]}\\

Mmmm, Lucie, j'ai fait le tour, \\
De tant d'histoires d'amour. \\
J'ai bien, bien assez de courage, \\
Pour tourner d'autres pages, sâche... \\

Que le temps nous est compté. \\
Faut jamais se retourner en se disant, \\
"Que c'est dommage, \\
d'avoir passé l'âge" \\

Lucie, Lucie t'encombre pas \\
De souvenirs, de choses comme ça. \\
Aucun regret ne vaut le coup \\
Pour qu'on le garde en nous.. \\

\textbf{[Refrain]}\\
\end{multicols}

\newpage
\normalsize
\h*{Elle a les yeux revolver}

Un peu spéciale, elle est célibataire \\
Le visage pâle, les cheveux en arrière \\
Et j'aime ça \\
Elle se dessine sous des jupes fendues \\
Et je devine des histoires défendues \\
C'est comme ça \\
Tell'ment si belle quand elle sort \\
Tell'ment si belle, je l'aime tell'ment si fort \\

\begin{bfseries}
[Refrain:]\\
Elle a les yeux revolver, elle a le regard qui tue \\
Elle a tiré la première, m'a touché, c'est foutu \\
Elle a les yeux revolver, elle a le regard qui tue \\
Elle a tiré la première, elle m'a touché, c'est foutu \\
\end{bfseries}

Un peu larguée, un peu seule sur la terre \\
Les mains tendues, les cheveux en arrière \\
Et j'aime ça \\
A faire l'amour sur des malentendus \\
On vit toujours des moments défendus \\
C'est comme ça \\
Tell'ment si femme quand elle mord \\
Tell'ment si femme, je l'aime tell'ment si fort \\

\textbf{[Refrain]}\\

Son corps s'achève sous des draps inconnus \\
Et moi je rêve de gestes défendus \\
C'est comme ça \\
Un peu spéciale, elle est célibataire \\
Le visage pâle, les cheveux en arrière \\
Et j'aime ça \\
Tell'ment si femme quand elle dort \\
Tell'ment si belle, je l'aime tell'ment si fort \\

\textbf{[Refrain]}

\newpage
\small
\h*{Cendrillon}

Cendrillon pour ses vingt ans \\
Est la plus jolie des enfants \\
Son bel amant, le prince charmant \\
La prend sur son cheval blanc \\
Elle oublie le temps \\
Dans ce palais d'argent \\
Pour ne pas voir qu'un nouveau jour se lève \\
Elle ferme les yeux et dans ses rêves \\
Elle part, jolie petite histoire \\
Elle part, jolie petite histoire \\

Cendrillon pour ses trente ans \\
Est la plus triste des mamans \\
Le prince charmant a foutu l'camp \\
Avec la belle au bois dormant \\
Elle a vu cent chevaux blancs \\
Loin d'elle emmener ses enfants \\
Elle commence à boire \\
A traîner dans les bars \\
Emmitouflée dans son cafard \\
Maintenant elle fait le trottoir \\
Elle part, jolie petite histoire \\
Elle part, jolie petite histoire \\

Dix ans de cette vie ont suffi \\
A la changer en junkie \\
Et dans un sommeil infini \\
Cendrillon voit finir sa vie \\
Les lumières dansent \\
Dans l'ambulance \\
Mais elle tue sa dernière chance \\
Tout ça n'a plus d'importance \\
Elle part \\
Fin de l'histoire \\

Notre père, qui est si vieux \\
As-tu vraiment fait de ton mieux ? \\
Car sur la terre et dans les cieux \\
Tes anges n'aiment pas devenir vieux. \\


\newpage
\normalsize
\h*{Aux Champs élysées}
\begin{multicols}{2}

Je m'baladais sur l'avenue \\
Le cœur ouvert a l'inconnu \\
J'avais envie de dire bonjour \\
À n'importe qui \\

N'importe qui, et ce fut toi \\
Et je te dis n'importe quoi \\
Il suffisait de te parler \\
Pour t’apprivoiser \\

\begin{bfseries}
[Refrain:]\\
Aux champs Elysées \\
Aux champs Elysées \\

Au soleil sous la pluie \\
A midi ou à minuit \\
Il y a tout ce que vous voulez \\
Aux champs Elysées \\
\end{bfseries}

Tu m’as dis j’ai rendez-vous \\
Dans un sous sol avec des fous \\
Qui vivent la guitare a la main \\
Du soir au matin \\

Alors je t’ai accompagnée \\
On a chanté on a dansé \\
Et on n’a même pas pensé \\
A s’embrasser \\

\textbf{[Refrain]}\\

Hier soir deux inconnus \\
Et ce matin sur l’avenue \\
Deux amoureux tout étourdis \\
Par la longue nuit \\

Et de l’étoile a la concorde \\
Un orchestre a mille cordes \\
Tous les oiseaux du point du jour \\
Chantent l’amour \\

\textbf{[Refrain] (x2)}\\

Aux champs Elysées \\

Aux champs Elysées

\end{multicols}

\newpage
\normalsize
\h*{Belle histoire}
\begin{multicols}{2}
C'est un beau roman\\
C'est une belle histoire\\
C'est une romance d'aujourd'hui\\
Il rentrait chez lui, là-haut vers le brouillard\\
Elle descendait dans le Midi, le Midi\\

Ils se sont trouvés au bord du chemin\\
Sur l'autoroute des vacances\\
C'était sans doute un jour de chance\\
Ils avaient le ciel à portée de main\\
Un cadeau de la Providence\\
Alors, pourquoi penser aux lendemains\\

Ils se sont cachés dans un grand champ de blé\\
Se laissant porter par le courant\\
Se sont raconté leurs vies qui commençaient\\
Ils n'étaient encore que des enfants, des enfants\\
Qui s'étaient trouvés au bord du chemin\\
Sur l'autoroute des vacances\\
C'était sans doute un jour de chance\\
Qui cueillirent le ciel au creux de leur main\\
Comme on cueille la Providence\\
Refusant de penser aux lendemains\\

C'est un beau roman\\
C'est une belle histoire\\
C'est une romance d'aujourd'hui\\
Il rentrait chez lui, là-haut vers le brouillard\\
Elle descendait dans le Midi, le Midi\\

Ils se sont quittés au bord du matin\\
Sur l'autoroute des vacances\\
C'était fini le jour de chance\\
Ils reprirent alors chacun leur chemin\\
Saluèrent la Providence\\
En se faisant un signe de la main\\

Il rentra chez lui, là-haut vers le brouillard\\
Elle est descendue là-bas dans le Midi\\
C'est un beau roman\\
C'est une belle histoire\\
C'est une romance d'aujourd'hui

\end{multicols}

\newpage
\normalsize
\h*{Henry}
\begin{multicols}{2}
Je m'présente, je m'appelle Henry\\
J'voudrais bien réussir ma vie, être aimé\\
Être beau gagner de l'argent\\
Puis surtout être intelligent\\
Mais pour tout ça il faudrait que\\
j'bosse à plein temps\\

J'suis chanteur, je chante pour mes copains\\
J'veux faire des tubes et que ça tourne bien, tourne bien\\
J'veux écrire une chanson dans le vent\\
Un air gai, chic et entraînant\\
Pour faire danser dans les soirées de\\
Monsieur Durand\\

\begin{bfseries}
[Refrain:]\\
Et partout dans la rue\\
J'veux qu'on parle de moi\\
Que les filles soient nues\\
Qu'elles se jettent sur moi\\
Qu'elles m'admirent, qu'elles me tuent\\
Qu'elles s'arrachent ma vertu\\
\end{bfseries}

Pour les anciennes de l'école\\
Devenir une idole\\
J'veux que toutes les nuits\\
Essoufflées dans leurs lits\\
Elles trompent leurs maris\\
Dans leurs rêves maudits\\
\columnbreak

Puis après je f'rai des galas\\
Mon public se prosternera devant\\
moi\\
Des concerts de cent mille personnes\\
Où même le tout-Paris s'étonne\\
Et se lève pour prolonger le combat\\

\textbf{[Refrain]}\\

Puis quand j'en aurai assez\\
De rester leur idole\\
Je remont'rai sur scène\\
Comme dans les années folles\\
Je f'rai pleurer mes yeux\\
Je ferai mes adieux\\

Et puis l'année d'après\\
Je recommencerai\\
Et puis l'année d'après\\
Je recommencerai\\
Je me prostituerai\\
Pour la postérité\\

Les nouvelles de l'école\\
Diront que j'suis pédé\\
Que mes yeux puent l'alcool\\
Que j'fais bien d'arrêter\\
Brûleront mon auréole\\
Saliront mon passé\\

Alors je serai vieux\\
Et je pourrai crever\\
Je me cherch'rai un Dieu\\
Pour tout me pardonner\\
J'veux mourir malheureux\\
Pour ne rien regretter
\end{multicols}

\newpage
\normalsize
\vbox{
\begin{minipage}[t][0.4\textheight][t]{\textwidth}
\h*{Torremolinos}
\small
Il y a une ville à la Costa Del Sol\\
Où il y a plus de Belges que d'Espagnols\\
Où il y a plus de Leonidas\\
Et de Bata que de Gambas\\

On ira tous tous tous à Torremolinos\\
Tous tous tous à Torre-molinos\\
Tous tous tous à Torremolinos\\
Tous tous tous à Torre-molinos\\

Même quand il pleut c'est génial\\
On sait poster des cartes postales\\
Ou boire un godet à l'Amicale\\
Des amis du camping municipal\\
Avec Sunair c'est super\\
Avec Airtour c'est l'aller et le retour\\
Avec Neckermann c'est géniann\\
Avec Nouvelles Frontières c'est pas cher\\

Pour moi Torremolinos c'est le paradis\\
Je crois bien que je vais mourir ici\\
Et quand je serai mort je veux qu'on m'enterre\\
A Torremolinos, son cimetière
\end{minipage}

\vspace{0.08\textheight}
\begin{minipage}[t][0.65\textheight][t]{\textwidth}
\h*{Quitter l’enfance}
\begin{multicols}{2}
\footnotesize
Un rien te fait rougir\\
Comme un feu de brindilles\\
Et tu hais les familles\\
Comme j'ai pu les haïr\\

Un rien te fait rougir\\
Et tu voudrais changer\\
Le sens du verbe aimer\\
Sans les mots pour le dire\\

Je lis ton innocence\\
Dans le noir de tes bas\\
Tu peux quitter l'enfance\\
Ton enfance\\
Ne te quitte pas\\

Tes lèvres se colorent\\
Maintenant que tu veilles\\
Et tes poupées sont vieilles\\
Maintenant que tu sors\\

Maintenant que tu danses\\
À chacun de tes pas\\
Tu peux quitter l'enfance\\
Ton enfance\\
Ne te quitte pas\\

Je ne peux rien te dire\\
Je n'ai pas été fille\\
J'ai haï les familles\\
Comme tu peux les haïr

\end{multicols}
\end{minipage}
}

\newpage
\h*{La vie par procuration}
\normalsize
\begin{multicols}{2}
\begin{bfseries}
[Refrain:]\\
Elle met du vieux pain sur son balcon\\
Pour attirer les moineaux les pigeons\\
Elle vit sa vie par procuration\\
Devant son poste de télévision\\
\end{bfseries}

Lever sans réveil, avec le soleil\\
Sans bruit, sans angoisse, la journée se passe\\
Repasser, poussière, y a toujours à faire\\
Repas solitaire, en point de repère\\

La maison si nette, qu'elle en est suspecte\\
Comme tous ces endroits où l'on ne vit pas\\
Les êtres ont cédés, perdu la bagarre\\
Les choses ont gagné, c'est leur territoire\\

Le temps qui nous casse, ne la change pas\\
Les vivants se fanent, mais les ombres pas\\
Tout va, tout fonctionne, sans but sans pourquoi\\
D'hiver en automne, ni fièvre ni froid\\

\textbf{[Refrain]}\\

Elle apprend dans la presse à scandale\\
La vie des autres qui s'étale\\
Mais finalement de moins pire en banal\\
Elle finira par trouver ça normal\\

Elle met du vieux pain sur son balcon\\
Pour attirer les moineaux les pigeons\\

Des crèmes et des bains qui font la peau douce\\
Mais ça fait bien loin que personne ne la touche\\
Des mois des années sans personne à aimer\\
Et jour après jour l'oubli de l'amour\\

Ses rêves et désirs si sages, si possible\\
Sans cri, sans délires sans inadmissible\\
Sur dix ou vingt pages de photos banales\\
Bilan sans mystères d'années sans lumière\\

\textbf{[Refrain]}
\\

Elle apprend dans la presse à scandale\\
La vie des autres qui s'étale\\
Mais finalement de moins pire en banal\\
Elle finira par trouver ça normal\\

Elle met du vieux pain sur son balcon\\
Pour attirer les moineaux les pigeons\\

Elle apprend dans la presse à scandale\\
La vie des autres qui s'étale\\
Mais finalement de moins pire en banal\\
Elle finira par trouver ça normal\\

Elle met du vieux pain sur son balcon\\
Pour attirer les moineaux les pigeons
\end{multicols}

\newpage
\normalsize
\h*{Alors regarde}

Le sommeil veut pas d' moi, tu rêves depuis longtemps.\\
Sur la télé la neige a envahi l'écran.\\
J'ai vu des hommes qui courent, une terre qui recule,\\
Des appels au secours, des enfants qu'on bouscule.\\
Tu dis qu' c'est pas mon rôle de parler de tout ça,\\
Qu'avant d' prendre la parole il faut aller là-bas.\\
Tu dis qu' c'est trop facile, tu dis qu' ça sert à rien,\\
Mais c't encore plus facile de ne parler de rien.\\

\begin{bfseries}
[Refrain:]\\
Alors regarde, regarde un peu...\\
Je vais pas me taire parce que t'as mal aux yeux.\\
Alors regarde, regarde un peu...\\
Tu verras tout c' qu'on peut faire si on est deux.\\
\end{bfseries}

Perdue dans tes nuances, la conscience au repos,\\
Pendant qu' le monde avance, tu trouves pas bien tes mots.\\
T' hésites entre tout dire et un drôle de silence.\\
T'as du mal à partir, alors tu joues l'innocence.\\

\textbf{[Refrain]}\\

Dans ma tête une musique vient plaquer ses images\\
Sur des rythmes d'Afrique mais j' vois pas l' paysage\\
toujours ces hommes au courent, une terre qui recule;\\
Des appels au secours des enfants qu'on bouscule\\

\textbf{[Refrain] (x2)}

\newpage
\normalsize
\h*{Imagine}

Imagine there's no heaven\\
It's easy if you try\\
No hell below us\\
Above us only sky\\
Imagine all the people\\
Living for today...\\

Imagine there's no countries\\
It isn't hard to do\\
Nothing to kill or die for\\
And no religion too\\
Imagine all the people\\
Living life in peace...\\

You may say I'm a dreamer\\
But I'm not the only one\\
I hope someday you'll join us\\
And the world will be as one\\

Imagine no possessions\\
I wonder if you can\\
No need for greed or hunger\\
A brotherhood of man\\
Imagine all the people\\
Sharing all the world...\\

You may say I'm a dreamer\\
But I'm not the only one\\
I hope someday you'll join us\\
And the world will live as one\\

\newpage
\normalsize
\h*{Let it be}

When I find myself in times of trouble\\
Mother Mary comes to me\\
Speaking words of wisdom, let it be.\\

And in my hour of darkness\\
She is standing right in front of me\\
Speaking words of wisdom, let it be.\\

\begin{bfseries}
[Refrain:]\\
Let it be, let it be, let it be, let it be,\\
Whisper words of wisdom, let it be !\\

\end{bfseries}

And when the broken hearted people\\
Living in the world agree,\\
There will be an answer, let it be.\\

For though they may be parted there is\\
Still a chance that they will see\\
There will be an answer, let it be.\\

Let it be, let it be, let it be, let it be,\\
Yeah there will be an answer, let it be !\\

\textbf{[Refrain] (x2)}\\

And when the night is cloudy,\\
There is still a light that shines on me,\\
Shine on until tomorrow, let it be.\\

I wake up to the sound of music\\
Mother Mary comes to me\\
Speaking words of wisdom, let it be.\\

Let it be, let it be, let it be, let it be,\\
Yeah there will be an answer, let it be !\\

\textbf{[Refrain]}

\newpage
\normalsize
\h*{Les miroirs dans la boue}
Dans l'orage d'une forêt sans âge\\
Aux abords du Poitou\\
A l'automne où je vivais chez vous\\
J'ai vu le visage d'une enfant sauvage\\
Qui portait un bijou\\
Les yeux verts noyés de cheveux roux\\

A l'automne où je vivais chez vous\\
Dieu fait des images avec les nuages\\
La pluie fait des miroirs dans la boue\\
Je t'ai cherchée partout\\
Je garde un mirage dans une drôle de cage\\
Comme savent construire les fous\\
Je t'ai cherchée partout\\

Elle avait l'âge des vagabondages\\
Pieds nus sur les cailloux\\
Dans les rivières où viennent boire les loups\\
A mon passage elle a pris mon bagage\\
Elle m'a suivi partout\\
Jusqu'à l'étage où j'avais mon verrou\\
Les yeux verts noyés de cheveux roux\\

\begin{bfseries}
[Refrain]:
Dieu fait des images avec les nuages\\
La pluie fait des miroirs dans la boue\\
Je t'ai cherchée partout\\
Je garde un mirage dans une drôle de cage\\
Comme savent construire les fous\\
Je t'ai cherchée partout\\
\end{bfseries}

Au lendemain de l'orage\\
Il restait un message\\
Vous me plaisiez beaucoup\\
Mais je n'pense pas avoir besoin de vous\\
Les yeux verts noyés de cheveux roux\\

\textbf{[Refrain]}

\newpage
\normalsize
\h*{L’aventurier}

Egaré dans la vallée infernale\\
Le héros s'appelle Bob Morane\\
A la recherche de l'Ombre Jaune\\
Le bandit s'appelle Mister Kali Jones\\
Avec l'ami Bill Ballantine\\
Sauvé de justesse des crocodiles\\
Stop au trafic des Caraïbes\\
Escale dans l'opération Nadawieb.\\

Le cœur tendre dans le lit de Miss Clark\\
Prisonnière du Sultan de Jarawak\\
En pleine terreur à Manicouagan\\
Isolé dans la jungle Birmane\\
Emprisonnant les flibustiers\\
L'ennemi est démasqué\\
On a volé le collier de Civa\\
Le Maharadjah en répondra.\\

\begin{bfseries}
[Refrain]:
Et soudain surgit face au vent\\
Le vrai héros de tous les temps\\
Bob Morane contre tout chacal\\
L'aventurier contre tout guerrier\\
Bob Morane contre tout chacal\\
L'aventurier contre tout guerrier\\
\end{bfseries}

Dérivant à bord du sampang\\
L'aventure au parfum d'Ylalang\\
Son surnom, Samouraï du Soleil\\
En démantelant le gang de l'Archipel\\
L'otage des guerriers du Doc Xhatan\\
Il s'en sortira toujours à temps\\
Tel l'aventurier solitaire\\
Bob Morane est le roi de la Terre.\\

\textbf{[Refrain]}

\newpage
\normalsize
\h*{Pas toi}
Graver l'écorce jusqu'à saigner\\
Clouer les portes, s'emprisonner\\
Vivre des songes à trop veiller\\
Prier des ombres et tant marcher\\
J'ai beau me dire qu'il faut du temps\\
J'ai beau l'écrire si noir sur blanc\\

Quoi que je fasse, où que je sois\\
Rien ne t'efface, je pense à toi\\

Passent les jours, vides sillons\\
Dans la raison et sans amour\\
Passe ma chance, tournent les vents\\
Reste l'absence, obstinément\\
J'ai beau me dire que c'est comme ça\\
Que sans vieillir, on n'oublie pas\\

Quoi que je fasse, où que je sois\\
Rien ne t'efface, je pense a toi\\
Et quoi que j'apprenne, je ne sais pas\\
Pourquoi je saigne et pas toi\\

Y a pas de haine, y a pas de roi\\
Ni dieu ni chaînes, qu'on ne combat\\
Mais que faut-il, quelle puissance\\
Quelle arme brise l'indifférence ?\\
Oh c'est pas juste, c'est mal écrit\\
Comme une injure, plus qu'un mépris\\

Quoi que je fasse, où que je sois\\
Rien ne t'efface, je pense à toi\\
Et quoi que j'apprenne, je ne sais pas\\
Pourquoi je saigne et pas toi\\
Et pas toi!\\
Et pas toi!\\
Ouhouhou

\newpage
\normalsize
\h*{Yesterday}
Yesterday\\
All my troubles seemed so far away\\
Now it looks as though they're here to stay\\
Oh I believe in yesterday\\

Suddenly\\
I'm not half the man I used to be\\
There the shadow hanging over me\\
Oh yesterday came suddenly\\

Why she had to go I don't know\\
She wouldn't say\\
I said something wrong\\
Now I long for yesterday\\

Yesterday\\
Love was such an easy game to play\\
Now I need a place to hide away\\
Oh I believe in yesterday\\

Why she had to go I don't know\\
She wouldn't say\\
I said something wrong\\
Now I long for yesterday\\

Yesterday\\
Love was such an easy game to play\\
Now I need a place to hide away\\
Oh I believe in yesterday\\

Mmm...

\newpage
\normalsize
\h*{Famille}

\begin{multicols}{2}
Et crever le silence \\
Quand c'est à toi que je pense \\
Je suis loin de tes mains \\
Loin de toi, loin des tiens \\
Mais tout ça n'a pas d'importance \\

J'connais pas ta maison \\
Ni ta ville, ni ton nom \\
Pauvre, riche, batard \\
Blanc, tout noir ou bizarre \\
Je reconnais ton regard \\

Et tu cherches une image \\
Et tu cherches un endroit \\
Où je dérive parfois \\

\begin{bfseries}
[Refrain]:\\
Tu es de ma famille \\
De mon ordre et de mon rang \\
Celle que j'ai choisie \\
Celle que je ressens \\
Dans cette armée de simple gens \\

Tu es de ma famille \\
Bien plus que celle du sang \\
Des poignées de secondes \\
Dans cet étrange monde \\
Qu'il te protège s'il entend \\
\end{bfseries}

\columnbreak

Tu sais pas bien où tu vas \\
Ni bien comment, ni pourquoi \\
Tu crois pas à grand chose \\
Ni tout gris, ni tout rose \\
Mais ce que tu crois, c'est à toi \\

T'es du parti des perdants \\
Consciemment, viscéralement \\
Et tu regardes en bas \\
Mais tu tomberas pas \\
Tant qu'on aura besoin de toi \\

Et tu prends les bonheurs \\
Comme grains de raisin \\
Petits bouts de petits riens \\

\textbf{[Refrain]}\\

Tu es de ma famille x2 \\
Du même rang, du même vent \\
Tu es de ma famille x2 \\
Même habitants du même temps \\
Tu es de ma famille x2 \\
Croisons nos vies de temps en temps 
\end{multicols}

\newpage
\small
\h*{Tous les garçons et les filles}

% \begin{multicols}{2}
Tous les garçons et les filles de mon âge \\
Se promènent dans la rue deux par deux \\
Tous les garçons et les filles de mon âge \\
Savent bien ce que c'est d'être heureux \\
Et les yeux dans les yeux \\
Et la main dans la main \\
Ils s'en vont, amoureux \\
Sans peur du lendemain \\
Oui mais moi, je vais seule \\
Par les rues, l'âme en peine \\
Oui mais moi, je vais seule \\
Car personne ne m'aime \\

Mais je forme mes nuits \\
Sont en tous points pareils \\
Sans joies et pleins d'ennuis \\
Personne ne murmure je t'aime à mon oreille \\

Tous les garçons et les filles de mon âge \\
Font ensembles des projets d'avenir \\
Tous les garçons et les filles de mon âge \\
Savent très bien ce qu'aimer veut dire \\
Et les yeux dans les yeux \\
Et la main dans la main \\
Ils s'en vont amoureux \\
Sans peur du lendemain \\
Oui mais moi, je vais seule \\
Par les rues, l'âme en peine \\
Oui mais moi, je vais seule \\
Car personne ne m'aime \\

Mais je forme mes nuits \\
Sont en tous points pareils \\
Sans joies et pleins d'ennuis \\
Pour quand donc pour moi brillera le soleil \\

Comme les garçons et les filles de mon âge \\
Connaître-je bientôt ce qu'est l'amour \\
Comme les garçons et les filles de mon âge \\
Je me demande quand viendra le jour \\
Où les yeux dans ses yeux \\
Et la main dans sa main \\
J'aurai le cœur heureux \\
Sans peur du lendemain \\
Le jour où je n'aurai plus du tout l'âme en peine \\
Le jour où, moi aussi, j'aurai quelqu'un qui m'aime
% \end{multicols}

\newpage
\small
\h*{Belle}

\begin{multicols}{2}
Belle \\
C'est un mot qu'on dirait inventé pour elle \\
Quand elle danse et qu'elle met son corps à jour, tel \\
Un oiseau qui étend ses ailes pour s'envoler \\
Alors je sens l'enfer s'ouvrir sous mes pieds \\

J'ai posé mes yeux sous sa robe de gitane \\
A quoi me sert encore de prier Notre-Dame ? \\
Quel \\
Est celui qui lui jettera la première pierre ? \\
Celui-là ne mérite pas d'être sur terre \\

Ô Lucifer ! \\
Oh ! Laisse-moi rien qu'une fois \\
Glisser mes doigts dans les cheveux d'Esmeralda \\

Belle \\
Est-ce le diable qui s'est incarné en elle \\
Pour détourner mes yeux du Dieu éternel ? \\
Qui a mis dans mon être ce désir charnel \\
Pour m'empêcher de regarder vers le Ciel ? \\

Elle porte en elle le péché originel \\
La désirer fait-il de moi un criminel ? \\
Celle \\
Qu'on prenait pour une fille de joie, une fille de rien \\
Semble soudain porter la croix du genre humain \\

Ô Notre-Dame ! \\
Oh ! Laisse-moi rien qu'une fois \\
Pousser la porte du jardin d'Esmeralda \\

Belle \\
Malgré ses grands yeux noirs qui vous ensorcellent \\
La demoiselle serait-elle encore pucelle? \\
Quand ses mouvements me font voir monts et merveilles \\
Sous son jupon aux couleurs de l'arc-en-ciel \\

Ma dulcinée laissez-moi vous être infidèle \\
Avant de vous avoir mené jusqu'à l'autel \\
Quel \\
Est l'homme qui détournerait son regard d'elle \\
Sous peine d'être changé en statue de sel \\

Ô Fleur-de-Lys \\
Je ne suis pas homme de foi \\
J'irai cueillir la fleur d'amour d'Esmeralda \\

J'ai posé mes yeux sous sa robe de gitane \\
À quoi me sert encore de prier Notre-Dame \\
Quel \\
Est celui qui lui jettera la première pierre \\
Celui-là ne mérite pas d'être sur terre \\

Ô Lucifer ! \\
Oh ! Laisse-moi rien qu'une fois \\
Glisser mes doigts dans les cheveux d'Esmeralda \\

Esmeralda…

\end{multicols}

\newpage
\normalsize
\h*{Le coup de soleil}

\begin{multicols}{2}

J'ai attrappé un coup de soleil, \\
Un coup d'amour, un coup d'je t'aime \\
J'sais pas comment, il faut qu'j'me rappelle \\
Si c'est un rêve, t'es super belle \\
J'dors plus la nuit, j'fais des voyages \\
Sur des bateaux qui font naufrages \\
J'te vois toute nue sur du satin \\
Et j'en dors plus, viens m'voir demain \\

Mais tu n'es pas là, et si je rêve tant pis \\
Quand tu t'en vas j'dors plus la nuit \\
Mais tu n'es pas là, et tu sais, j'ai envie d'aller là-bas \\
La fenêtre en face et d'visiter ton paradis. \\

J'mets tes photos dans mes chansons \\
Et des voiliers dans ma maison \\
J'voulais m'tirer, mais j'me tire plus \\
J'vis à l'envers, j'aime plus ma rue, \\
J'avais cent ans, j'me r'connais plus \\
J'aime plus les gens depuis qu'j't'ai vue \\
J'veux plus rêver, j'voudrais qu'tu viennes \\
Me faire voler, me faire je t'aime. \\

Mais tu n'es pas là, et si je rêve tant pis \\
Quand tu t'en vas j'dors plus la nuit \\
Mais tu n'es pas là, et tu sais, j'ai envie d'aller là-bas \\
La fenêtre en face et d'visiter ton paradis. \\

Ça y est, c'est sûr, faut qu'j'me décide \\
J'vais faire le mur et j'tombe dans l'vide \\
J'sais qu'tu m'attends près d'la fontaine \\
J't'ai vu descendre d'un arc-en-ciel \\
Je m'jette à l'eau des pluies d'été \\
J'fais du bateau dans mon quartier \\
Il fait très beau, on peut ramer \\
La mer est calme, on peut s'tirer \\

Mais tu n'es pas là, et si je rêve tant pis \\
Quand tu t'en vas j'dors plus la nuit \\
Mais tu n'es pas là, et tu sais, j'ai envie d'aller là-bas \\
Le fenêtre en face et d'visiter ton paradis. \\

Mais tu n'es pas là lalalala \\
Mais tu n'es pas là, et si je rêve tant pis \\
Quand tu t'en vas j'dors plus la nuit \\

Mais tu n'es pas là, \\
Mais tu n'es pas là \\

J'ai attrappé un coup de soleil, \\
Un coup d'amour, un coup d'je t'aime \\
J'sais pas comment, il faut qu'j'me rappelle \\
Et si je rêve tanpis \\

J'ai attrappé un coup de soleil, \\
Un coup d'amour, un coup d'je t'aime \\
Un coup d'amour, un coup d'je t'aime
\end{multicols}

\newpage
\normalsize
\vbox{
\begin{minipage}[t][0.4\textheight][t]{\textwidth}
\h*{La terre chante les couleurs}
\small
\begin{bfseries}
[Refrain]:\\
La terre chante les couleurs, les couleurs \\
Que Dieu à mises dans nos mains, dans nos mains. \\
\end{bfseries}

\begin{enumerate}
\item Crayon bleu, crayon bleu, Dieu dessine le ciel \\
Crayon noir, crayon noir, Dieu dessine la nuit \\
\item Crayon gris, crayon gris, Dieu dessine la pluie \\
Crayon d’or, crayon d’or, Dieu dessine le soleil \\
\item Crayon brun, crayon brun, Dieu dessine la terre \\
Crayon roux, crayon roux, Dieu dessine le feu \\
\item Crayon vert, crayon vert, Dieu dessine la mer \\
Crayon nid, crayon nid, Dieu dessine un oiseau \\
\item Crayon blanc, crayon blanc, Dieu dessine un enfant \\
Crayon Dieu, crayon Dieu, Dieu dessine l’amour. \\
\end{enumerate}
\end{minipage}
\vspace{0.08\textheight}
\begin{minipage}[t][0.65\textheight][t]{\textwidth}
\h*{La tendresse}
\begin{multicols}{2}
\footnotesize

On peut vivre sans richesse \\
Presque sans le sou \\
Des seigneurs et des princesses \\
Y'en a plus beaucoup \\
Mais vivre sans tendresse \\
On ne le pourrait pas \\
Non, non, non, non \\
On ne le pourrait pas \\

On peut vivre sans la gloire \\
Qui ne prouve rien \\
Être inconnu dans l'histoire \\
Et s'en trouver bien \\
Mais vivre sans tendresse \\
Il n'en est pas question \\
Non, non, non, non \\
Il n'en est pas question \\

Quelle douce faiblesse \\
Quel joli sentiment \\
Ce besoin de tendresse \\
Qui nous vient en naissant \\
Vraiment, vraiment, vraiment \\

Dans le feu de la jeunesse \\
Naissent les plaisirs \\
Et l'amour fait des prouesses \\
Pour nous éblouir \\
Oui mais sans la tendresse \\
L'amour ne serait rien \\
Non, non, non, non \\
L'amour ne serait rien \\

Un enfant vous embrasse \\
Parce qu'on le rend heureux \\
Tous nos chagrins s'effacent \\
On a les larmes aux yeux \\
Mon Dieu, mon Dieu, mon Dieu... \\
Dans votre immense sagesse \\
Immense ferveur \\
Faites donc pleuvoir sans cesse \\
Au fond de nos cœurs \\
Des torrents de tendresse \\
Pour que règne l'amour \\
Règne l'amour \\
Jusqu'à la fin des jours

\end{multicols}
\end{minipage}
}

\newpage
\normalsize
\h*{Stewball}

\begin{bfseries}
[Refrain]:\\
Il s'appelait Stewball. \\
C'était un cheval blanc. \\
Il était mon idole \\
Et moi, j'avais dix ans. \\
\end{bfseries}

Notre pauvre père, \\
Pour acheter ce pur sang, \\
Avait mis dans l'affaire \\
Jusqu'à son dernier franc. \\

Il avait dans la tête \\
D'en faire un grand champion \\
Pour liquider nos dettes \\
Et payer la maison \\

Et croyait à sa chance. \\
Il engagea Stewball \\
Par un beau dimanche \\
Au grand prix de St-Paul. \\

"Je sais, dit mon père, \\
Que Stewball va gagner." \\
Mais, après la rivière, \\
Stewball est tombé. \\

Quand le vétérinaire, \\
D'un seul coup, l'acheva, \\
J'ai vu pleurer mon père \\
Pour la première fois. \\

\textbf{[Refrain]}

\newpage
\normalsize
\h*{Légende indienne}

Quand Dieu fit l'univers il y eut sur la terre \\
Des milliers d'animaux inconnus aujourd'hui \\
Mais la plus jolie dans ce vert paradis \\
La plus drôle la plus mignonne, c'était la licorne \\

Y avait des gros crocodiles et des orangs-outangs \\
Des affreux reptiles et des jolis moutons blancs \\
Des chats des rats des éléphants mais la plus mignonne \\
De toutes les bêtes à cornes, c'était la licorne. \\

Quand il vit les pécheurs faire leurs premiers péchés \\
Dieu se mit en colère et appela Noé: \\
Mon bon vieux Noé, je vais noyer la terre \\
Construis-moi un grand bateau pour flotter sur l'eau \\

Mets y des gros crocodiles et des orangs-outangs \\
Des affreux reptiles et des jolis moutons blancs \\
Des chats des rats des éléphants mais n'oublie pas \\
La mignonne, la jolie licorne. \\

Quand son bateau fut prêt à surmonter les flots, \\
Noé y fit monter les animaux deux par deux \\
Et déjà la pluie commençait à tomber quand il cria \\
Seigneur! j'ai fait pour le mieux \\

J'ai mis deux gros crocodiles et des orangs-outangs \\
Des affreux reptiles et des jolis moutons blancs \\
Des chats des rats des éléphants mais il n'y manque personne, \\
À part les deux mignonnes, les jolies licornes. \\

Elles riaient les mignonnes et pataugeaient dans l'eau, \\
S'amusant comme des folles, sans voir que le bateau \\
Emmené par Noé, les avait oubliées \\
Et depuis jamais personne n'a vu de licorne \\

On voit des gros crocodiles et des orangs-outangs \\
Des affreux reptiles et des jolis moutons blancs \\
Des chats des rats des éléphants mais jamais personne \\
Ne verra la mignonne, la jolie licorne!

\newpage
\normalsize
\h*{Le petit bonheur}

\begin{multicols}{2}
C'était un petit bonheur \\
Que j'avais ramassé \\
Il était tout en pleurs \\
Sur le bord d'un fossé \\
Quand il m'a vu passer \\
Il s'est mis à crier: \\
"Monsieur, ramassez-moi \\
Chez vous amenez-moi \\

Mes frères m'ont oublié, je suis tombé, je suis malade \\
Si vous n'me cueillez point, je vais mourir, quelle ballade ! \\
Je me ferai petit, tendre et soumis, je vous le jure \\
Monsieur, je vous en prie, délivrez-moi de ma torture" \\

J'ai pris le p'tit bonheur \\
L'ai mis sous mes haillons \\
J'ai dit: " Faut pas qu'il meure \\
Viens-t'en dans ma maison " \\
Alors le p'tit bonheur \\
A fait sa guérison \\
Sur le bord de mon cœur \\
Y avait une chanson \\

Mes jours, mes nuits, mes peines, mes deuils, mon mal, tout fut oublié \\
Ma vie de désœuvré, j'avais dégoût d'la r'commencer \\
Quand il pleuvait dehors ou qu'mes amis m'faisaient des peines \\
J'prenais mon p'tit bonheur et j'lui disais: "C'est toi ma reine" \\

Mon bonheur a fleuri \\
Il a fait des bourgeons \\
C'était le paradis \\
Ça s'voyait sur mon front \\
Or un matin joli \\
Que j'sifflais ce refrain \\
Mon bonheur est parti \\
Sans me donner la main \\

J'eus beau le supplier, le cajoler, lui faire des scènes \\
Lui montrer le grand trou qu'il me faisait au fond du cœur \\
Il s'en allait toujours, la tête haute, sans joie, sans haine \\
Comme s'il ne pouvait plus voir le soleil dans ma demeure \\

J'ai bien pensé mourir \\
De chagrin et d'ennui \\
J'avais cessé de rire \\
C'était toujours la nuit \\
Il me restait l'oubli \\
Il me restait l'mépris \\
Enfin que j'me suis dit: \\
Il me reste la vie \\

J'ai repris mon bâton, mes deuils, mes peines et mes guenilles \\
Et je bats la semelle dans des pays de malheureux \\
Aujourd'hui quand je vois une fontaine ou une fille \\
Je fais un grand détour ou bien je me ferme les yeux \\
...Je fais un grand détour ou bien je me ferme les yeux... \\

\end{multicols}

\newpage
\normalsize
\vbox{
\begin{minipage}[t][0.4\textheight][t]{\textwidth}
\h*{Le pénitencier}
\normalsize
\begin{multicols}{2}
Les portes du pénitencier \\
Bientôt vont se fermer \\
Et c'est là que je finirai ma vie \\
Comme d'autres gars l'ont finie \\
Pour moi ma mère a donné \\
Sa robe de mariée \\
Peux-tu jamais me pardonner \\
Je t'ai trop fait pleurer \\
Le soleil n'est pas fait pour nous \\
C'est la nuit qu'on peut tricher \\
Toi qui ce soir a tout perdu \\
Demain tu peux gagner. \\

O mères, écoutez-moi \\
Ne laissez jamais vos garçons \\
Seuls la nuit traîner dans les rues \\
Ils iront tout droit en prison \\
Toi la fille qui m'a aimé \\
Je t'ai trop fait pleurer \\
Les larmes de honte que tu as versées \\
Il faut les oublier \\
Les portes du pénitencier \\
Bientôt vont se fermer \\
Et c'est là que je finirai ma vie \\
Comme d'autres gars l'ont finie.

\end{multicols}
\end{minipage}
\vspace{0.08\textheight}
\begin{minipage}[t][0.65\textheight][t]{\textwidth}
\h*{Je n’aurai pas le temps}
\begin{multicols}{2}
\normalsize

Je n'aurai pas le temps, pas le temps \\
Même en courant \\
Plus vite que le vent \\
Plus vite que le temps \\
Même en volant \\
Je n'aurai pas le temps, pas le temps \\
De visiter toute l'immensité \\
D'un si grand univers \\
Même en cent ans \\
Je n'aurai pas le temps de tout faire \\

J'ouvre tout grand mon coeur \\
J'aime de tous mes yeux \\
C'est trop peu \\
Pour tant de coeurs et tant de fleurs \\
Des milliers de jours \\
C'est bien trop court, bien trop court \\

Et pour aimer \\
Comme l'on doit aimer quand on aime vraiment \\
Même en cent ans \\
Je n'aurai pas le temps, pas le temps \\

J'ouvre tout grand mon coeur \\
J'aime de tous mes yeux \\
C'est trop peu \\
Pour tant de coeurs et tant de fleurs \\
Des milliers de jours \\
C'est bien trop court, c'est bien trop court.

\end{multicols}
\end{minipage}
}

\newpage
\normalsize
\h*{Fais comme l’oiseau}


\begin{bfseries}
[Refrain]:\\
Fais comme l'oiseau \\
Ça vit d'air pur et d'eau fraîche, un oiseau \\
D'un peu de chasse et de pêche, un oiseau \\
Mais jamais rien ne l'empêche, l'oiseau, d'aller plus haut \\
\end{bfseries}

Mais je suis seul dans l'univers \\
J'ai peur du ciel et de l'hiver \\
J'ai peur des fous et de la guerre \\
J'ai peur du temps qui passe, dis \\
Comment peut on vivre aujourd'hui \\
Dans la fureur et dans le bruit \\
Je ne sais pas, je ne sais plus, je suis perdu \\

\textbf{[Refrain]} \\

Mais l'amour dont on m'a parlé \\
Cet amour que l'on m'a chanté \\
Ce sauveur de l'humanité \\
Je n'en vois pas la trace, dis \\
Comment peut on vivre sans lui ? \\
Sous quelle étoile, dans quel pays ? \\
Je n'y crois pas, je n'y crois plus, je suis perdu \\

\textbf{[Refrain]} \\

Mais j'en ai marre d'être roulé \\
Par des marchands de liberté \\
Et d'écouter se lamenter \\
Ma gueule dans la glace, dis \\
Est-ce que je dois montrer les dents ? \\
Est-ce que je dois baisser les bras ? \\
Je ne sais pas, je ne sais plus, je suis perdu \\

\textbf{[Refrain]} 

\newpage
\normalsize
\h*{Education sentimentale}

\begin{multicols}{2}
Ce soir à la brume \\
Nous irons, ma brune \\
Cueillir des serments \\
Cette fleur sauvage \\
Qui fait des ravages \\
Dans les cœurs d'enfants \\
Pour toi, ma princesse \\
J'en ferai des tresses \\
Et dans tes cheveux \\
Ces serments, ma belle \\
Te rendront cruelle \\
Pour tes amoureux \\

Demain à l'aurore \\
Nous irons encore \\
Glaner dans les champs \\
Cueillir des promesses \\
Des fleurs de tendresse \\
Et de sentiment \\
Et sur la colline \\
Dans les sauvagines \\
Tu te coucheras \\
Dans mes bras, ma brune \\
Eclairée de lune \\
Tu te donneras \\

C'est au crépuscule \\
Quand la libellule \\
S'endort au marais \\
Qu'il faudra, voisine \\
Quitter la colline \\
Et vite rentrer \\
Ne dis rien, ma brune \\
Pas même à la lune \\
Et moi, dans mon coin \\
J'irai solitaire \\
Je saurai me taire \\
Je ne dirai rien \\

Ce soir à la brume \\
Nous irons, ma brune \\
Cueillir des serments \\
Cette fleur sauvage \\
Qui fait des ravages \\
Dans les cœurs d'enfants \\
Pour toi, ma princesse \\
J'en ferai des tresses \\
Et dans tes cheveux \\
Ces serments, ma belle \\
Te rendront cruelle \\
Pour tes amoureux
\end{multicols}

\newpage
\normalsize
\h*{Un petit gamin}

\begin{enumerate}
\item Un petit gamin, à la mine légère \\
Par ses parents était bien fort gâté. \\
Tout en faisant l’école buissonnière, \\
En chemin il se mit à chanter. \\

\begin{bfseries}
[Refrain]:\\
Oh la, oh lala, oh la, oh la, oh lala \\
Oh la, oh lala, oh la, oh lala. \\
\end{bfseries}

\item Dans un verger, notre petit bonhomme \\
Vit un pommier et se laissa tenter. \\
Tout en bourrant ses deux poches de pommes, \\
En chemin, il se mit à manger. \\

\textbf{[Refrain]: Miam…} \\

\item e garde champêtre, caché derrière un arbre, \\
Envoi son chien comme après un voleur. \\
Le chien l’attrape par le fond de la culotte, \\
En chemin, il se mit à crier \\

\textbf{[Refrain]: Aïe…} \\

\item Le lendemain, à l’école du village,
Le maître lui dit : te voilà bien puni,
A l’avenir, promets-moi d’être sage,
Et chante en chœur avec tous tes amis :

\textbf{[Refrain]: Oh la…} \\

\textbf{[Refrain]: Miam…} \\

\textbf{[Refrain]: Aïe…}
\end{enumerate}

\newpage
\large
\h*{La mer}

Il avait une fleur entre les dents, entre les dents, \\
Il avait dans le cœur l'amour du large et des grands vents, \\
Et quand il embarquait par trop gros temps, il nous disait : \\
"Nous, on n'a pas de cimetière, on a la mer" \\

Quand il ne revint pas un beau matin, un beau matin, \\
Les femmes ne pleuraient pas, elles joignaient seulement leurs mains, \\
Les hommes ne disaient rien mais ils pensaient : "Un jour, qui sait... \\
Nous, on n'a pas de cimetière, on a la mer" \\

Lalalala… \\
Et quand la vague un jour nous ramena une fleur, \\
On l'a compris, qu'il nous laissait encore son coeur. \\

Il avait une fleur entre les dents, entre les dents, \\
Il avait dans le coeur l'amour du large et des grands vents, \\
Quand nous parlons de lui, tout en riant, on se redit : \\
"Nous, on n'a pas de cimetière, on a la mer".

\vbox{
\begin{minipage}[t][0.4\textheight][t]{\textwidth}
\h*{Ecoute dans le vent}
\small
\begin{multicols}{2}
Combien de routes un garçon doit-il faire \\
Avant qu'un homme il ne soit ? \\
Combien l'oiseau doit-il franchir de mers \\
Avant de s'éloigner du froid ? \\
Combien de morts un canon peut-il faire \\
Avant que l'on oublie sa voix ? \\


\begin{bfseries}
[Refrain:]\\
Ecoute mon ami \\
Ecoute dans le vent \\
Ecoute, la réponse dans le vent. \\
\end{bfseries}

Combien de fois doit-on lever les yeux \\
Avant que de voir le soleil ? \\
Combien d'oreilles faut-il aux malheureux \\
Avant d'écouter leurs pareils ? \\
Combien de pleurs faut-il à l'homme heureux \\
Avant que son cœur ne s'éveille ? \\

\textbf{[Refrain]}\\

Combien d'années faudra t-il à l'esclave \\
Avant d'avoir sa liberté, \\
Combien de temps un soldat est-il brave \\
Avant de mourir oublié ? \\
Combien de mers franchira la colombe \\
Avant que nous vivions en paix ? \\

Ecoute mon ami \\
Ecoute dans le vent \\
Ecoute, la réponse dans le vent \\
Ecoute, la réponse est dans le vent.


\end{multicols}
\end{minipage}
\vspace{0.08\textheight}
\begin{minipage}[t][0.65\textheight][t]{\textwidth}
\h*{Le bon Dieu s’énervait}
\begin{multicols}{2}
\small

Le bon Dieu s'énervait dans son atelier. \\
"Ça fait trois ans déjà que j'ai planté cet arbre \\
Et j'ai beau l'arroser à longueur de journée, \\
Il pousse encore moins vite que ma barbe." \\

Pour faire un arbre, Dieu que c'est long. (3×) \\
Pour faire un arbre, mon Dieu que c'est long. \\

Le bon Dieu s'énervait dans son atelier. \\
"Sur ce maudit baudet, dix ans j'ai travaillé. \\
Je n'arrive pas à le faire avancer \\
Et encore moins à le faire reculer." \\

Pour faire un âne, Dieu que c'est long. (3×) \\
Pour faire un âne, mon Dieu que c'est long. \\
\columnbreak

Le bon Dieu s'énervait dans son atelier \\
En regardant Adam marcher à quatre pattes. \\
"Et pourtant, nom d'une pipe, j'avais tout calculé \\
Pour qu'il marche sur ses deux pieds." \\

Pour faire un homme, Dieu que c'est long. (3×) \\
Pour faire un homme, mon Dieu que c'est long. \\

Le bon Dieu s'énervait dans son atelier \\
En regardant le monde qu'il avait fabriqué. \\
"Les gens se battent comme des chiffonniers \\
Et je ne peux plus dormir en paix." \\

Pour faire un monde, Dieu que c'est long. (3×) \\
Pour faire un monde, mon Dieu que c'est long.

\end{multicols}
\end{minipage}
}

\newpage
\normalsize
\h*{Le Sud}
C'est un endroit qui ressemble à la Louisiane \\
A l'Italie \\
Il y a du linge étendu sur la terrasse \\
Et c'est joli \\

On dirait Le Sud \\
Le temps dure longtemps \\
Et la vie sûrement \\
Plus d'un million d'années \\
Et toujours en été. \\

Il y a plein d'enfants qui se roulent sur la pelouse \\
Il y a plein de chiens \\
Il y a même un chat, une tortue, des poissons rouges \\
Il ne manque rien \\

On dirait Le Sud \\
Le temps dure longtemps \\
Et la vie sûrement \\
Plus d'un million d'années \\
Et toujours en été. \\

Un jour ou l'autre il faudra qu'il y ait la guerre \\
On le sait bien \\
On n'aime pas ça, mais on ne sait pas quoi faire \\
On dit c'est le destin \\

Tant pis pour Le Sud \\
C'était pourtant bien \\
On aurait pu vivre \\
Plus d'un million d'années \\
Et toujours en été.

\newpage
\normalsize
\h*{Le petit âne gris}

Ecoutez cette histoire, que l'on m'a racontée. \\
Du fond de ma mémoire, je vais vous la chanter. \\
Elle se passe en Provence, au milieu des moutons, \\
Dans le sud de la France, au pays des santons. \\

Quand il vint au domaine, y avait un beau troupeau. \\
Les étables étaient pleines de brebis et d'agneaux. \\
Marchant toujours en tête aux premières lueurs, \\
Pour tirer sa charrette, il mettait tout son cœur. \\

Au temps des transhumances, il s'en allait heureux, \\
Remontant la Durance, honnête et courageux \\
Mais un jour, de Marseille, des messieurs sont venus. \\
La ferme était bien vieille, alors on l'a vendue. \\

Il resta au village, tout le monde l'aimait bien, \\
Vaillant, malgré son âge et malgré son chagrin. \\
Image d'évangile, vivant d'humilité, \\
Il se rendait utile auprès du cantonnier. \\

Cette vie honorable, un soir, s'est terminée. \\
Dans le fond d'une étable, tout seul il s'est couché. \\
Pauvre bête de somme, il a fermé les yeux. \\
Abandonné des hommes, il est mort sans adieux. \\

Mm mm mmm mm... \\
Cette chanson sans gloire \\
Vous racontait la vie, vous racontait l'histoire \\
D'un petit âne gris...

\newpage
\normalsize
\h*{Je cherche fortune}

\begin{bfseries}
[Refrain:]\\
Je cherche fortune \\
Autour du « Chat Noir » \\
Au clair de la lune \\
À Montmartre le soir \\
\end{bfseries}

\begin{multicols}{2}
\begin{enumerate}
\item Chez l'boulanger \\
Fais-moi crédit \\
J'ai pas d'argent \\
J'paierai sam'di \\
Si tu n'veux pas \\
M'donner du pain \\
J'te fourre la tête \\
Dans ton pétrin \\

\item Chez le boucher ... \\
si tu n’veux pas \\
M'donner d'gigot \\
J'te fourre la tête \\
Dans ton frigo \\

\item Chez l'cordonnier ... \\
si tu n’veux pas \\
M'donner d'sabots \\
J'te fourre la tête \\
Sous ton manteau \\

\item Chez l'pharmacien ... \\
si tu n’veux pas \\
M'donner d'aspro \\
J'te fourre la tête \\
Dans tes bocaux \\

\item Chez M'sieur l'Curé ... \\
Me confesser \\
si tu n’veux pas \\
J'te fourre la tête \\
Dans l'bénitier \\

\item Chez Monsieur l'Maire ... \\
si tu n’veux pas \\
Me marier \\
J'te fourre la tête \\
Dans l'encrier
\end{enumerate}
\end{multicols}

\newpage
\normalsize
\h*{L’aventure}

\setlength{\leftskip}{2cm}
\begin{bfseries}
[Refrain:]\\
L'aventure commence à l'aurore \\
A l'aurore de chaque matin \\
L'aventure commence alors \\
Que la lumière nous lave les mains \\

L'aventure commence à l'aurore \\
Et l'aurore nous guide en chemin \\
L'aventure c'est le trésor \\
Que l'on découvre à chaque matin \\

Pour Martin c'est le fer sur l'enclume \\
Pour César le vin qui chantera \\
Pour Yvon c'est la mer qu'il écume \\
C'est le jour qui s'allume \\
C'est le blé que l'on bat \\

L'aventure commence à l'aurore \\
A l'aurore de chaque matin \\
L'aventure commence alors \\
Que la lumière nous lave les mains \\
\end{bfseries}

\setlength{\leftskip}{0pt}
\begin{multicols}{2}
Tout ce que l'on cherche à redécouvrir \\
Fleurit chaque jour au coin de nos vies \\
La grande aventure il faut la cueillir \\
Entre notre église et notre mairie \\
Entre la barrière du grand-père Machin \\
Et le bois joli de monsieur l'Baron \\
Et entre la vigne de notre voisin \\
Et le doux sourire de la Madelon \\
La Madelon… \\

\textbf{[Refrain]}\\

\columnbreak

Tous ceux que l'on cherche à pouvoir aimer \\
Sont auprès de nous et à chaque instant \\
Dans le creux des rues, dans l'ombre des prés \\
Au bout du chemin, au milieu des champs \\
Debout dans le vent et semant le blé \\
Pliés vers le sol, saluant la terre \\
Assis près des vieux et tressant l'osier \\
Couchés au soleil, buvant la lumière \\
Dans la lumière…
\end{multicols}

\newpage
\normalsize
\h*{L’auvergnat}

\begin{multicols}{2}
Elle est à toi cette chanson \\
Toi l'Auvergnat qui sans façon \\
M'as donné quatre bouts de bois \\
Quand dans ma vie il faisait froid \\
Toi qui m'as donné du feu quand \\
Les croquantes et les croquants \\
Tous les gens bien intentionnés \\
M'avaient fermé la porte au nez \\
Ce n'était rien qu'un feu de bois \\
Mais il m'avait chauffé le corps \\
Et dans mon âme il brûle encore \\
A la manière d'un feu de joie \\


\begin{bfseries}
[Refrain:]\\
Toi l'Auvergnat quand tu mourras \\
Quand le croqu'mort t'emportera \\
Qu'il te conduise à travers ciel \\
Au père éternel \\
\end{bfseries}

Elle est à toi cette chanson \\
Toi l'hôtesse qui sans façon \\
M'as donné quatre bouts de pain \\
Quand dans ma vie il faisait faim \\
Toi qui m'ouvris ta huche quand \\
Les croquantes et les croquants \\
Tous les gens bien intentionnés \\
S'amusaient à me voir jeûner \\
Ce n'était rien qu'un peu de pain \\
Mais il m'avait chauffé le corps \\
Et dans mon âme il brûle encore \\
A la manière d'un grand festin \\

\textbf{[Refrain]}\\

Elle est à toi cette chanson \\
Toi l'étranger qui sans façon \\
D'un air malheureux m'as souri \\
Lorsque les gendarmes m'ont pris \\
Toi qui n'as pas applaudi quand \\
Les croquantes et les croquants \\
Tous les gens bien intentionnés \\
Riaient de me voir emmener \\
Ce n'était rien qu'un peu de miel \\
Mais il m'avait chauffé le corps \\
Et dans mon âme il brûle encore \\
A la manière d'un grand soleil \\

\textbf{[Refrain]}
\end{multicols}

\newpage
\footnotesize

\begin{multicols}{2}
\vbox{\begin{minipage}[t][0.48\textheight][b]{0.5\textwidth}
\h*{As-tu vu la vache?}
As-tu vu la vache, la vache aux yeux bleus ? \\
Toujours à la tâche, elle faisait «meuh ! meuh !» \\
Avec sa p’tite queue nature terminée par un plumet, \\
Elle battait la mesure pendant qu’les oiseaux chantaient. \\

Tous les bœufs, tous les bœufs \\
Tous les bœufs aimaient la vache. \\
Mais la vache, mais la vache \\
N’en aimait aucun d’eux. \\

Elle aimait un taureau, Olé ! \\
Qu’elle avait vu à Bilbao. \\
A la foire aux bestiaux. \\
Qu’il était fort, qu’il était beau \\
C’était un vrai taureau costaud, Olé ! \\

Elle pleurait la vache : « Meuh ! » \\
Après son bien-aimé, \\
Qui était décédé \\
A la coco, à la riri, à la dada \\
A la corrida, Olé!

\end{minipage}}

\vbox{\begin{minipage}[t][0.2\textheight][b]{0.5\textwidth}
\h*{Hevenou shalom aleichem}
Hevenou shalom aleichem \\
Hevenou shalom aleichem \\
Hevenou shalom aleichem \\
Hevenou shalom, shalom, shalom aleichem.
\end{minipage}}

\vbox{\begin{minipage}[t][0.15\textheight][b]{0.5\textwidth}
\h*{Petrouchka}
C’est le marchand Petrouchka qui revient, lalala \\
D’or est rempli son sac et il est content, lalala \\
Quand ses chevaux fatigués auront bu, lalala \\
Jusqu’au matin il pourra rire et chanter.
\end{minipage}}

\vbox{\begin{minipage}[t][0.43\textheight][b]{0.5\textwidth}
\h*{Dans mon pays d’Espagne}
Dans mon pays d'Espagne, olé! (bis) \\
Y a un soleil comme ça (bis) \\

Dans mon pays d'Espagne, olé! (bis) \\
Y a des guitares comme ça (bis) \\
Y a un soleil comme ça (bis) \\

Dans mon pays d'Espagne, olé! (bis) \\
Y a des danseuses comme ça (bis) \\
Y a des guitares comme ça (bis) \\
Y a un soleil comme ça (bis) \\

… \\

Y a la mer... \\
Y a des taureaux... \\
Y a des corridas... \\
Y a des flamencos...
\end{minipage}}

\vbox{\begin{minipage}[t][0.30\textheight][b]{0.5\textwidth}
\h*{Qui peut faire de la voile?}
Qui peut faire de la voile sans vent \\
Qui peut ramer sans rames \\
Et qui peut quitter son ami \\
Sans verser de larmes \\

Je peux faire de la voile sans vent \\
Je peux ramer sans rames \\
Mais ne peux quitter mon ami \\
Sans verser de larmes
\end{minipage}}

\vbox{\begin{minipage}[t][0.15\textheight][b]{0.5\textwidth}
\h*{Vent frais}
Vent frais, vent du matin \\
Vent qui souffle au sommet des grands pins \\
Joie du vent qui passe \\
Allons dans le grand vent
\end{minipage}}

\end{multicols}

\newpage
\small
\vbox{
\begin{minipage}[t][0.5\textheight][t]{\textwidth}
\vspace{0.00\textheight}
\h*{Buvons un coup}
\begin{multicols}{2}
Buvons un coup ma serpette est perdue \\
Mais le manche, mais le manche \\
Buvons un coup ma serpette est perdue \\
Mais le manche est revenu ! \\

Buvons deux coups ma fourchette est tordue \\
Mais le manche, mais le manche \\
Buvons deux coups ma fourchette est tordue \\
Mais le manch' n'est pas perdu ! \\

Buvons trois coups ma lorgnette est fourchue \\
Mais la tige, mais la tige \\
Buvons trois coups ma lorgnette est fourchue \\
Mais la tige est bien dessus ! \\

Buvons quatre coups ma hachette est fendue \\
Mais le manche, mais le manche \\
Buvons quatre coups ma hachette est fendue \\
Mais le manche est bien dessus ! \\

Buvons cinq coups ma trompette est tordue \\
Mais la note, mais la note, \\
Buvons cinq coups ma trompette est tordue \\
Mais la note est revenue ! \\

Buvons six coups ma raquette est rompue \\
Mais le manche, mais le manche \\
Buvons six coups ma raquette est rompue \\
Mais le manche est bien dessus ! \\

Buvons sept coups ma chaussette est fourbue \\
Mais la jambe, mais la jambe \\
Buvons sept coups ma chaussette est fourbue \\
Mais la jambe est bien dessus ! \\

Buvons huit coups ma pincette est fondue \\
Mais le manche, mais le manche \\
Buvons huit coups ma pincette est fondue \\
Mais le manche est revenu !
\end{multicols}
\end{minipage}
\begin{minipage}[t][0.5\textheight][t]{\textwidth}
\vspace{0.00\textheight}
\h*{Ma liberté}
\begin{multicols}{2}
Ma liberté \\
Longtemps je t'ai gardée \\
Comme une perle rare \\
Ma liberté \\
C'est toi qui m'a aidé \\
À larguer les amarres \\
Pour aller n'importe où, pour aller jusqu'au bout des chemins de fortune \\
Pour cueillir, en rêvant, une rose des vents sur un rayon de lune \\

Ma liberté \\
Devant tes volontés \\
Mon âme était soumise \\
Ma liberté \\
Je t'avais tout donné \\
Ma dernière chemise \\
Et combien j'ai souffert \\
Pour pouvoir satisfaire tes moindres exigences \\
J'ai changé de pays, j'ai perdu mes amis pour gagner ta confiance \\

Ma liberté \\
Tu as su désarmer \\
Toutes mes habitudes \\
Ma liberté \\
Toi qui m'a fait aimer \\
Même la solitude \\
Toi qui m'as fait sourire \\
Quand je voyais finir une belle aventure \\
Toi qui m'as protégé quand j'allais me cacher pour soigner mes blessures \\

Ma liberté \\
Pourtant je t'ai quittée \\
Une nuit de Décembre \\
J'ai déserté les chemins écartés \\
Que nous suivions ensemble \\
Lorsque sans me méfier \\
Les pieds et poings liés, je me suis laissé faire \\
Et je t'ai trahi pour une prison d'amour et sa belle geôlière \\
Et je t'ai trahi pour une prison d'amour et sa belle geôlière
\end{multicols}
\end{minipage}
}

\newpage
\footnotesize

\vbox{
\begin{minipage}[t][0.53\textheight][t]{\textwidth}
\vspace{0.00\textheight}
\h*{Ma belle gazelle}
\begin{multicols}{2}
\begin{enumerate}
\item Un grand lion d'Afrique était amoureux \\
D'une romantique gazelle aux yeux bleus \\
Ce lion au cœur tendre, qui l'eut, qui l'eut dit \\
Pleurait dans la lande, chantait dans la nuit \\

\begin{bfseries}
[Refrain:]\\
Ma belle gazelle \\
Ma belle gazelle \\
Ma belle gazelle \\
C'est toi que je veux \\
Ma belle gazelle \\
Ma belle gazelle \\
Ma belle, c'est toi que je veux \\
\end{bfseries}

\item Sur mon territoire, sans risquer ta vie \\
Tu peux venir boire à l'eau de mon puits \\
C'est fini la guerre que l'on te faisait \\
Contre ma crinière, vient dormir en paix \\

\item Souviens-toi mon ange qu'au temps de Noé \\
Nous vivions ensemble sans nous disputer \\
à nous deux ma blonde on peut tout changer \\
Et refaire le monde, pour l'éternité \\

\item Il mit tant de flamme, dans son beau discours \\
Qu'même l'hippopotame en pleura d'amour \\
Les étoiles au ciel jaillirent de partout \\
Quand la demoiselle vint au rendez-vous \\

\item Les loups, les panthères, tous les rhinocéros \\
Et tous les dromadaires vinrent pour les noces \\
Ils se marièrent et de leur union \\
Naquirent, ma chère, des p'tits gazilions!
\end{enumerate}
\end{multicols}
\end{minipage}

\begin{minipage}[t][0.5\textheight][t]{\textwidth}
\vspace{0.00\textheight}
\h*{As-tu compté les étoiles?}
\begin{multicols}{2}


As-tu compté les étoiles \\
Et les astres radieux \\
Déployant aux nuits sans voile \\
Leur cortège dans les cieux ? \\
Dieu qui leur donna \\
La vie et l’éclat, \\
Dieu qui leur fixa \\
La course et le pas, \\
Sait aussi quel est leur nombre \\
Et ne les oublie pas. \\


As-tu compté les abeilles \\
Butinant parmi les fleurs, \\
Papillons, mouches vermeilles, \\
Sans soucis et travailleurs ? \\
Dieu qui les vêtit \\
Couleur paradis, \\
Dieu qui leur fournit \\
Vivres et logis, \\
Sait aussi quel est leur nombre \\
Et ne les oublie pas. \\

As-tu compté les fleurettes \\
Souriant au gai printemps, \\
Boutons d’or et pâquerettes, \\
Fleurs des bois et fleurs des champs ? \\
Celui qui leur fit \\
Ces riches habits, \\
Celui qui leur mit \\
Ce frais coloris, \\
Sait aussi quel est leur nombre \\
Et ne les oublie pas. \\

As-tu compté les nuées \\
Passant dans les champs du ciel \\
Et les gouttes de rosée \\
Aux reflets de l’arc-en-ciel ? \\
Dieu qui fit le temps \\
Sombre ou éclatant, \\
Les ruisseaux chantants \\
Et les flots grondants \\
Sait aussi quel est leur nombre \\
Et ne les oublie pas. \\

Sais-tu combien, sur la terre, \\
Vivent d’enfants comme toi, \\
Dans le luxe ou la misère, \\
Fils de pauvre ou fils de rois ? \\
Dieu les connaît tous \\
Dieu les aime tous \\
Dieu les garde tous \\
Et Dieu les veut tous. \\
Tu es aussi dans le nombre \\
De ceux qu’Il n’oublie pas.

\end{multicols}
\end{minipage}
}

\newpage
\normalsize
\h*{Fanchon}

Amis, il faut faire une pause \\
J'aperçois l'ombre d'un bouchon \\
Buvons à l'aimable Fanchon \\
Chantons pour elle quelque chose \\

\begin{bfseries}
[Refrain:]\\
Ah que son entretien est doux \\
Qu'il a de mérite et de gloire \\
Elle aime à rire, elle aime à boire \\
Elle aime à chanter comme nous \\
Elle aime à rire, elle aime à boire \\
Elle aime à chanter comme nous \\
Elle aime à rire, elle aime à boire \\
Elle aime à chanter comme nous \\
TROMPETTE! \\
\end{bfseries}

Fanchon, quoique bonne chrétienne \\
Fut baptisée avec du vin \\
Un Bourguignon fut son parrain \\
Une Bretonne sa marraine \\

\textbf{[Refrain]} \\

Fanchon préfère la grillade \\
A tous les mets plus délicats \\
Son teint prend un nouvel éclat \\
Quand on lui verse une rasade \\

\textbf{[Refrain]} \\

Fanchon ne se montre cruelle \\
Que lorsqu'on lui parle d'amour \\
Mais moi, je ne lui fais la cour \\
Que pour m'enivrer avec elle

\newpage
\large
\h*{Siffler sur la colline}
Je l'ai vu près d'un laurier, elle gardait ses blanches brebis \\
Quand j'ai demandé d'où venait sa peau fraîche elle m'a dit \\
C'est d'rouler dans la rosée qui rend les bergères jolies \\
Mais quand j'ai dit qu'avec elle je voudrais y rouler aussi \\

\begin{bfseries}
[Refrain:]\\
Elle m'a dit ... \\
Elle m'a dit d'aller siffler là-haut sur la colline \\
De l'attendre avec un petit bouquet d'églantines \\
J'ai cueilli des fleurs et j'ai sifflé tant que j'ai pu \\
J'ai attendu, attendu, elle n'est jamais venue \\
\end{bfseries}

Zaï zaï zaï zaï zaï, zaï zaï zaï zaï, zaï zaï zaï zaï (2×) \\

A la foire du village un jour je lui ai soupiré \\
Que je voudrais être une pomme suspendue à un pommier \\
Et qu'à chaque fois qu'elle passe elle vienne me mordre dedans \\
Mais elle est passée tout en me montrant ses jolies dents. \\

\textbf{[Refrain]} \\

Zaï zaï zaï zaï zaï, zaï zaï zaï zaï, zaï zaï zaï zaï (2×) \\

Oh oh, oh oh (2×) \\

\newpage
\normalsize
\h*{Les copains d’abord}

\begin{multicols}{2}

Non, ce n'était pas le radeau \\
De la Méduse, ce bateau \\
Qu'on se le dise au fond des ports \\
Dise au fond des ports \\
Il naviguait en père peinard \\
Sur la grand-mare des canards \\
Et s'appelait les Copains d'abord \\
Les Copains d'abord \\

Ses fluctuat nec mergitur \\
C'était pas d'la littérature \\
N'en déplaise aux jeteurs de sort \\
Aux jeteurs de sort \\
Son capitaine et ses matelots \\
N'étaient pas des enfants d'salauds \\
Mais des amis franco de port \\
Des copains d'abord \\

C'était pas des amis de luxe \\
Des petits Castor et Pollux \\
Des gens de Sodome et Gomorrhe \\
Sodome et Gomorrhe \\
C'était pas des amis choisis \\
Par Montaigne et La Boétie \\
Sur le ventre, ils se tapaient fort \\
Les copains d'abord \\

C'était pas des anges non plus \\
L'Évangile, ils l'avaient pas lu \\
Mais ils s'aimaient toutes voiles dehors \\
Toutes voiles dehors \\
Jean, Pierre, Paul et compagnie \\
C'était leur seule litanie \\
Leur Credo, leur Confiteor \\
Aux copains d'abord \\

Au moindre coup de Trafalgar \\
C'est l'amitié qui prenait l'quart \\
C'est elle qui leur montrait le nord \\
Leur montrait le nord \\
Et quand ils étaient en détresse \\
Qu'leurs bras lançaient des S.O.S \\
On aurait dit des sémaphores \\
Les copains d'abord \\

Au rendez-vous des bons copains \\
Y avait pas souvent de lapins \\
Quand l'un d'entre eux manquait à bord \\
C'est qu'il était mort \\
Oui, mais jamais, au grand jamais \\
Son trou dans l'eau n'se refermait \\
Cent ans après, coquin de sort \\
Il manquait encore \\

Des bateaux j'en ai pris beaucoup \\
Mais le seul qu'ait tenu le coup \\
Qui n'ait jamais viré de bord \\
Mais viré de bord \\
Naviguait en père peinard \\
Sur la grand-mare des canards \\
Et s'appelait les Copains d'abord \\
Les Copains d'abord
\end{multicols}


\normalsize

\vbox{
\begin{minipage}[t][0.45\textheight][t]{\textwidth}
\vspace{0.00\textheight}
\h*{Nous aimons vivre…}
\begin{multicols}{2}
Nous aimons vivre au fond des bois \\
Aller coucher sur la dure \\
La forêt nous dit de ses mille voix (bis) \\
Lance-toi dans la grande aventure. \\

Nous aimons vivre sur nos chevaux \\
Dans les plaines du Caucase \\
Emportés par leur rapide galop (bis) \\
Nous allons plus vite que Pégase. \\

Nous aimons vivre auprès du feu \\
Et chanter sous les étoiles \\
La nuit claire nous dit de ses mille feux (bis) \\
Sois gai lorsque le ciel est sans voile.
\end{multicols}
\end{minipage}

\begin{minipage}[t][0.5\textheight][t]{\textwidth}
\vspace{0.00\textheight}
\h*{Comme un soleil}
\begin{multicols}{2}
\begin{bfseries}
[Refrain:]\\
Comme un soleil, comme une éclaircie \\
Comme une fleur que l'on cueille entre les orties \\
Elle doit venir, comme vient le beau temps \\
Elle doit venir comme vient le printemps \\
\end{bfseries}

Demandez-moi tout ce que vous voulez \\
Et sans regrets je vous le donne \\
Mais dites-moi où je la trouverai \\
Celle qui comprendra, celle qui me dira \\
"Où que tu ailles je vais avec toi \\
Quel que soit le chemin, je te suis pas à pas" \\
Et s'il m'arrivait alors de tomber \\
C'est elle qui me relèverait \\

\textbf{[Refrain]}\\

Demandez-moi tout ce que vous voulez \\
De ne plus jamais voir personne \\
De renoncer aux parfums de l'été \\
Aux accords de guitare, aux fumées de la gloire \\
Demandez-moi de ne plus croire en rien \\
Pourvu que je la voie au bout de mon chemin \\
Demandez-moi tout ce que vous voulez \\
Mais dites-moi où la trouver \\

\textbf{[Refrain]}\\

\end{multicols}
\end{minipage}
}

\newpage
\normalsize

\vbox{

\begin{minipage}[t][0.5\textheight][t]{\textwidth}
\vspace{0.00\textheight}
\h*{Comme un soleil}
\begin{multicols}{2}
\begin{enumerate}

\item Viens mon petit gars, emboîte mon pas, \\
C’est si beau, c’est merveilleux quand on est à deux \\
Viens mon petit gars, emboîte mon pas, \\
C’est si beau, c’est merveilleux ce qui nous vient de Dieu \\

\begin{bfseries}
[Refrain:]\\
Partons, partons ensemble \\
Viens mon petit gars, viens mon petit gars \\
Partons, partons ensemble, viens, viens, viens \\
\end{bfseries}

\item Viens mon petit gars, il nous faut partir \\ \\
Il le faut pour découvrir les fleurs et les bois \\ \\
Viens mon petit gars, il nous faut partir \\ \\
Il le faut pour découvrir la nature sous nos pas \\ \\

\item Viens mon petit gars, rendons grâce à Dieu \\
Et prions-le pour tous ceux qui n’admirent pas \\
Viens mon petit gars, rendons grâce à Dieu \\
Et prions-le pour tous ceux dont les yeux ne voient pas

\end{enumerate}

\end{multicols}
\end{minipage}

\begin{minipage}[t][0.5\textheight][t]{\textwidth}
\vspace{0.00\textheight}
\h*{Trois esquimaux}
Trois esquimaux, autour d’un brasero \\
Ecoutaient l’un d’eux qui sur son banjo \\
Rythmait le mortel ennui \\
Du pays du soleil de minuit. \\
‘Y a pas de soleil en Alaska \\
Woudji, woudji, woudji, wa, wa, wa. \\
Sur la banquise pas de mimosas \\
Pas de petits moutons sautant sur le gazon \\
Pas de rutabagas et pas de bouillons gras. \\
Houm, bala houm, bala houm, bala houm, ba la la \\
Houm, bala houm, bala houm, bala houm.
\end{minipage}

}

\vbox{

\begin{minipage}[t][0.55\textheight][t]{\textwidth}
\vspace{0.00\textheight}
\h*{Le port de Tacoma}
\begin{multicols}{2}

C'est dans la cale qu'on met les rats, \\
houla la houla, \\
C'est dans la cale qu'on met les rats, \\
houla houla. \\

\begin{bfseries}
[Refrain:]\\
Parés à virer,
Les gars, faudrait y aller.
On s' repos'ra quand on arriv'ra
Dans le port de Tacoma.
\end{bfseries}

C'est dans la pipe qu'on met l'tabac, \\
houla la houla, \\
C'est dans la pipe qu'on met l'tabac, \\
houla houlala. \\

\textbf{[Refrain]}\\

C'est dans la mer qu'on met les mâts, \\
houla la houla, \\
C'est dans la mer qu'on met les mâts, \\
houla houlala. \\

\textbf{[Refrain]}\\

C'est dans la gueule qu'on se met l'tafia, \\
houla la houla, \\
C'est dans la gueule qu'on se met l'tafia, \\
houla houlala. \\

\textbf{[Refrain]}\\

Mais les filles, ça s'met dans les bras, \\
houla la houla, \\
Mais les filles, ça s'met dans les bras, \\
houla houlala. \\
\end{multicols}
\end{minipage}

\begin{minipage}[t][0.55\textheight][t]{\textwidth}
\vspace{0.00\textheight}
\h*{Santiano}
\begin{multicols}{2}
C’est un fameux trois-mâts fin comme un oiseau \\
Hisse et ho, Santiano ! \\
Si Dieu veut toujours droit devant, \\
Nous irons jusqu'à San Francisco. \\

Je pars pour de longs mois en laissant Margot. \\
Hisse et ho, Santiano ! \\
D'y penser j'avais le cœur gros \\
En doublant les feux de Saint-Malo. \\

Tiens bon la vague et tiens bon levent. \\
Hisse et ho, Santiano ! \\
Si Dieu veut toujours droit devant \\
Nous irons jusqu’à San Francisco. \\

On prétend que là-bas l'argent coule à flots. \\
Hisse et ho, Santiano ! \\
On trouve l'or au fond des ruisseaux. \\
J'en ramènerai plusieurs lingots. \\

Un jour, je reviendrai chargé de cadeaux. \\
Hisse et ho, Santiano ! \\
Au pays, j'irai voir Margot. \\
A son doigt, je passerai l'anneau. \\

Tiens bon la vague et tiens bon le vent. \\
Hisse et ho, Santiano ! \\
Sur la mer qui fait le gros dos, \\
Nous irons jusqu'à San Francisco.
\end{multicols}
\end{minipage}
}

\newpage
\small
\vbox{

\begin{minipage}[t][0.55\textheight][t]{\textwidth}
\vspace{0.00\textheight}
\h*{Les amoureux des bancs publics}
\begin{multicols}{2}
Les gens qui voient de travers \\
Pensent que les bancs verts \\
Qu'on voit sur les trottoirs \\
Sont faits pour les impotents ou les ventripotents \\
Mais c'est une absurdité \\
Car à la vérité \\
Ils sont là c'est notoire \\
Pour accueillir quelque temps les amours débutants \\

\begin{bfseries}
[Refrain:]\\
Les amoureux qui s'bécott'nt sur les bancs publics \\
Bancs publics, bancs publics \\
En s'fouttant pas mal du regard oblique \\
Des passants honnêtes \\
Les amoureux qui s'bécott'nt sur les bancs publics \\
Bancs publics, bancs publics \\
En s'disant des "Je t'aime" pathétiques \\
Ont des p'tit's gueul' bien sympatiques \\
\end{bfseries}

Ils se tiennent par la main \\
Parlent du lendemain \\
Du papier bleu d'azur \\
Que revêtiront les murs de leur chambre à coucher \\
Ils se voient déjà doucement \\
Ell' cousant, lui fumant \\
Dans un bien-être sûr \\
Et choisissent les prénoms de leur premier bébé \\

\textbf{[Refrain]}\\

Quand les mois auront passé \\
Quand seront apaisés \\
Leurs beaux rêves flambants \\
Quand leur ciel se couvrira de gros nuages lourds \\
Ils s'apercevront émus \\
Qu' c'est au hasard des rues \\
Sur un d'ces fameux bancs \\
Qu'ils ont vécu le meilleur morceau de leur amour \\

\textbf{[Refrain]}\\

Quand la saint' famill' machin \\
Croise sur son chemin \\
Deux de ces malappris \\
Ell' leur décoche hardiment des propos venimeux \\
N'empêch' que tout' la famille \\
Le pèr', la mèr', la fille \\
Le fils, le Saint Esprit \\
Voudrait bien de temps en temps pouvoir s'conduir' comme eux

\end{multicols}
\end{minipage}


\begin{minipage}[t][0.55\textheight][t]{\textwidth}
\vspace{0.00\textheight}
\h*{A Paris l’est une vieille}
\begin{multicols}{2}
\end{multicols}
\end{minipage}

}
\end{document}
